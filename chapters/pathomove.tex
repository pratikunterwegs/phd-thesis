%************************************************
\chapter{Rapid Evolution of Movement Strategies Following Pathogen Introduction}\label{ch:pathomove}
%************************************************

\noindent \textbf{Pratik R. Gupte}, Gregory F. Albery\textsuperscript{1}, Jakob R.L. Gismann, and Franz J. Weissing

\marginpar{
    \textsuperscript{1} Wissenschaftskolleg zu Berlin, Germany.
}

\section*{Abstract}

\small{
    Animal social interactions always have a spatial context, and are the outcomes of evolved strategies that balance the costs and benefits of being sociable.
    We examine how animals balance the risk of pathogen transmission against the benefits of social information about resource patches, and the consequences for the emergent structure of animal social networks.
    We study a scenario in which an undetectable yet fitness-reducing infectious pathogen spills over into a population which has initially evolved movement rules in its absence.
    Pathogen spillover leads to a rapid evolutionary shift in animal social-movement strategies.
    The post-spillover strategy mix is controlled by a combination of landscape productivity and disease cost.
    Generally, animals adopt a dynamic social distancing approach, trading more movement (and less intake) for lower infection risk.
    Post-spillover populations are more widely dispersed over the landscape, and thus have less clustered social networks than their pre-spillover ancestors.
    Simple network epidemiological models show that diseases do indeed spread more slowly through pathogen-adapted animal societies.
    Our model suggests how the introduction of an infectious pathogen to a population rapidly changes social structure even when infections are undetectable, and how such events might make populations more resilient to future disease outbreaks.
    Overall, we offer both a general modelling framework and initial predictions for the evolutionary consequences of wildlife pathogen spillovers.

    \medskip

    % \noindent {\large{\color{Maroon}$\Delta$}} Published in the \textit{Journal of Animal Ecology} as Gupte et al. (2021). A guide to pre-processing high throughput tracking data.
}

\clearpage