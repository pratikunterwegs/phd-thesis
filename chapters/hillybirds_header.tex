\begin{refsection}

%************************************************
\chapter{Land-cover and Climate Shape Bird Distributions in a Tropical Biodiversity Hotspot}\label{ch:hillybirds}
\chaptermark{Citizen Science \& Bird Distributions}
%************************************************
% Using citizen science to parse climatic and land cover influences on

{\noindent Vijay Ramesh\textsuperscript{1}, \textbf{Pratik R. Gupte}, Morgan Tingley\textsuperscript{2}, V.V. Robin\textsuperscript{3}, and Ruth S. de Fries\textsuperscript{1}}

\graffito{
    \textsuperscript{1} Columbia University, USA.
    
    \medskip
    
    \textsuperscript{2} University of California --- Los Angeles, USA.
    
    \medskip

    \textsuperscript{3} Indian Institute for Science Education and Research --- Tirupati, India.
}

\section*{Abstract}

\small{
    Disentangling associations between species occupancy and its environmental drivers --- climate and land cover --- along tropical mountains is imperative to predict species distributional changes in the future.
    Previous studies have primarily focused on identifying such associations in temperate mountain systems.
    Using 1.29 million robustly processed citizen science observations contributed to eBird between 2013 and 2021, we examined the role of climatic and landscape variables and its association with bird species occurrence within a tropical biodiversity hotspot, the southern Western Ghats in India.
    Using an occupancy modeling framework, we found that temperature seasonality, precipitation seasonality, and the proportion of evergreen forests were significantly associated with species-specific probabilities of occupancy for 78\% (n = 43 birds), 38\% (n = 21 birds), and 27\% (n = 15 birds) of bird species examined, respectively.
    Our study shows that several forest birds (n = 18 species) were negatively associated with temperature seasonality, highlighting narrow thermal niches for such species.
    The probability of occupancy of six forest species and eight generalist species was positively associated with precipitation seasonality, indicating potential associations between rainfall and resource availability, and thereby, species occurrence.
    A smaller number of largely generalist species (n = 9 birds) were positively associated with human-modified land cover types --- including the proportion of agriculture/settlements and plantations.
    Our study shows that rigorously filtered citizen science observations can be used to identify associations between environmental drivers and species occupancy on tropical mountains.
    Though current distributions of tropical montane birds of the Western Ghats are strongly associated with climatic factors (mainly, temperature seasonality), naturally occurring land cover types (forests) are critical to sustaining montane avifauna across human-modified landscapes in the long run.

    \bigskip

    {\noindent \large{$\Delta$}} \normalfont Under review at \textit{Ecography} as Ramesh et al. ``Using citizen science to parse climatic and land cover influences on bird occupancy within a tropical biodiversity hotspot''.
}

\clearpage

