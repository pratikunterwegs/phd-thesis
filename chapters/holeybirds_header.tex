%************************************************
\chapter{Direct Effects of Flight Feather Moult on Bird Movement and Habitat Selection}\label{ch:holeybirds}
\chaptermark{Wing Moult \& Movement}
%************************************************

{\noindent \textbf{Pratik R. Gupte}, Yosef Kiat\textsuperscript{1}, Sivan Toledo\textsuperscript{2}, and Ran Nathan\textsuperscript{3}}

\marginpar{
    
    \textsuperscript{1} University of Haifa, Israel.
    
    \medskip
    
    \textsuperscript{2} Tel Aviv University, Israel.
    
    \medskip

    \textsuperscript{3} The Hebrew University of Jerusalem, Israel.
}

\section*{Abstract}

\small{
    The flight surfaces of bird wings require regular renewal through a process called moult --- shedding worn out feathers and growing fresh ones.
    moult presents birds with the dilemma of needing more resources for feather growth just when their flight capacity is reduced due to feather loss, making them vulnerable to predation.
    We combined mechanistic and experimental approaches to present a first quantification of the direct effects of wing moult on the movement and habitat selection of four non-migratory passerine species. 
    We followed the movement of birds in different moult stages using a high-throughput tracking system. 
    Taking a viewshed ecology approach, we examined how birds used areas sheltered from observation by potential predators.
    We found that species' moult strategies determined whether they adjusted their movement to their wing condition. 
    Among species that adapted movement to moult status, natural moult led to increased movement between habitat patches, but artificial feather removal led to shorter between-patch movements. 
    Across moult status, birds preferred low-visibility areas that are sheltered from predators.
    Our work demonstrates how birds’ fine-scale adaptive movement decisions are intertwined with their evolved physiological strategies, and that birds can adopt their predators’ spatial perspectives at landscape scales, pre-emptively avoiding areas where they could be observed.
    Overall, we show how combining experimental and tracking approaches with mechanistic, biologically-grounded estimates of landscape attributes allows cross-species comparisons of movement strategies.

    \bigskip

    {\noindent \large{$\Delta$}} \normalfont A manuscript in preparation for \textit{PNAS}.

    % Code to replicate the analyses is on Zenodo at: \textit{\color{red}Zenodo link}, and on Github at: 
    % \hyperlink{github.com/pratikunterwegs/holeybirds}{github.com/pratikunterwegs/holeybirds}.
    % Tracking data are available from DataverseNL as a draft: \textit{\color{red}draft data link}.
    % Data will be at this persistent link after publication: \textit{\color{red}persistent data link}.
}

\clearpage
