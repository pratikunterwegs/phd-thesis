%************************************************
\chapter{Direct Effects of Flight Feather Moult on Bird Movement and Habitat Selection}\label{ch:holeybirds}
\chaptermark{Wing Moult \& Movement}
%************************************************

{\noindent \textbf{Pratik R. Gupte}, Yosef Kiat\textsuperscript{1}, Yoav Barton\textsuperscript{1}, Anat Levi\textsuperscript{1}, Ulrike Schl{\"a}gel\textsuperscript{2}, Johannes Signer\textsuperscript{3}, Sivan Toledo\textsuperscript{4}, and Ran Nathan\textsuperscript{1}}

\graffito{
    
    \textsuperscript{1} The Hebrew University of Jerusalem, Israel.
    
    \medskip
    
    \textsuperscript{2} University of Potsdam, Germany.
    
    \medskip

    \textsuperscript{3} University of G{\"o}ttingen, Germany.

    \medskip

    \textsuperscript{4} Tel-Aviv University.
}

\section*{Abstract}

% \small{
    The flight surfaces of bird wings require regular renewal through a process called moult --- shedding worn out feathers and growing fresh ones.
    Moult presents birds with the dilemma of needing more resources for feather growth just when their flight capacity is reduced due to feather loss, making them more vulnerable to predation.
    We combined mechanistic and experimental approaches to present a first quantification of the direct effects of wing moult on the movement and habitat selection of four non-migratory passerine species. 
    We followed the movement of moulting birds using a high-throughput tracking system. 
    Taking a viewshed ecology approach, we examined how birds used areas sheltered from observation by potential predators.
    We found that species' moult rate determined whether they adjusted their movement to their wing condition.
    Among species that adapted movement to moult rate, natural moult led to increased movement between habitat patches, whereas artificial feather removal led to shorter between-patch movements. 
    Across moult rates, birds preferred lower visibility areas that are more sheltered from visual predators.
    Our study revealed that birds' fine-scale adaptive movement decisions are intertwined with their evolved physiological strategies, and they can adopt the spatial perspective of their predators at landscape scales, pre-emptively avoiding areas where they could be observed.
    Overall, we show how combining experimental and tracking approaches with mechanistic, biologically-grounded estimates of landscape attributes allows cross-species comparisons of movement strategies in response to moult dynamics.

    \bigskip

    {\noindent \large{$\Delta$}} \normalfont A manuscript under review at \textit{PNAS}.

    % Code to replicate the analyses is on Zenodo at: \textit{\color{red}Zenodo link}, and on Github at: 
    % \hyperlink{github.com/pratikunterwegs/holeybirds}{github.com/pratikunterwegs/holeybirds}.
    % Tracking data are available from DataverseNL as a draft: \textit{\color{red}draft data link}.
    % Data will be at this persistent link after publication: \textit{\color{red}persistent data link}.
% }

\clearpage
