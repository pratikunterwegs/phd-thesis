
\phantomsection
\addtocontents{toc}{\protect\vspace{\beforebibskip}}%
% \addcontentsline{toc}{chapter}{\tocEntry{\color{black}\itshape{General Introduction: The Current Frontiers of Animal Movement Ecology}}}%
\chapter{Introduction: Current Frontiers in the Study of Animal Movement}\label{ch:introduction}
\chaptermark{Introduction}

{{Pratik R. Gupte}}

% \begin{center}
%     \emph{Coming back to where you started is not the same as never leaving.}\\
%     \medskip
%     -- \small{Terry Pratchett}
% \end{center}

% Movement is a fascinating phenomenon.
% It integrates a deep, implicit \textit{feel} for the fundamental organisation of the universe, with surprising agency: that things, colloquially speaking, need not be the same, or different.
% All animals move, whether actively or passively, at some stage of their lives.
% As humans, we have projected our own motivations on to animals around us, and arrived at a relatively good understanding of why animals move, i.e., the ecological drivers of animal movement.
% In this, we have the advantage of not being too far from our own past, before we had quite successfully insulated ourselves from the effects of such drivers.
% This allows us to take the perspective of other animals when looking at a landscape; essentially, to mentally \textit{model} movement decisions across it.

% {\color{red} WORK IN PROGRESS}

To be completed.

\vfill

\clearpage
