
\addtocontents{toc}{\protect\vspace{\beforebibskip}}%
% \addcontentsline{toc}{chapter}{\tocEntry{\color{black}\itshape{Synthesis: Linking the Ecology and Evolution of Animal Movement with Mechanistic, Individual Based Models}}}%
\chapter{Synthesis: Linking the Ecology and Evolution of Animal Movement with Mechanistic, Individual Based Models}\label{ch:discussion}
\chaptermark{Synthesis}

{\noindent \textbf{Pratik R. Gupte}}

\medskip

{\noindent \large{$\Delta$}} Adapted from a manuscript invited by \textit{Ecology Letters}.

\medskip

Movement is key to animal ecology across spatial and temporal scales, as nearly all ecological processes have an explicit spatial context \citep{nathan2008a}.
By moving, animals can track seasonal fluctuations in resources \citep{geremia2019,abrahms2021a}, and facilitate or avoid ecological interactions such as inter- and intra-specific competition (e.g. \cite{duckworth2007}), predation \citep[e.g.][]{kohl2018}, parasitism \citep{weinstein2018}, mating displays and breeding site selection \citep{kempenaers2017}, as well as the transmission of animal culture \citep{jesmer2018,klump2021}, or infectious pathogens \citep[][see also Chapter~\ref{ch:pathomove}]{weinstein2018,monk2022,stroeymeyt2018}.
Mobile animals engineer their local environments, transferring nutrients \citep{leroux2018}, inducing local green-up \citep{geremia2019}, and affecting the distributions of overlapping species \citep[see e.g.][]{kohl2018,leroux2018,duckworth2007,monk2022}.
Perturbed natural regimes of animal movements (e.g. due to climate change), can severely impact humans, from direct conflict \citep{abrahms2021} to potential pathogen outbreaks \citep{carlson2022a,wille2022}; conversely, natural distributions of wildlife could aid climate mitigation by regulating key biotic and abiotic processes \citep{schmitz2018,malhi2022}.
Animal movement is thus crucial to a sound understanding of ecological processes and patterns generally \citep{jeltsch2013,schlagel2020}.

\section*{Reflections on Part~\ref{part:eco}}

\textbf{Part~\ref{part:eco}} of this thesis took an empirical approach to animal movements and distributions.
Animal movement ecology has benefited greatly from the adoption of advanced animal tracking technology, and especially from the proliferation of GPS loggers \citep{cagnacci2010}.
Yet the majority of species of birds and mammals (leave alone reptiles or amphibians) cannot currently be tracked because most high-resolution loggers are much too heavy for them to bear safely \parencite{kays2015}.
High-throughput tracking systems --- which we described in \textcite{nathan2022} --- such as ATLAS with its lightweight tags, can allow researchers to achieve at regional scales, a far more detailed understanding of animal movement than sought by \textcite{wikelski2007} when floating the idea of ICARUS (see now \cite{jetz2022}).
Yet data from these systems is not as conservatively `cleaned' as that from GPS tracking, and this is because the original end uses of each of these systems are very different.
In Chapter~\ref{ch:preprocessing}, I showed how a set of simple techniques and workflows can be used to substantially improve the quality of raw ATLAS data.
\textcite[]{beardsworth2022mee} have now shown that the accuracy of ATLAS systems (in this case, the Wadden Sea ATLAS system; \cite{bijleveld2021}) --- after applying these cleaning methods --- is comparable to GPS tracking, but with a much higher sampling rate.
An interconnected network of such high-throughput systems could represent one option for how animals could be tracked at high spatio-temporal resolution at large, continental scales \citep{nathan2022}.

Meanwhile, even data from earlier iterations of high-throughput systems can be made usable by robust filtering and cleaning.
In Chapter~\ref{ch:holeybirds}, I used data from the original ATLAS system deployed in the Hula Valley in Israel, to study how moult --- the loss and regrowth of flight feathers --- affects bird movement and habitat selection.
% \graffito{
%     Before beginning this project, my primary experience with bird moult had been during my master's thesis fieldwork on Kolguev island in the Russian Arctic in 2016.
%     Geese breed there in the summer, undergoing full wing moult, which created a small window where they could be corralled into enclosures, to be tagged with GPS loggers.
% }
This work is supported by recent findings that birds' flight characteristics affect whether they will risk crossing even narrow open tracts, such as forest roads \citep{claramunt2022}.
The results here suggest that predation risk avoidance could be a possible mechanism by which some areas that appear productive become unsuitable for many bird species --- agricultural fields for instance provide little cover from aerial predators.
Birds have long been anecdotally known to avoid certain features such as water bodies despite being powerful flyers, to the extent that this has prevented entire taxa from colonising archipelagos \citep{diamond1981}.
This effect is now much better quantified by studying the migration of raptors across open water \citep{nourani2020}.
Both road- and water-crossing avoidance seem bizarre to human observers, possibly because as a terrestrial species, we subconsciously think of flight as a certain kind of invulnerability, whether from environmental hazards or active hunters.
It is much more likely that we simply do not --- and perhaps cannot --- really appreciate how complex flight is, and the many risks it holds.
One of the main conclusions of Chapter~\ref{ch:holeybirds} (and of Part~\ref{part:evo}), then, is the importance of adopting the perspective of the study species, the \textit{individual in its context}, when seeking to understand the short- and long-term drivers of animals' behaviour.

Individual movement and habitat selection scale up to produce species distributions and community composition \citep{schlagel2020a}.
In Chapter~\ref{ch:hillybirds}, we showed mixed results for the effects of climate and land cover types on the distributions of birds in the Nilgiri and Anamalai Hills of southern India, a biodiversity hotspot.
One drawback of individual-centric animal tracking studies has been small sample sizes due to the costs of tracking technologies; this means that most studies focus on tracking at most a few species.
Citizen science studies allow researchers to accomplish two key goals at once: \textit{(i)} harness the spatial and temporal spread of amateur naturalists to collect ecological data, potentially from dozens of species at once, and \textit{(ii)} in so doing, strengthen public participation in the scientific process.
% The findings are not as clear as in \textcite{freeman2018}, who showed that Andean birds were on the ``escalator to extinction''.
The current era of large ecological datasets being widespread is only about a decade old; for example, \textit{eBird} data from India span only a seven year period since 2015.
This makes it especially challenging to compare the drivers of current species distributions across regions with different histories of monitoring, or within a region that has not been long monitored.
The Nilgiri Hills offer a unique opportunity to integrate historical data from the 1820s onwards --- primarily birds shot as specimens by British colonial officials --- into modern species distribution analyses.
This project, for which I gathered data from bird specimens stored in the Natural History Museum of London's Tring collections, is currently ongoing.
When complete it could stand alongside efforts such as the Grinnell Resurvey \citep{tingley2009a,tingley2009b} in demonstrating how nearly two centuries of land cover and climatic change have affected species distributions.

\section*{Reflections on Part~\ref{part:evo}}

Given its importance to animal ecology, movement is often implicitly included in the cornerstone models of eco-evolutionary theory (group topology: \cite{hamilton1971}; resource exploitation: \cite{fretwell1970, charnov1976}; community assembly: \cite{macarthur1967}).
Yet we currently lack theory that links the short-term ecological drivers and outcomes of movement with its evolutionary causes --- essentially, there is no evolutionary extension to the `movement ecology paradigm' \citep{holyoak2008}.
This hinders insight into how intensified selection on mobile species due to global change \citep[e.g.][]{vangils2016} would affect animal movement and related phenomena.
% Ecological theory, in order to provide general insights, must necessarily simplify reality down to essential processes.
One such simplification has long been to consider movement to be a population-level property shared by all individuals.
Work on consistent behavioural differences in animals, including in movement, suggests that this assumption is not well supported \citep{spiegel2017,shaw2020,stuber2022,webber2018,webber2020,abrahms2017}.
Movement is also often modelled as either a random or an optimal process --- both of these assumptions are simplistic, and animals integrate many internal and external cues when making movement decisions \citep{nathan2008a}.
Individual-based simulation models (IBMs) are well suited to representing movement as a decision made after integrating multiple cues in complex ecological contexts \citep{huston1988,deangelis2019}, and can include simplified evolutionary dynamics \citep{getz2015,getz2016,netz2021}.
This in the class of models I have used in Part~\ref{part:evo} for broad conceptual insight into the evolution of animal movement strategies.

In Chapter~\ref{ch:kleptomove}, I showed how animal movement and competition strategies jointly evolve, using an individual-based model with 10,000 individuals moving about on a grid of over 250,000 cells --- among the largest IBMs in this field of study.
The model demonstrated a number of interesting outcomes that could form the basis for future work.
For instance, I showed that individual variation in preferences for environmental cues is reliably evolved in simple foraging contexts, without apparent trade-offs in foraging strategies, and that social information is key to moving and foraging in consumer populations.
When individuals can adopt a kleptoparasitic strategy, they may do so even when environmental cues indicate that a `producer' strategy \parencite{beauchamp2008} might be more suitable.
In this sense, certain competition and foraging strategies may actually represent `personalities' as they were originally conceived of --- suboptimal choices despite countervailing information \parencite{sih2004}.
Unlike other chapters in this thesis, I cannot be sure that this one will lead to substantial developments in eco-evolutionary theory, and see it more as a culmination of theory in the once-key field of foraging competition studies.
Chapter~\ref{ch:patternprocess} is a direct development of of Chapter~\ref{ch:kleptomove}.
Here, I adapted movement paths generated in Chapter~\ref{ch:kleptomove} to investigate popular statistical tools in movement ecology: repeatability analysis, and step-selection analysis.
Inferring processes (mechanisms) from observed patterns (phenomena) is a common pursuit in movement ecology.
My analysis shows that there are substantial risks to doing so naively --- spatial personalities \parencite{stuber2022} may actually result from underlying differences in movement and competition strategies.
This highlights the importance of a detailed natural history understanding of the study species and its ecological context.


Finally, Chapter~\ref{ch:pathomove}, I tackled a scenario that in expected to become increasingly common --- the transmission of novel pathogens from one species to another \citep{carlson2022a}.
Indeed currently the hitherto poorly known tropical African disease monkeypox is spreading across the globe, SARS-CoV-2 has seen multiple introductions to animals, including abundant wildlife \citep{kuchipudi2022}, and the H5N1 strain of avian influenza has been spreading through multiple temperate species \citep{wille2022}.
My relatively simple model of the trade-off between social information use (in a foraging context), and the risk of pathogen transmission generated clear predictions for how such novel pathogen introductions should affect the evolution of host sociality.
Worryingly, a cascading effect of decreased host sociality in most scenarios is poorer ecological performance in terms of harvesting resources from the landscape, leaving populations vulnerable to other environmental risks.
The potential consequences for other members of an ecological community --- albeit via the mechanism of mortality --- are borne out by \citet{monk2022}, who studied the effects of the introduction of mange to vicu\~{n}as in Patagonia.
The scenarios I modelled may actually be too mild, and novel pathogen spillovers could exterminate their hosts, rather than force the evolution of less gregarious social systems.
The scale of future work required in this field is daunting: identifying outbreaks as they happen, often in remote areas and involving poorly known species; determining patterns of species' spatial overlap that could aid cross-species transmission beyond the initial spillover; determining which species --- for a range of reasons --- may be at heightened risk from an epi- or panzootic outbreak; and finally, determining a response that preserves species while minimising risk of further spillover.

Theory rarely addresses the long-term, evolutionary causes or consequences of movement, despite adaptive reasoning underpinning many models \citep{charnov1976,fretwell1970}.
Evolutionary models of movement rules treat them as population properties (as in \cite*[]{dejager2011,dejager2020}, or \cite*[]{morris2011}), whereas movement is an individual-level outcome, and it is on individual outcomes that selection acts.
When individuals with different movement strategies have equivalent fitness, populations may show movement polymorphisms \citep{wolf2012,shaw2020,getz2015}.
Including evolutionary dynamics in movement models could thus provide initial predictions for when individual variation (with its own consequences; \cite{spiegel2017}) should be expected.
We could also gain insight into how movement strategies could possibly evolve under various ecological scenarios.
This second aspect is often ignored, possibly because evolution is considered too slow to be relevant to the understanding and management of ecological dynamics over a few decades.
This assumption is mistaken, as evolution can be both rapid and adaptive \citep[][]{bonnet2022}.

In this \textit{Synthesis}, I propose a framework for mechanistic models of intermediate complexity that integrate both the ecological dynamics of animal movement, and their evolutionary causes and consequences.
The key feature of such models is to let individual-level ecological outcomes in one generation influence which movement strategies are present in future generations, thus establishing a feedback loop between animals' evolutionary history and their current spatial ecology.
Specifically, I advocate that movement be modelled as a response to local cues rather than a random walk \citep[`mechanistic'; see][]{mueller2011}.
This is the approach I have taken in Chapters~\ref{ch:kleptomove} and \ref{ch:pathomove}.
In the models presented in those chapters, I used IBMs, in which individuals can be given different \emph{movement preferences} and thus make quite different decisions when presented with similar cues \citep{getz2015,white2018}.
Yet an open question when including such behavioural variation is whether the emergent outcomes may be transient phenomena that are quite different from the dynamics obtained on evolutionary timescales.
This is especially the case when modelling processes that have fitness consequences for some behavioural types (such as disease in \cite[]{white2018}).
Consequently, I additionally advocate for movement models to be embedded in an evolutionary context, with individuals' movement outcomes subject to selection, and their movement preferences subject to random change (mutation).
With a simple case study, I contrast the conceptual insights from my modelling approach, with those from currently widespread model implementations, with special reference to emergent patterns.


\section*{What Role for Models in Understanding the Evolution of Movement?}

While preparing \textcite{nathan2022}, I found that each time I tried to lay out my view for how evolutionary IBMs should be used in movement ecology, some other anonymous author would change it back to his own view.
I rather disagree with that view, which appears in the final paper, so I shall present that view first, and counter it with my own.
In \textcite{nathan2022}, we wrote,
\begin{quote}
    \emph{
        Using genetic algorithms, initial candidate rulesets for individual decision-making can evolve into a robust ruleset that is able to reproduce the unique range and quality of spatial and temporal patterns in high-throughput data (``reinforcement learning').
    }
\end{quote}
This approach seeks to recover patterns seen in real empirical data from simulations, with the hope that the simulated mechanisms (`candidate rulesets') that produced them are similar to those animating real individuals (`true mechanisms').
However, a wide range of mechanisms can produce very similar phenomena, making it difficult to determine whether the `true' mechanism is approximated by any of the candidate simulated mechanisms.
Essentially, it is challenging to determine processes from patterns (as I covered in Chapter~\ref{ch:patternprocess}).

Open questions remain about how rulesets, or mechanisms, should be encoded in models.
% Below I show how the choice of mechanism --- in this case, habitat selection rules --- can affect model conclusions quite substantially.
Habitat selection rules which are complex functions of the information available to individuals are likely to be challenging to interpret.
For example, movement decisions based on outputs computed by artificial neural networks were first proposed over a decade ago \parencite{mueller2011}, but they have not seen widespread adoption in the ecology and evolution literature (but see \cite{netz2021}).
Furthermore, it is also unclear how these mechanisms should undergo evolution --- in \textcite{nathan2022}, we suggested reinforcement learning acting on the simulated mechanisms, based on the similarity of simulated movement paths with real animal movements.
The concept of genetic algorithms is borrowed from the fields of artificial intelligence and computer science \parencite{deangelis2019}, and represents their idea of biological processes.
The solution-oriented approach of artificial intelligence is poorly suited to ecology and evolution, in which there \textit{are} no single correct solutions --- and in which, moreover, individuals interact not only with the environment, but also with each other.
Consequently, the implementation of genetic algorithms such as `simulated annealing' \parencite{getz2015}, is not a good choice for conceptual eco-evolutionary models.

I propose a different way forward: rather than working backwards from empirical phenomena to potential mechanisms, to instead \textit{work forwards from plausible mechanisms to potential emergent outcomes}.
This requires first accepting that IBMs are best suited to broad conceptual insight into `What if \ldots?' scenarios, such as in Chapter~\ref{ch:pathomove} (although detailed predictive frameworks do exist, e.g. \cite{bocedi2014}).
Second, I suggest beginning with plausible, well-supported movement mechanisms, such as individual perception and integration of local cues when making movement decisions \parencite{nathan2008a}.
% Of course, any biological mechanism is an emergent outcome of constituent sub-mechanisms, down to the molecular level; some abstraction is therefore necessary.
Having selected salient mechanisms, a plausible ecological context is also key --- a population foraging on a landscape is a solid starting point.
The main feature of these models, however, is to let the ecological outcomes for individuals in one generation (such as intake) determine the mixture of movement decision-making mechanisms in the next generation, through inheritance (with variation arising \textit{via} mutations; see below, or Chapters~\ref{ch:kleptomove},~\ref{ch:pathomove}).
For simplicity, as seen in the example models here, and in Chapters~\ref{ch:kleptomove} and~\ref{ch:pathomove}, some ecological and evolutionary aspects will have to be set aside.
% This is not to say that issues such as sexual selection, non-random mating, or flexible population sizes are not important, but rather that researchers should make their own judgments about which features of biological systems are important to their study.
In addition to an initial understanding of how mechanisms can lead to unexpected emergent outcomes, the class of models I advocate are well suited to examining how these emergent outcomes could change following perturbations in environmental regimes, as I do in Chapter~\ref{ch:pathomove} (see also \cite{botero2015}).

\section*{Insights from Mechanistic, Evolved Step Selection Compared with Random Walks and Optimal Movement}

Here, I show that considering movement as the outcome of evolved preferences for locally available cues leads to very different ecological outcomes when compared to mainstream frameworks such as random walks and optimal local movement.
These differences can be important when such models are used as baselines against which to compare patterns observed from empirical animal tracking data, or to make predictions for how key ecological processes --- such as the transmission of pathogens or culture --- occur in animal populations (pathogens: \cite{white2018,white2018b,cantor2021,scherer2020}; culture: \cite{romano2020,romano2021,cantor2021,cantor2021a}).
The key feature of the class of models I advocate is that \textit{(1)} individuals make decisions according to individual-specific preferences for available environmental cues, \textit{(2)} individuals' preferences for environmental cues, \textit{taken together}, form their behavioural strategy, and \textit{(3)} individuals' behavioural strategies have evolutionary consequences, i.e., the ecological outcomes in one generation determines the population mixture of behavioural strategies in the next generation.
Here I focus on movement preferences and movement strategies, which are among the behavioural strategies of individuals, and which may also facilitate or constrain which behavioural strategies individuals can employ \citep{nathan2008a,spiegel2017}.

I compare ecological outcomes in 20 replicates of four scenarios of a model with the same ecological processes, that are borrowed from the model in Chapter~\ref{ch:pathomove}.
In my model, 200 individuals inhabit a landscape of 30 square units, which also contains 450 discrete food items (see Fig.~\ref{fig:compare}A).
Food items are patchily distributed to form distinct clusters (N = 30, 15 items per cluster).
Individuals can move in increments of 1 distance unit (like a king in chess), and can perceive items ($F$) and other individuals at locations 1 distance unit away.
When individuals perceive a food item, they pick it up and must handle it for 5 time-steps until they can gain its energetic benefit \citep{ruxton1992,gupte2021a,gupte2022c}; we call such individuals `handlers' ($H$).
While individuals are handling an item, they are immobilised.
Individuals compete with each other exploitatively and an item once picked up by an individual is unavailable to its neighbours; these individuals continue searching for food, and we call them `non-handlers' ($N$).
Items regenerate at the same location after a fixed number of timesteps, which we call the regeneration time ($T_R$; default = 100), and while an item is regenerating, it cannot be sensed by nearby individuals.
Individuals have a lifetime of 400 timesteps, over which they forage and move over the landscape.
The model's four scenarios differ in their implementation of individual movement.

\subsection*{Random Walks and Optimal Movement}

In \textit{scenario 1}, individuals perform a random walk, and have a uniform probability of either remaining in their current location, or moving in a direction chosen from among eight locations within 1 distance unit (see Fig.~\ref{fig:compare}A); here, movement is independent of local cues.
In \textit{scenario 2}, individuals move in a way that is considered locally optimal in foraging ecology \citep{stephens2019,scherer2020}.
Each individual assesses local cues at its current location, and eight surrounding locations --- the number of available food items ($F$), and the number of potential competitors ($N + H$) --- and moves to the location with the highest potential intake pay-off, which is given by $(F / (N + H + 1)) + \epsilon$; $\epsilon$ is a small error term.
We initially contrast these two scenarios to show how adding semi-mechanistic decision making to individual movement can affect the outcomes of movement IBMs.
As expected, optimally moving scenario 2 individuals had a higher per-capita intake than randomly moving scenario 1 individuals (Fig.~\ref{fig:compare}).
Individuals with higher intake should be expected to move less, as my model --- in line with foraging ecology theory \citep{charnov1976} --- explicitly considers a trade-off between movement and intake.
Surprisingly then, optimally moving individuals moved \textit{more} than random walkers (Fig.~\ref{fig:compare}).

Individuals in both scenarios had very similar numbers of spatial associations with other individuals (Fig.~\ref{fig:compare}, Fig.~\ref{fig:networks}).
Furthermore, these differences between movement scenarios were maintained on homogeneous landscapes as well (see SI Appendix X), contrary to the expectation that a more uniform spatial distribution of resources would allow random walkers to improve their outcomes, relative to optimal movement.
Overall, this comparison demonstrates the importance of active decision making in animal movement, and suggests why animals have evolved sophisticated sensory apparatuses to gather information from their environment \citep{avgar2013,berger2022,mann2021,swain2021}.
Such evolution is likely to be strongly dependent on fine-scale ecological conditions, primarily the \textit{availability} of information in the environment, as well as the energetic cost of evolving and maintaining sensory capabilities \citep{swain2021}.
However, one overlooked aspect in such models is how exactly individuals should incorporate local cues into movement decisions, which we tackle in the next two scenarios.

\subsection*{Movement as Evolved, Mechanistic Decision-making}

Optimal movement models are often labelled mechanistic as they include environmental cues in decision making \citep[e.g.][]{scherer2020}, yet the potential payoff of a location is strongly influenced by the functional response of intake in relation to competitors --- this is often simply assumed by modellers \citep[][]{vandermeer1997}.
Such implementations make the implicit evolutionary assumption that all individuals weigh cues from their social and resource environments similarly, as such weighting is supposed to be the stable outcome of selection fine-tuning individuals' preferences.
However, the functional responses commonly used in foraging theory are often modelled on empirical data, and as such may not be properly generalisable to other eco-evolutionary contexts, nor sufficient to capture the mechanistic links between individuals' social environment, their behavioural decisions, and the consequences for intake.
I showcase a more mechanistic way in which individuals can determine their optimal step when making foraging movements, which is to have distinct preferences for local cues (food items and potential competitors).
These preferences are similar to the coefficients of habitat- and step-selection functions \citep[][; see more below]{fortin2005,avgar2013,avgar2016,fieberg2010}.

In \textit{scenario 3}, individuals assesses local cues --- the number of available food items ($F$), and the number of potential competitors ($N + H$) --- at eight locations around themselves, and move to the location with the highest assessed suitability: $S = s_FF + s_C(N + H) + \epsilon$.
Here, $s_F$ and $s_C$ are inherited \textit{movement preferences} for food items and potential competitors respectively, and can take any positive or negative numeric values; $\epsilon$ is a small error term.
It is the relative contribution of $s_F$ and $s_C$ that determines individuals' \textit{movement strategy}.
I initialised the populations to have a broad range of movement strategies, so that it contained individuals with different combinations of preferences and avoidances of either food items or competitors.
Scenario 3 individuals had a lower median intake, and moved more overall, than individuals in scenarios 1 or 2.
As expected, intake was strongly positively correlated with the relative strength of individuals' preference for food items, $s_F$ (LMM coefficient here), while being uncorrelated with individuals' relative selection for competitors (LMM coefficient here).
Unsurprisingly then, scenario 3 populations were strongly clustered in space; however, they did not encounter substantially more individuals than scenario 2 individuals on average, although the variance in the number of encounters was large.

Given substantial inter-individual differences in intake based on relative preference for food items, biologists would expect that movement strategies that prefer food items would have a selection advantage.
I tested this expectation by adding an evolutionary component to the model: over 100 generations, individuals reproduce, passing on their preferences ($s_F$, $s_C$) to their offspring.
The preference values undergo random, independent mutations with a probability $p$ = 0.01, and with a mutation step size drawn from a Cauchy distribution with a scale of 0.01.
Consequently, most mutations are small, but larger mutations do occasionally occur.
The number of offspring is proportional to individuals' intake of food items during their lifetime \citep{netz2021,gupte2021a,gupte2022c}.
For simplicity, we assume asexual, haploid individuals, and that population sizes are fixed, and that generations do not overlap.
I implemented large-scale natal dispersal, such that individuals are typically initialised (`born') within a standard deviation of 10 units of their parents --- this makes scenario 3 relatively similar to the random initialisation of individual positions in scenarios 1 and 2 (but see SI Appendix 1 for an explanation of the outcomes under small-scale natal dispersal; see also \cite{gupte2021a,gupte2022c}).
These modelling choices must be explicitly implemented in simulation models' code, bringing the assumptions of classical models --- treated as received wisdom and hence ignored --- to the fore.

I found that scenario 3 population converged to a similar movement strategy within only a few (100) generations, across all twenty replicates.
This strategy was to primarily prefer moving towards food items, while having a small preference or avoidance of potential competitors.
The `evolved' scenario 3 individuals had better ecological performance than their ancestral populations (which we consider the first generation, G = 1), taking in more food items on average, and moving less.
Indeed, these populations outperformed both the random walk and optimal movement implementations as well.
Adapting their movement strategies to the landscape also affected the social structure of scenario 3 populations --- there were fewer isolated individuals, more spatial clustering, and consequently, individuals encountered more unique conspecifics on average (higher mean degree; Fig.~\ref{fig:networks}).

\subsection*{Adding detail to mechanistic movement decision-making}

A key feature of individual-based simulation models is their ability to incorporate great amounts of ecological detail \citep{deangelis2019}.
With a simple extension to scenario 3, we show how to add biologically relevant details to models, and how these details can affect model outcomes.
Foraging can be a form of public information, serving as an indirect cue of the presence of resources, and furthermore, helping distinguish between individuals that are \textit{immediate} competitors (here, non-handlers), and those which are only future potential competitors \citep[][; here, handlers]{dall2005,giraldeau2018,beauchamp2008,beauchamp2013}.
Thus in my \textit{scenario 4}, we allow individuals to sense the handling status of nearby potential competitors, and to have separate heritable preferences for handlers ($sH$) and non-handlers ($sN$).
Individuals assess the suitability of locations as $S = S = s_FF + s_HH + s_NN + \epsilon$; $\epsilon$ is a small error term.
I implemented similar evolutionary and dispersal dynamics as in scenario 3.

Individuals in the final generation (G = 200) of scenario 4 had mostly evolved a movement strategy that we describe as `handler tracking', i.e., having a preference for successful neighbours handling a food item ($sH >$ 0), but avoiding unsuccessful neighbours that were still searching for a food item \citep[($sN <$ 0)][]{gupte2021a,gupte2022c}.
This is similar to the scenario 2 `optimal movement' strategy, and allows individuals to avoid areas with many potential competitors, and thus a lower potential intake.
Importantly however, it allows individuals to use indirect social information \citep{dall2005,spiegel2016a}, in the form of the positions of successful neighbours, to find resource clusters --- even when these clusters are not immediately perceptible (due to earlier depletion).
Consequently, scenario 4 individuals outperform both scenario 1 and scenario 2 individuals by having substantially higher mean per-capita intake (Fig.~\ref{fig:compare}.
While not shown here, this naturally leads to the conclusion that the resource landscape in scenario 4 is more depleted than in scenarios 1 and 2.
Scenario 4 individuals, after 200 generations of selection, also outperform scenario 4 populations that have not undergone selection (i.e., their ancestors), demonstrating the difference that adding evolutionary dynamics makes even to a mechanistic, habitat selection model \citep[such as][]{white2018}.

Scenario 4 individuals' evolved use of social information on the potential locations of resource clusters also leads them to have more spatial associations with conspecifics --- indeed, up to three times as many as in the random walk and optimal movement models (Fig.~\ref{fig:compare}).
These associations likely occur at or near resource clusters, leading to substantial spatial-social clustering in the final generation of scenario 3 populations (Fig.~\ref{fig:networks}).
Scenario 4 individuals across replicates associated with more individuals (degree, mean $\pm$ SD = ) than in scenarios 1, 2 and 3.
Spatial-social structure in animal populations can have important consequences for a wide range of processes and phenomena in animal ecology, including the transmission of animal culture \citep[such as foraging tactics or migration routes][]{romano2020,romano2021,jesmer2018,klump2021}, as well as the spread of infectious pathogens \citep{white2017,white2018,white2018b,webber2022,ezenwa2016,albery2020}.

\subsection*{Emergent Model Properties: Consequences for Disease Transmission}

To examine the consequences for disease ecology, I began by constructing social networks by recording individuals' pairwise associations (distance $<$ perception range) in all four scenarios \citep{farine2015}.
I ran simple epidemiological models with SIR assumptions over the social networks emerging in each scenario \citep[25 replicates per network; 1 network per scenario replicate][]{white2017,csardi2006,bailey1975}, with a hypothetical disease (transmission probability $\beta$ = 2.0, recovery probability $\gamma$ = 1.0; see Fig.~\ref{fig:sir}).
Between the non-mechanistic scenarios, the disease spread more rapidly through networks arising from the optimal movement model, than the random walk model.
In scenario 3, the disease spread much more rapidly through the population after selection for movement preferences (G = 100), than through the ancestral population that had not undergone such selection (G = 1), despite both populations being similarly spatially clustered.
This is not unexpected as initially (G = 1), there are a number of individuals that avoid potential competitors ($s_C <$ 0), and this could hinder transmission in comparison with their descendants, fewer of which are similarly isolated.
Importantly, the unevolved mechanistic movement implementation had a lower infection peak than the random walk and optimal movement implementations.
A model beginning with diverse movement strategies, and not including an evolutionary component when implementing mechanistic movement \citep[such as][]{white2018}, would correctly conclude that mechanistic movement implementations differ from more mainstream movement modelling approaches in epidemiological outcomes, but crucially miss that this difference is that a disease would spread much more rapidly through the population.
Scenario 4 obtains a similar result --- populations adapted to their landscape cluster together, which allows a disease to spread even faster through their network than in scenario 3.

Overall, individual-based models offer the potential for more realistic social networks to emerge, with the expectation of more robust epidemiological insights from these networks \citep[][]{lunn2021}.
The development of spatially explicit movement-disease models allows both movement and disease transmission dynamics to be modelled explicitly \citep{white2018b,white2018,scherer2020,gupte2022c}.
However, network models conditioned on emergent social networks allow multiple pathogen scenarios to be run on the same network, saving computational time, and they also enable comparison with empirical work in the disease ecology of moving animals \citep{wilber2022}.
The importance of network formation processes in epidemiological modelling is already well known \citep[][]{white2017,wilber2022}.
My example model demonstrates how social networks emerging from individual-based movement models are sensitive to the modelling of the movement process, and how model details can affect predictions about disease outbreaks, and animal spatial-social ecology generally \citep{webber2018,webber2022}.
Furthermore, including evolutionary dynamics in combined movement-disease models can reveal surprisingly rapid evolutionary transitions in social movement strategies that make populations more robust to the spread of pathogens (see Chapter~\ref{ch:pathomove}).

% \subsection*{Discussion and Concluding paragraph}

% Importance of natal dispersal --- sexual reproduction and non-random mating --- the effect of landscape structure and productivity --- Implementation of the suitability scoring using complex functions such as neural networks (discuss \cite{mueller2011,deangelis2019}).

\section*{Estimating the Evolutionary Consequences of Movement Strategies from Empirical Data}

Finally, a key barrier to achieving a unified evolutionary ecology of animal movement is understanding the evolutionary consequences of animal movement strategies.
These consequences may be broken down into two key components, survival, and fecundity; together these determine lifetime fitness.
Ecologists, taking a phenomenological approach, have been able to make some headway in examining the evolutionary consequences of some movements, such as annual migrations.
% This has involved classifying animals' whole migration tracks into broad categories, to investigate whether there are survival or breeding success differences among migratory strategies.
For example \textcite{sergio2022} recently showed that compensation for drift in the north-south migration route, caused by lateral east-west winds, improved with age in black kites (\emph{Milvus migrans}), but that this was mainly due to poorly navigating individuals dying while young.
While this study highlights the importance of movement for evolutionary dynamics --- in this case, conferring a survival benefit to better navigators --- the inferences are often specific to particular taxa, and difficult to generalise.
However, we currently possess some methods that could be used to link the consequences of movement across temporal scales.

At relatively short temporal scales of a few tracking seasons, one approach is to study movement in the context of a common `currency', energy.
Combined experimental-observational approaches, linking respirometry measures of resting metabolic rate, doubly-labelled water measures for daily metabolic output, and tri-axial accelerometry and movement tracking, have paved the way for robust estimates of daily energy expenditure in free living animals \parencite{stothart2016}.
Animals' spatial settings can impose or alleviate metabolic costs, leading to the broader approach of studying `energy landscapes', i.e., environmental factors that change the ``cost of transport'' \parencite{shepard2013}.
Building off this work, we can now estimate how the cost of navigating through landscapes can affect large-scale patterns of animal space-use \parencite[e.g.][]{gallagher2017}.
Yet more recent work is probing how animals' fine-scale movement decisions can be linked directly to the energetic costs of those decisions \parencite{klappstein2022}.
% While appealing, we must recognise that many reference relationships between telemetry measures and movement metrics have not always accounted for the possibility of individual variation in physiology or metabolism, leaving open the possibility that movements that seem energy intensive might not be so for some individuals that are specialised in moving that way.
Overall then, the frameworks for measuring energetic loss in moving animals are well developed, and this can be linked to reductions in both individual survival and fecundity.

The positive effects of movement are more challenging to measure.
Energetic gain, for instance, requires the detection of foraging bouts.
Here too, the addition of accelerometry data can be useful in detecting sudden bursts of activity, especially those associated with predation attempts \parencite{williams2014,bryce2017}.
However, it is still challenging to remotely and automatically determine the energetic gain from a predation event.
The task of measuring the calorific value of forage is easier for herbivorous species, as vegetation cover and quality can often be quantified from remote sensing platforms \parencite{pettorelli2011}.
The caveat here is that the spatial resolution of remotely sensed data is often low.
Movement itself cannot confer increased fecundity, but can indirectly facilitate more or higher quality breeding attempts through increased sampling of breeding opportunities \parencite[as in][]{kempenaers2017}.
Yet movement data can be very useful in determining whether individuals have bred successfully, and uncover the characteristics of good nesting sites \parencite{picardi2020}.
Integrating the analysis of tri-axial acceleration data, could help refine current methods for detecting breeding or nesting outcomes, at least in some taxa \parencite{schreven2021}.
At larger temporal scales, individuals' preferences for energy landscapes could be linked to their survival or reproductive success, for global comparisons of the potential evolutionary consequences of movement strategies.

{ \begin{center} \barfont{-.-} \end{center} }

% Finally, global changes --- especially the ubiquitous presence of people --- severely impact animal movements \parencite*[]{tucker2018}.
% This makes it challenging to determine whether the movements we can observe are truly reflective of animals' natural, evolved movement preferences (this is always tricky, see Chapter~\ref{ch:patternprocess}) --- apart from a clear avoidance of humans.
% Significant methodological roadblocks do remain: it is still challenging to link animal tracking data with other outcomes, such as multi-year breeding success.
% That is partially because lifetime animal tracking \parencite{getz2008,nathan2008a} is still beyond current technologies \parencite{kays2015,hussey2015,nathan2022,jetz2022}.
%  --- and this is primarily an issue of increasing data-loggers' battery life so that they can collect data over many years rather than months.
% Reducing loggers' temporal resolution is one possible solution, but low-resolution data are often insufficient to detect key aspects of the movement process \parencite{nathan2022}, such as nesting success \parencite{picardi2020}.
