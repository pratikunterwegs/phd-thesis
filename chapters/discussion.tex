
\phantomsection
% \addtocontents{toc}{\protect\vspace{\beforebibskip}}%
% \addcontentsline{toc}{chapter}{\tocEntry{\color{black}\scshape\bfseries{General~Discussion: Linking the Ecology and Evolution of Animal Movement with Mechanistic, Individual Based Models}}}%
\chapter{{\color{gray}General Discussion}\\Linking the Ecology and Evolution of Animal Movement with Mechanistic, Individual Based Models}
% \chaptermark{Linking the Ecology and Evolution of Animal Movement with Mechanistic, Individual Based Models}

{\sffamily{Pratik R. Gupte}}

{\color{red}WORK IN PROGRESS: FIND PROPOSAL BELOW}

Movement is key to animal ecology as many ecological processes have an explicit spatial context. Movement links individual morphological and behavioural traits with population-level outcomes across contexts. This includes foraging, intra- and inter-specific interactions such as competition and predation, as well as the transmission of pathogens and animal culture. 

As animal movement ecology closes in on fine temporal scales, there is also increased interest in using models to tackle large, evolutionary timescales.

Ecologists recognise that movement strategies have evolved in an evolutionary context, and increasingly, that evolution can act on short, ecological timescales. Rapid global changes mean that animals’ evolutionary contexts are also shifting. This makes integrating the potential feedbacks between ecology and evolution key to understanding emergent phenomena (such as sociality and culture), and traits (such as body size, or cognitive complexity) that are strongly associated with movement.

In our proposed Ideas and Perspectives piece, we seek to highlight the role of mechanistic, individual-based models (IBMs) in linking the ecology and evolution of animal movement. 

We shall first lay out a framework for thinking about key eco-evolutionary processes with a movement component. We shall discuss how such models can be brought in line with current individual-centric methods in animal movement ecology, such as step-selection analysis. We cover the tools and techniques that enable the modelling of populations of individuals, each with evolved traits, including movement strategies. 

Using three simple, spatially explicit, eco-evolutionary models, we shall demonstrate some key concepts:
Mechanistic, spatial IBMs are especially suitable for theoretical work within the movement ecology paradigm, which is explicitly individual-centric.
Including evolutionary dynamics in simple ecological scenarios can allow for substantial individual variation in animal traits.
Feedbacks between ecological and evolutionary processes can lead to extremely rapid shifts in emergent population outcomes.
