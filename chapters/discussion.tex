
\addtocontents{toc}{\protect\vspace{\beforebibskip}}%
% \addcontentsline{toc}{chapter}{\tocEntry{\color{black}\itshape{Synthesis: Linking the Ecology and Evolution of Animal Movement with Mechanistic, Individual Based Models}}}%
\chapter{Synthesis: Linking the Ecology and Evolution of Animal Movement with Mechanistic, Individual Based Models}\label{ch:discussion}
\chaptermark{Synthesis}

Movement is key to animal ecology across spatial and temporal scales, as nearly all ecological processes have an explicit spatial context \citep{nathan2008a}.
By moving, animals can track seasonal fluctuations in resources \citep{geremia2019,abrahms2021a}, and facilitate or avoid ecological interactions such as inter- and intra-specific competition (e.g. \cite{duckworth2007}), predation \citep[e.g.][]{kohl2018}, parasitism \citep{weinstein2018}, mating displays and breeding site selection \citep{kempenaers2017}, as well as the transmission of animal culture \citep{jesmer2018,klump2021}, or infectious pathogens \citep[][see also Chapter~\ref{ch:pathomove}]{weinstein2018,monk2022,stroeymeyt2018}.
Mobile animals engineer their local environments, transferring nutrients \citep{leroux2018}, inducing local green-up \citep{geremia2019}, and affecting the distributions of overlapping species \citep[see e.g.][]{kohl2018,leroux2018,duckworth2007,monk2022}.
Perturbed natural regimes of animal movements (e.g. due to climate change), can severely impact humans, from direct conflict \citep{abrahms2021} to potential pathogen outbreaks \citep{carlson2022a,wille2022}; conversely, natural distributions of wildlife could aid climate mitigation by regulating key biotic and abiotic processes \citep{schmitz2018,malhi2022}.
Animal movement is thus crucial to a sound understanding of ecological processes and patterns generally \citep{jeltsch2013,schlagel2020}.

\section*{Reflections on Part~\ref{part:eco}}

\textbf{Part~\ref{part:eco}} of this thesis took an empirical approach to animal movements and distributions.
Answering this question has benefited greatly from the adoption of advanced animal tracking technology, and especially from the proliferation of GPS loggers \citep{cagnacci2010}.
Yet the majority of species of birds and mammals (leave alone reptiles or amphibians) cannot currently be tracked because most high-resolution loggers are much too heavy for them to bear safely.
High-throughput tracking systems such as ATLAS with its lightweight tags can allow researchers to achieve, at regional scales, a far more detailed understanding of animal movement than sought by \textcite{wikelski2007} when floating the idea of ICARUS (see now \cite{jetz2022}).
Yet data from these systems is not as conservatively `cleaned' as that from GPS tracking, and this is because the original end uses of each of these systems are very different.
In Chapter~\ref{ch:preprocessing}, I showed how a set of simple techniques and workflows can be used to substantially improve the quality of raw ATLAS data.
\textcite[]{beardsworth2022mee} have now shown that the accuracy of ATLAS systems (in this case, the Wadden Sea ATLAS system; \cite{bijleveld2021}) --- after applying these cleaning methods --- is comparable to GPS tracking, but with a much higher sampling rate.
An interconnected network of such high-throughput systems could represent one option for how animals could be tracked at high spatio-temporal resolution at large, continental scales \citep{nathan2022}.

Meanwhile, even data from earlier iterations of high-throughput systems can be made usable by robust filtering and cleaning.
In Chapter~\ref{ch:holeybirds}, I used data from the original ATLAS system deployed in the Hula Valley in Israel, to study how moult --- the loss and regrowth of flight feathers --- affects bird movement and habitat selection.
\graffito{
    Before beginning this project, my primary experience with bird moult had been during my master's thesis fieldwork on Kolguev island in the Russian Arctic in 2016.
    Geese breed there in the summer, undergoing full wing moult, which created a small window where they could be corralled into enclosures, to be tagged with GPS loggers.
}
This work is supported by recent findings that birds' flight characteristics affect whether they will risk crossing even narrow open tracts, such as forest roads \citep{claramunt2022}.
The results here suggest that predation risk avoidance could be a possible mechanism by which some areas that appear productive become unsuitable for many bird species --- agricultural fields for instance provide little cover from aerial predators.
Birds have long been anecdotally suspected to avoid certain features such as water bodies despite being powerful flyers, to the extent that this has prevented entire taxa from colonising archipelagos  \citep{diamond1981}.
Indeed, this effect is now much better quantified by studying the migration of raptors across open water \citep{nourani2020}.
Both behaviours seem bizarre to human observers, possibly because as a terrestrial species, we subconsciously think of flight as a certain kind of invulnerability, whether from environmental hazards or active hunters.
It is much more likely that we simply do not --- and perhaps cannot --- really appreciate how complex flight is, and the many risks it holds.
One of the main conclusions of Chapter~\ref{ch:holeybirds} (and of Part~\ref{part:evo}), then, is the importance of adopting the perspective of the study species, the \textit{individual in its context}, when seeking to understand the short- and long-term drivers of animals' behaviour.

Individual movement and habitat selection scale up to produce species distributions and community composition \citep{schlagel2020a}.
In Chapter~\ref{ch:hillybirds}, we showed mixed results for the effects of climate and land cover types on the distributions of birds in the Nilgiri and Anamalai Hills of southern India, a biodiversity hotspot.
One drawback of individual-centric animal tracking studies has been small sample sizes due to the costs of tracking technologies; this means that most studies focus on tracking at most a few species.
Citizen science studies allow researchers to accomplish two key goals at once: \textit{(i)} harness the spatial and temporal spread of amateur naturalists to collect ecological data, potentially from dozens of species at once, and \textit{(ii)} in so doing, strengthen public participation in the scientific process.
The findings are not as clear as in \textcite{freeman2018}, who showed that Andean birds were on the ``escalator to extinction''.
The current era of large ecological datasets being widespread is only about a decade old; for example, \textit{eBird} data from India span only a seven year period since 2015.
This makes it especially challenging to compare the drivers of current species distributions across regions with different histories of being studied, or within a region that has not been long studied.
The Nilgiri Hills offer a unique opportunity to integrate historical data from the 1820s onwards --- primarily birds shot as specimens by British colonial officials --- into modern species distribution analyses.
This project, for which I gathered data from bird specimens stored in the Natural History Museum of London's Tring collections, is currently ongoing.
When complete it could stand alongside efforts such as the Grinnell Resurvey \citep{tingley2009a,tingley2009b} in demonstrating how a century of land cover and climatic change has affected species distributions.

\section*{Reflections on Part~\ref{part:evo}}

Given its importance to animal ecology, movement is often implicitly included in the cornerstone models of eco-evolutionary theory (group topology: \cite{hamilton1971}; resource exploitation: \cite{fretwell1970, charnov1976}; community assembly: \cite{macarthur1967}).
Yet we currently lack theory that links the short-term ecological drivers and outcomes of movement with its evolutionary causes --- essentially, there is no evolutionary extension to the `movement ecology paradigm' \citep{holyoak2008}.
This hinders insight into how intensified selection on mobile species due to global change \citep[e.g.][]{vangils2016} would affect animal movement and related phenomena.
% Ecological theory, in order to provide general insights, must necessarily simplify reality down to essential processes.
One such simplification has long been to consider movement to be a population-level property shared by all individuals.
Work on consistent behavioural differences in animals, including in movement, suggests that this assumption is not well supported \citep{spiegel2017,shaw2020,stuber2022,webber2018,webber2020,abrahms2017}.
Movement is also often modelled as either a random or an optimal process --- both of these assumptions are simplistic, and animals integrate many internal and external cues when making movement decisions \citep{nathan2008a}.
Individual-based simulation models (IBMs) are well suited to representing movement as a decision made after integrating multiple cues in complex ecological contexts \citep{huston1988,deangelis2019}, and can include simplified evolutionary dynamics \citep{getz2015,getz2016,netz2021}.
This in the class of models I have used in Part~\ref{part:evo} for broad conceptual insight into the evolution of animal movement strategies.

In Chapter~\ref{ch:kleptomove}, I showed how animal movement and competition strategies jointly evolve, using an individual-based model with 10,000 individuals moving about on a grid of over 250,000 cells --- among the largest IBMs in this field of study.
The model had a number of interesting outcomes:

In Chapter~\ref{ch:pathomove}, I tackled a scenario that in expected to become increasingly common --- the transmission of novel pathogens from one species to another \citep{carlson2022a}.
Indeed currently the hitherto poorly known tropical African disease monkeypox is spreading across the globe, SARS-CoV-2 has seen multiple introductions to animals, including abundant wildlife \citep{kuchipudi2022}, and the H5N1 strain of avian influenza has been spreading through multiple temperate species \citep{wille2022}.
My relatively simple model of the trade-off between social information use (in a foraging context), and the risk of pathogen transmission generated clear predictions for how such novel pathogen introductions should affect the evolution of host sociality.
Worryingly, a cascading effect of decreased host sociality in most scenarios is poorer ecological performance in terms of harvesting resources from the landscape, leaving populations vulnerable to other environmental risks.
The potential consequences for other members of an ecological community --- albeit via the mechanism of mortality --- are borne out by \citet{monk2022}, who studied the effects of the introduction of mange to vicu\~{n}as in Patagonia.
The scenarios I modelled may actually be too mild, and novel pathogen spillovers could exterminate their hosts, rather than force the evolution of less gregarious social systems.
The scale of future work required in this field is daunting: identifying outbreaks as they happen, often in remote areas and involving poorly known species; determining patterns of species' spatial overlap that could aid cross-species transmission beyond the initial spillover; determining which species --- for a range of reasons --- may be at heightened risk from an epi- or panzootic outbreak; and finally, determining a response that preserves species while minimising risk of further spillover.

Specifically, I advocate that movement be modelled as a response to local cues rather than a random walk \citep[`mechanistic'; see][]{mueller2011}.
This is the approach I have taken in Chapters~\ref{ch:kleptomove} and \ref{ch:pathomove}.
In the models presented in those chapters, I used IBMs, in which individuals can be given different \emph{movement preferences} and thus make quite different decisions when presented with similar cues \citep{getz2015,white2018}.
Yet an open question when including such behavioural variation is whether the emergent outcomes may be transient phenomena that are quite different from the dynamics obtained on evolutionary timescales.
This is especially the case when modelling processes that have fitness consequences for some behavioural types (such as disease in \cite[]{white2018}).

Theory rarely addresses the long-term, evolutionary causes or consequences of movement, despite adaptive reasoning underpinning many models \citep{charnov1976,fretwell1970}.
Evolutionary models of movement rules treat them as population properties (as in \cite*[]{dejager2011,dejager2020}, or \cite*[]{morris2011}), whereas movement is an individual-level outcome, and it is on individual outcomes that selection acts.
When individuals with different movement strategies have equivalent fitness, populations may show movement polymorphisms \citep{wolf2012,shaw2020,getz2015}.
Including evolutionary dynamics in movement models could thus provide initial predictions for when individual variation (with its own consequences; \cite{spiegel2017}) should be expected.
We could also gain insight into how movement strategies could possibly evolve under various ecological scenarios.
This second aspect is often ignored, possibly because evolution is considered too slow to be relevant to the understanding and management of ecological dynamics over a few decades.
This assumption is mistaken, as evolution can be both rapid and adaptive \citep[][]{bonnet2022}.
Consequently, we additionally advocate for movement models to be embedded in an evolutionary context, with individuals' movement outcomes subject to selection, and their movement preferences subject to random change (mutation).

In this \textit{Perspective}, we propose a framework for mechanistic models of intermediate complexity that integrate both the ecological dynamics of animal movement, and their evolutionary causes and consequences.
The key feature of such models is to let individual-level ecological outcomes in one generation influence which movement strategies are present in future generations, thus establishing a feedback loop between animals' evolutionary history and their current spatial ecology.
With two case studies we contrast the conceptual insights from our modelling approach, with those from currently widespread model implementations, with special reference to emergent patterns.
We first demonstrate how movement as the outcome of natural selection on habitat selection leads to differences from both random and optimal movement, for key aspects of animal spatial ecology, including resource harvesting, spatial associations, and the spread of infectious diseases.
We then demonstrate how considering movement as evolved habitat selection leads to important differences in the evolutionary dynamics of traits commonly linked to movement ---  perception range and body size.
Finally, we discuss developments and remaining challenges in linking our modelling approach with the individual-centric empirical study of animal movement using tracking data.
Overall, we hope to convince readers that adopting mechanistic models of intermediate complexity, and including an evolutionary component, could provide much needed theoretical predictions in animal movement ecology.

\section*{reproducibility and computation}

\section*{A conceptual framework for mechanistic, eco-evolutionary individual-based models of animal movement}

% and habitat selection \citep{morris2003}

% a key concept in individual behavioural differences in movement ecology is the difficulty in addressing whether differences in outcomes (movement paths etc) are reflective of differences in movement preferences! cite stuber and replies to stuber, especially orr and noa.

% The key feature of the class of models we advocate is that \textit{(1)} individuals make decisions according to individual-specific preferences for available environmental cues, \textit{(2)} individuals' preferences for environmental cues, \textit{taken together}, form their behavioural strategy, and \textit{(3)} individuals' behavioural strategies have evolutionary consequences, i.e., the ecological outcomes in one generation determines the population mixture of behavioural strategies in the next generation.
% Here we focus on movement preferences and movement strategies, which are among the behavioural strategies of individuals, and which may also facilitate or constrain which behavioural strategies individuals can employ \citep{nathan2008a,spiegel2017}.

\section*{Insights from mechanistic, evolved step selection compared with random walks and optimal movement}

We begin by demonstrating how the implementation of the movement process in IBMs can substantially alter model conclusions.
We show that considering movement as the outcome of evolved preferences for locally available cues leads to very different ecological outcomes when compared to mainstream frameworks such as random walks and optimal local movement (see Box 2).
These differences can be important when such models are used as baselines against which to compare patterns observed from empirical animal tracking data [CITE examples], or to make predictions for how key ecological processes --- such as the transmission of pathogens or culture --- occur in animal populations (pathogens: \cite{white2018,white2018b,cantor2021,scherer2020}; culture: \cite{romano2020,romano2021,cantor2021,cantor2021a}).

We compare ecological outcomes in 20 replicates of four scenarios of a model with the same ecological processes, that are borrowed from \citet{gupte2022c}.
In our model, 200 individuals inhabit a landscape of 30 square units, which also contains 450 discrete food items (see Fig.~\ref{fig:compare}A).
Food items are patchily distributed to form distinct clusters (N = 30, 15 items per cluster).
Individuals can move in increments of 1 distance unit (like a king in chess), and can perceive items ($F$) and other individuals at locations 1 distance unit away.
When individuals perceive a food item, they pick it up and must handle it for 5 time-steps until they can gain its energetic benefit \citep{ruxton1992,gupte2021a,gupte2022c}; we call such individuals `handlers' ($H$).
While individuals are handling an item, they are immobilised.
Individuals compete with each other exploitatively and an item once picked up by an individual is unavailable to its neighbours; these individuals continue searching for food, and we call them `non-handlers' ($N$).
Items regenerate at the same location after a fixed number of timesteps, which we call the regeneration time ($T_R$; default = 100), and while an item is regenerating, it cannot be sensed by nearby individuals.
Individuals have a lifetime of 400 timesteps, over which they forage and move over the landscape.
The model's four scenarios differ in their implementation of individual movement.

\subsection*{Random walks and optimal movement}

In \textit{scenario 1}, individuals perform a random walk, and have a uniform probability of either remaining in their current location, or moving in a direction chosen from among eight locations within 1 distance unit (see Fig.~\ref{fig:compare}A); here, movement is independent of local cues.
In \textit{scenario 2}, individuals move in a way that is considered locally optimal in foraging ecology \citep{stephens2019,scherer2020}.
Each individual assesses local cues at its current location, and eight surrounding locations --- the number of available food items ($F$), and the number of potential competitors ($N + H$) --- and moves to the location with the highest potential intake pay-off, which is given by $(F / (N + H + 1)) + \epsilon$; $\epsilon$ is a small error term.
We initially contrast these two scenarios to show how adding semi-mechanistic decision making to individual movement can affect the outcomes of movement IBMs.
As expected, optimally moving scenario 2 individuals had a higher per-capita intake than randomly moving scenario 1 individuals (Fig.~\ref{fig:compare}).
Individuals with higher intake should be expected to move less, as our model --- in line with foraging ecology theory \citep{charnov1976} --- explicitly considers a trade-off between movement and intake.
Surprisingly then, optimally moving individuals moved \textit{more} than random walkers (Fig.~\ref{fig:compare}).
Individuals in both scenarios had very similar numbers of spatial associations with other individuals (Fig.~\ref{fig:compare}, Fig.~\ref{fig:networks}).
Furthermore, these differences between movement scenarios were maintained on homogeneous landscapes as well (see SI Appendix X), contrary to the expectation that a more uniform spatial distribution of resources would allow random walkers to improve their outcomes, relative to optimal movement.
Overall, this comparison demonstrates the importance of active decision making in animal movement, and suggests why animals have evolved sophisticated sensory apparatuses to gather information from their environment \citep{avgar2013,berger2022,mann2021,swain2021}.
Such evolution is likely to be strongly dependent on fine-scale ecological conditions, primarily the \textit{availability} of information in the environment, as well as the energetic cost of evolving and maintaining sensory capabilities \citep{swain2021}.
However, one overlooked aspect in such models is how exactly individuals should incorporate local cues into movement decisions, which we tackle in the next two scenarios.

\subsection*{Movement as evolved, mechanistic decision-making}

Optimal movement models are often labelled mechanistic as they include environmental cues in decision making \citep[e.g.][]{scherer2020}, yet the potential payoff of a location is strongly influenced by the functional response of intake in relation to competitors --- this is often simply assumed by modellers \citep[][]{vandermeer1997}.
Such implementations make the implicit evolutionary assumption that all individuals weigh cues from their social and resource environments similarly, as such weighting is supposed to be the stable outcome of selection fine-tuning individuals' preferences.
However, the functional responses commonly used in foraging theory are often modelled on empirical data, and as such may not be properly generalisable to other eco-evolutionary contexts, nor sufficient to capture the mechanistic links between individuals' social environment, their behavioural decisions, and the consequences for intake.
We showcase a more mechanistic way in which individuals can determine their optimal step when making foraging movements, which is to have distinct preferences for local cues (food items and potential competitors).
These preferences are similar to the coefficients of habitat- and step-selection functions \citep[][; see more below]{fortin2005,avgar2013,avgar2016,fieberg2010}.

In \textit{scenario 3}, individuals assesses local cues --- the number of available food items ($F$), and the number of potential competitors ($N + H$) --- at eight locations around themselves, and move to the location with the highest assessed suitability: $S = s_FF + s_C(N + H) + \epsilon$.
Here, $s_F$ and $s_C$ are inherited \textit{movement preferences} for food items and potential competitors respectively, and can take any positive or negative numeric values; $\epsilon$ is a small error term.
It is the relative contribution of $s_F$ and $s_C$ that determines individuals' \textit{movement strategy}.
We initialised our populations to have a broad range of movement strategies, so that it contained individuals with different combinations of preferences and avoidances of either food items or competitors.
Scenario 3 individuals had a lower median intake, and moved more overall, than individuals in scenarios 1 or 2.
As expected, intake was strongly positively correlated with the relative strength of individuals' preference for food items, $s_F$ (LMM coefficient here), while being uncorrelated with individuals' relative selection for competitors (LMM coefficient here).
Unsurprisingly then, scenario 3 populations were strongly clustered in space; however, they did not encounter substantially more individuals than scenario 2 individuals on average, although the variance in the number of encounters was large.

Given substantial inter-individual differences in intake based on relative preference for food items, biologists would expect that movement strategies that prefer food items would have a selection advantage.
We tested this expectation by adding an evolutionary component to the model: over 100 generations, individuals reproduce, passing on their preferences ($s_F$, $s_C$) to their offspring.
The preference values undergo random, independent mutations with a probability $p$ = 0.01, and with a mutation step size drawn from a Cauchy distribution with a scale of 0.01.
Consequently, most mutations are small, but larger mutations do occasionally occur.
The number of offspring is proportional to individuals' intake of food items during their lifetime \citep{netz2021,gupte2021a,gupte2022c}.
For simplicity, we assume asexual, haploid individuals, and that population sizes are fixed, and that generations do not overlap.
We implemented large-scale natal dispersal, such that individuals are typically initialised (`born') within a standard deviation of 10 units of their parents --- this makes scenario 3 relatively similar to the random initialisation of individual positions in scenarios 1 and 2 (but see SI Appendix 1 for an explanation of the outcomes under small-scale natal dispersal; see also \cite{gupte2021a,gupte2022c}).
These modelling choices must be explicitly implemented in simulation models' code, bringing the assumptions of classical models --- treated as received wisdom and hence ignored --- to the fore.

We found that scenario 3 population converged to a similar movement strategy within only a few (100) generations, across all twenty replicates.
This strategy was to primarily prefer moving towards food items, while having a small preference or avoidance of potential competitors.
The `evolved' scenario 3 individuals had better ecological performance than their ancestral populations (which we consider the first generation, G = 1), taking in more food items on average, and moving less.
Indeed, these populations outperformed both the random walk and optimal movement implementations as well.
Adapting their movement strategies to the landscape also affected the social structure of scenario 3 populations --- there were fewer isolated individuals, more spatial clustering, and consequently, individuals encountered more unique conspecifics on average (higher mean degree; Fig.~\ref{fig:networks}).

\subsection*{Adding detail to mechanistic movement decision-making}

A key feature of individual-based simulation models is their ability to incorporate great amounts of ecological detail \citep{deangelis2019}.
With a simple extension to scenario 3, we show how to add biologically relevant details to models, and how these details can affect model outcomes.
Foraging can be a form of public information, serving as an indirect cue of the presence of resources, and furthermore, helping distinguish between individuals that are \textit{immediate} competitors (here, non-handlers), and those which are only future potential competitors \citep[][; here, handlers]{dall2005,giraldeau2018,beauchamp2008,beauchamp2013}.
Thus in our \textit{scenario 4}, we allow individuals to sense the handling status of nearby potential competitors, and to have separate heritable preferences for handlers ($sH$) and non-handlers ($sN$).
Individuals assess the suitability of locations as $S = S = s_FF + s_HH + s_NN + \epsilon$; $\epsilon$ is a small error term.
We implemented similar evolutionary and dispersal dynamics as in scenario 3.

Individuals in the final generation (G = 200) of scenario 4 had mostly evolved a movement strategy that we describe as `handler tracking', i.e., having a preference for successful neighbours handling a food item ($sH >$ 0), but avoiding unsuccessful neighbours that were still searching for a food item \citep[($sN <$ 0)][]{gupte2021a,gupte2022c}.
This is similar to the scenario 2 `optimal movement' strategy, and allows individuals to avoid areas with many potential competitors, and thus a lower potential intake.
Importantly however, it allows individuals to use indirect social information \citep{dall2005,spiegel2016a}, in the form of the positions of successful neighbours, to find resource clusters --- even when these clusters are not immediately perceptible (due to earlier depletion).
Consequently, scenario 4 individuals outperform both scenario 1 and scenario 2 individuals by having substantially higher mean per-capita intake (Fig.~\ref{fig:compare}.
While not shown here, this naturally leads to the conclusion that the resource landscape in scenario 4 is more depleted than in scenarios 1 and 2.
Scenario 4 individuals, after 200 generations of selection, also outperform scenario 4 populations that have not undergone selection (i.e., their ancestors), demonstrating the difference that adding evolutionary dynamics makes even to a mechanistic, habitat selection model \citep[such as][]{white2018}.

Scenario 4 individuals' evolved use of social information on the potential locations of resource clusters also leads them to have more spatial associations with conspecifics --- indeed, up to three times as many as in the random walk and optimal movement models (Fig.~\ref{fig:compare}).
These associations likely occur at or near resource clusters, leading to substantial spatial-social clustering in the final generation of scenario 3 populations (Fig.~\ref{fig:networks}).
Scenario 4 individuals across replicates associated with more individuals (degree, mean $\pm$ SD = ) than in scenarios 1, 2 and 3.
Spatial-social structure in animal populations can have important consequences for a wide range of processes and phenomena in animal ecology, including the transmission of animal culture \citep[such as foraging tactics or migration routes][]{romano2020,romano2021,jesmer2018,klump2021}, as well as the spread of infectious pathogens \citep{white2017,white2018,white2018b,webber2022,ezenwa2016,albery2020}.

\subsection*{Emergent model properties: Consequences for disease transmission}

To examine the consequences for disease ecology, we began by constructing social networks by recording individuals' pairwise associations (distance $<$ perception range) in all four scenarios \citep{farine2015}.
We ran simple epidemiological models with SIR assumptions over the social networks emerging in each scenario \citep[25 replicates per network; 1 network per scenario replicate][]{white2017,csardi2006,bailey1975}, with a hypothetical disease (transmission probability $\beta$ = 2.0, recovery probability $\gamma$ = 1.0; see Fig.~\ref{fig:sir}).
Between the non-mechanistic scenarios, the disease spread more rapidly through networks arising from the optimal movement model, than the random walk model.
In scenario 3, the disease spread much more rapidly through the population after selection for movement preferences (G = 100), than through the ancestral population that had not undergone such selection (G = 1), despite both populations being similarly spatially clustered.
This is not unexpected as initially (G = 1), there are a number of individuals that avoid potential competitors ($s_C <$ 0), and this could hinder transmission in comparison with their descendants, fewer of which are similarly isolated.
Importantly, the unevolved mechanistic movement implementation had a lower infection peak than the random walk and optimal movement implementations.
A model beginning with diverse movement strategies, and not including an evolutionary component when implementing mechanistic movement \citep[such as][]{white2018}, would correctly conclude that mechanistic movement implementations differ from more mainstream movement modelling approaches in epidemiological outcomes, but crucially miss that this difference is that a disease would spread much more rapidly through the population.
Scenario 4 obtains a similar result --- populations adapted to their landscape cluster together, which allows a disease to spread even faster through their network than in scenario 3.


Overall, individual-based models offer the potential for more realistic social networks to emerge, with the expectation of more robust epidemiological insights from these networks \citep[][]{lunn2021}.
The development of spatially explicit movement-disease models allows both movement and disease transmission dynamics to be modelled explicitly \citep{white2018b,white2018,scherer2020,gupte2022c}.
However, network models conditioned on emergent social networks allow multiple pathogen scenarios to be run on the same network, saving computational time, and they also enable comparison with empirical work in the disease ecology of moving animals \citep{wilber2022}.
The importance of network formation processes in epidemiological modelling is already well known \citep[][]{white2017,wilber2022}.
Our model demonstrates how social networks emerging from individual-based movement models are sensitive to the modelling of the movement process, and how model details can affect predictions about disease outbreaks, and animal spatial-social ecology generally \citep{webber2018,webber2022}.
Furthermore, including evolutionary dynamics in combined movement-disease models can reveal surprisingly rapid evolutionary transitions in social movement strategies that make populations more robust to the spread of pathogens \citep{gupte2022c}.

\subsection*{Discussion and Concluding paragraph}

Importance of natal dispersal --- sexual reproduction and non-random mating --- the effect of landscape structure and productivity --- Implementation of the suitability scoring using complex functions such as neural networks (discuss \cite{mueller2011,deangelis2019}).

% \newrefcontext[sorting=ynt]



% \section*{Case studies: why movement is key to ecological patterns}

% \section*{Why include evolution in animal movement studies}

% \subsection*{Importance of the individual-centric view}

% \subsection*{How evolved movement strategies are different from random movement}

% \subsection*{Evolution of complex traits can be very rapid}

% \section*{Conceptual ingredients of eco-evolutionary individual-based models}

% \section*{Practical aspects of implementing eco-evolutionary individual-based models}

% \subsection*{Modelling individuals}

% Biological theory has usually considered broad abstract representations of individuals as members of a population, with the population having a set of traits that can be described by simple statistics, such as the mean and standard deviation.
% Individual-based models, by contrast, naturally require that each individual in a simulated population be considered separately \citep{macnamara-houston197x}.
% Individuals typically have attributes, and can `perform' specific actions, which are usually interactions with their environment or with other individuals. --- [add more].

% Taking an individual-centric view naturally suggests that simulations written in a computer language follow imperative, object oriented programming.
% In this paradigm, individuals are `objects', often of a specific class or type, and all objects of this type share a common set of attributes (which may be thought of as traits), as well as possible methods or functions (which may be thought of as actions).
% Indeed, object oriented programming designed in the 1960s was explicitly inspired by biological systems. [CITE]

% \begin{quotation}
%     I thought of objects being like biological cells and/or individual computers on a network, only able to communicate with messages (so messaging came at the very beginning --- it took a while to see how to do messaging in a programming language efficiently enough to be useful).
% \end{quotation}

% Individuals may differ in their trait values, creating population heterogeneity, and these differences may influence individuals' performance of certain actions --- for example, the frequency of updating a search strategy, the likelihood of adopting a specific competition strategy, or orientation of movement towards a resource \citep{swain2021}. --- [add more].
% At a core practical level, however, an implementation that considers an entire population as an object, rather than every individual, is also an `individual-based' model, so long as the population object stores individual traits separately.

% \subsection*{Modelling landscapes}

% The key question when modelling individuals' environment is whether the model has an implicit spatial component, or whether it is spatially explicit --- and indeed, whether the choice matters.
% The cornerstone theoretical models of animals' movement ecology are not spatially explicit (and neither are they individual-based).
% Neither the Ideal Free Distribution \citep{fretwell1970}, nor Optimal Foraging Theory \citep{charnov1976} have a specifically spatial component.
% This is not to say that the two models do not have a clear spatial context, as both were inspired by specific scenarios --- the micro-scale distribution of New World warblers \citep[IFD:][]{fretwell1970}, and the movement of a forager between resource patches \citep{charnov1976}.
% It is rather the case that given the very simple assumptions of these models, that explicitly modelling space does not add substantially to the points they are attempting to make.

% When these assumptions are adjusted even slightly, taking space into account begins to make a difference to model outcomes [CITE HERE].
% For example, in models where individuals differ in competitive ability, there is a distinct spatial separation of more and less competitive types of individuals.
% Competitive animals occupy the most productive patches, where they have to contend with other competitive individuals, while less competitive individuals are displaced on to less productive patches [CITE HERE].
% This simple theoretical prediction is recovered in multiple systems and at multiple scales --- for example, in different species of geese, large families which as social units have high competitive ability relative to single birds, tend to occupy the profitable leading edge of grazing flocks (small spatial scale), and also the most profitable wintering sites following annual autumn migration [CITE BLACK AND SOMEONE BARNIES, SOMEONE BRENT GEESE].

% When modelling more complex links between individuals' ecological traits, an explicit spatial context becomes more important.
% For example, in producer-scrounger models \citep[e.g.][]{beauchamp2008}, or in the kleptoparasite model of Chapter~\ref{ch:kleptomove}, individuals' ability to profit from their landscape is mechanistically linked to \emph{which other individuals} are in the vicinity.
% Such models could still be represented in a spatially implicit way \citep{cressman2006,krivan2008,garay2015,garay2020}.
% These models are primarily concerned with either micro-scale, or very long term animal distributions, and omit the movement process.

% The point at which models tip over into requiring a spatially explicit context is when they consider movement as carrying a cost.
% Movement costs may be direct, in the form of the energy required for displacement over a certain distance, and this requires a distance calculation which implies a specific spatial configuration of the landscape.
% Yet movement costs may also be indirect, in the form of time or opportunity costs incurred when an individual moves between heterogeneously distributed resources --- this is how models in Chapters~\ref{ch:kleptomove} and \ref{ch:pathomove} are structured.
% In this latter case, the spatial structure of the landscape has a strong influence on the structure of movement costs, on which more below.

% \subsubsection*{Spatially implicit landscapes}

% Early theoretical models in animal spatial ecology were almost all spatially implicit \citep{fretwell1970,charnov1976}.

% mention movement between patches which is optimal --- movement that is random and taken from ideal gas theory --- random walks and animal spatial distributions resulting from random walks --- etc etc
% end with, this is not what we really want to cover here.

% cover some modern examples, lunn 2021 etc. mention also dinuzzo and griffen 2020.

% \subsubsection*{Spatially explicit landscapes}

% Spatially explicit representations of landscapes in animal movement models have become more common with the increase and proliferation of both computing power, and the programming skills of researchers in the field.

% give early examples --- mcnamara and houston --- deangelis' work --- deangelis review 2019 --- spiegel 2017, white 2018, etc etc --- talk about the different models and what they showed

% Spatially explicit landscapes can be challenging to implement, and involve multiple design decisions.
% These decisions are influenced by the biology of the system being modelled, as well as theory and previous implementations in the field.
% For example, many of the design decisions of Chapter~\ref{ch:kleptomove} are based on a previous implementation of a similar model in \citet{netz2021a}, and the spatial component of Chapter~\ref{ch:pathomove} is strongly influenced by the example models shown in \citet{spiegel2017}.
% Overall, two major design decisions are whether to implement discrete or continuous space, and whether to have this space host discrete or continuous resource items.

% \paragraph*{Discrete space implementations}

% In a discrete space implementation, the landscape has a finite number of locations at which an individual can reside, and to and from which it can move.
% Essentially, the landscape takes the form of a grid, and individuals can only occupy one tile or grid cell at a time.
% Individuals usually only move between connected cells, with cells connected with other cells depending on the implementation.
% Most grids are considered to be comprised of square cells, and these cells may be connected to those with which they share a side (4 cells, the von Neumann neighbourhood), or to those with which they share a corner (8 cells, the Moore neighbourhood).
% This square grid-cell implementation has a background in theoretical spatial ecology as the basis for cellular automata models \citep[even those concerned with animals][]{jeltschf.1997}, and it is widely used in animal movement modelling.
% For example, it is used in Chapter~\ref{ch:kleptomove}, as well as in \citet{white2018,dinuzzo2020,scherer2020} and \citet{netz2021a}.
% However, more complex implementations, including movements between un-connected cells (a Moore neighbourhood $>$ 1), can also be implemented --- this is already done in Chapter~\ref{ch:kleptomove} for fleeing handlers.

% [MENTION THEORETICAL BACKGROUND]

% Implementing discrete space landscapes leads to the question of whether the cells and the landscape overall represent relative spatial scales correctly.
% In Chapter~\ref{ch:kleptomove}, the landscape has 512 cells per side, and this is 25\% larger than the maximum number of cells a model individuals could reach in their lifetimes.
% This large spatial scale, relative to individuals' per-timestep or per-lifetime movements, allows Chapter~\ref{ch:kleptomove} to present its gridded landscape as representing relatively small, yet not microscopic, resource patches.
% The cells in discrete-space models are as as large or as small as they need to be from the perspective of biological theory.
% In Chapter~\ref{ch:kleptomove}, the $512^2$ cells represent `patches' of intermediate size, and movement is limited to the Moore neighbourhood.
% It is important to remember that in such conceptual models, the insights are more important than the exact matching of the model's various spatial and temporal scales.

% Gridded landscapes with discrete cells are relatively easy to implement in a range of programming languages.
% The grid cells are easily captured by a simple, two dimensional matrix data type in most cases, and such data types are easily constructed in low level languages, or available by default in scripting languages such as R \citep{rcoreteam2020}.
% Multiple such matrices can be stacked into an array, with each layer representing some specific ecological aspect of the landscape, and the model.
% While one layer may hold the number of individuals in each cell, another can easily represent the quantity of resource items available in the cell, and yet more layers can represent a range of other environmental conditions that impose costs or confer benefits upon individuals.

% The advantage of gridded landscapes is that relations between cells, and between individuals and other landscape or model components, are strongly constrained.
% Individuals usually cannot affect the localised dynamics of any cell but their own; for example, they cannot interact with competitors or resources on a neighbouring cell.
% These limits on possible interactions have clear computational benefits in terms of reducing model run times.
% To calculate the number of potential social interactions on a cell for instance, it is sufficient to determine the set of all individuals occupying the same cell.
% This task can be sped up or made easier to code by assigning each cell a unique identifying index (with cell [0,0] being cell 1, cell [0,1] = 2), and so forth.
% This bounds the maximum number of comparisons required to find all the potential interaction partners of an individual to $N - 1$, as a single integer (the focal individual's cell) is matched with the remaining $N - 1$ individuals in a population of $N$ individuals.
% In contrast, the same task --- counting neighbours for potential interactions --- requires potentially much more computation in continuous space models (see below).

% Individual-based models using gridded landscapes are common extensions to an increasingly popular method in the empirical study of animal movement, step-selection analysis (SSA) \citep{fortin2005,avgar2016,fieberg2021}.
% The link between mechanistic movement modelling and step-selection analyses is covered below, as well as in Chapter~\ref{ch:patternprocess}.
% The main reason that SSA --- which is agnostic to how space is implemented --- uses grid-based landscapes in its simulations of animal movement and habitat use, is that environmental data is available in the from of gridded raster layers.
% Consequently, packages such as \emph{amt} analyse animals' movements in continuous space, but predict their habitat utilisation as discrete raster layers \citep{signer2017,signer2019}.

% Simplified interactions are also the main disadvantage of discrete, gridded landscapes.
% It is somewhat unrealistic to maintain that two individuals can be in adjacent grid cells and yet are unable to interact with each other.
% This has implications for the local, small scale population and resource density, which may be lower when calculated separately per cell, than when calculated at the level of the group in continuous space.
% On the other hand, determining group size and local density in continuous space models can also be challenging, as delineating groups (i.e., clusters of individuals) is a complex task --- discrete space models make this step easier to implement by having each group be restricted to a single cell.
% One way of reconciling the natural objection to the `cell-as-group' idea is to maintain that the cells are of a spatial scale that allows for probabilistic interactions within the cell, but not with individuals in neighbouring cells.
% However, there is no specific constraint when implementing a model that an individual should \emph{not} be able to interact with individuals from neighbouring cells.
% Coding this latter case, though, discards the advantages of the gridded landscape (limited interactions), and researchers should probably consider implementing a continuous space landscape instead.

% While this section has focussed on two dimensional landscapes, it is often worth considering a one dimensional grid as well.
% One dimensional landscapes have a long history in animal spatial ecology, with an early appearance in the concept of the `selfish herd' [cite Hamilton 1969/7?].

% cite some other recent papers using one dimensional landscapes --- explain why they worked or did not etc etc.

% \paragraph*{Continuous space implementations}

% Models with continuous space implementations allow individuals (and discrete resources, if any) to take fractional positions on the model landscape.
% Examples of this approach include Chapter~\ref{ch:pathomove}, as well as \citet{spiegel2016,spiegel2017}, among many others [ADD MORE EXAMPLES].
% This modelling approach aligns closely with how spatial positions are usually represented in empirical methods, as pairs of decimal coordinates.
% Individuals' movement distances and angles are constrained only by the specifics of the implementation (see below).
% [TALK ABOUT THEORETICAL BACKGROUND --- IDEAL GAS EQUATION ETC.?]

% Continuous space implementations can handle the issue of spatial scales somewhat better than discrete space models.
% For example, movement distances can be capped at a fraction of perception ranges, and both can be fractions of the total modelled landscape size.
% Similarly, the distribution of resources is also easier to implement in terms of a density ($N$ resources per unit area), although this does depend on the spatial distributions of resources.
% In Chapter~\ref{ch:pathomove}, the landscape has an area of $60^2$ units$^2$, with an individual perception and movement range of 1.0 units, and a resource density of 0.5 food items per unit area.
% While the scaling of movement and perception in Chapter~\ref{ch:pathomove} is arbitrary (as the model aims primarily for insight), it is often easier to convince readers that continuous space models are `more realistic' than patch-based models with gridded landscapes.

% In some models, individuals simply move about on a landscape without interacting with a large number of landscape components \citep[see some models in][]{spiegel2017}.
% In such models it is sufficient to keep track of individuals' immediate coordinates, and these models are equally easily implemented as continuous space models or discrete space models.
% Continuous space implementations may be better suited to models in which \textit{(1)} individuals can vary their movement distance and heading, or in which \textit{(2)} realistic local interactions are especially important.
% For instance, Chapter~\ref{ch:pathomove} models two small-scale processes: exploitation competition, and pathogen transmission.
% Both of these processes are strongly dependent on exactly how many, and which, individuals are within interaction or transmission range.
% This makes the exact distance between individuals and features (such as resources) important, making the precise locations of individuals key to the model (although some disease-movement models adopt a patch approach; see \citealt{white2018,white2018b,jeltschf.1997}).
% This, combined with the capacity to handle multiple spatial scales, is continuous space landscapes' main advantage over grid-based landscapes.

% Continuous space landscapes can unlock model choices that would not be possible with a gridded landscape.
% For instance, individual movement can be modelled in much more flexible ways: individuals need not choose among movement locations, but can instead choose a movement distance and a heading, and move to that location (see details below) \citep{spiegel2017,mueller2011}. [CITE MORE].
% While it is also possible to combine this latter option with gridded landscapes, it is challenging to combine gridded and continuous space representations of related phenomena (landscape and movement).
% Continuous space also allows for various implementations of resource landscape structure that is not always easy to fit into a gridded format, and is especially useful when trying to create an environment with patchily distributed resources (see below).
% Algorithms that discretise heterogeneity in continuous space (such as for resources) into gridded values are abundant, however, and can achieve similar implementations for landscapes with discrete cells [CITE NLMR and NLMPy].

% Continuous space simulations are also widely available as extensions to popular analyses in the empirical study of animal movement \citep{noonan2019,calabrese2016,calabrese2018,fleming2014,fleming2015,gurarie2017,gurarie2016}.
% These implementations however, are not intended to take into account interactions with the environment, or with other individuals [BUT SEE THIS SINGLE AUTHOR PAPER - SPANISH NAME].
% Rather, these methods simulate plausible animal movement paths after fitting a continuous time movement model to empirical animal position-tracking data.
% Since these methods are not conditioned on environmental data, they are not constrained by the gridded structure of remote sensing data layers.

% The wide flexibility of continuous space, and the many interactions that may occur among landscape components (depending on model parameterisation) offer opportunities for complex model structures, but this can easily lead to drawbacks.
% First, because interactions between model components, such as individuals, are not neatly compartmentalised into cells, it is necessary to repeatedly determine whether any two interacting entities are within the appropriate range to do so.
% This problem is exacerbated when there are multiple types of interaction, each with their own ranges.
% For example, in Chapter~\ref{ch:pathomove}, individuals perceive food items, neighbours handling food items, and neighbours looking for food items.
% When infected, they also unknowingly transmit a pathogen to neighbours within range.
% This requires \emph{(1)} a distance calculation between each individual and each food item, and \emph{(2)} a distance calculation between all pairs of individuals --- this is repeated up to 500,000 times in the \emph{Pathomove} model.
% These calculations represent a substantial computational challenge, especially for studies requiring multiple replicates and parameter combinations.

% \paragraph*{Representing resources: Discrete and continuous implementations}

% Many ecological models that explicitly or implicitly have a movement component include some representation of resources.
% For example, classical individual-to-population models often assume a steady inflow of resources from the which individuals can benefit, but in such a way that the rate of inflow is not affected (continuous input models [CITE]).
% The reason for this is that an input rate is analytically tractable, whereas realistic implementations of discrete, depleteable resource items are challenging to incorporate in mathematical models.
% Individual-based simulation models, however, can represent substantially more complex consumer-resource interactions, and a broad range of mechanisms linking the two, from how individuals detect resources, to how they gain benefits from them, and how inter-individual associations are shaped by the presence or possession of resources (as in Chapters~\ref{ch:kleptomove} and \ref{ch:pathomove}).
% In addition, simulation models can represent both large- and fine-scale variation in the number, quality, and type of resources available.

% Spatial models representing continuous resource input usually define the resources at any given location as the outcome of a mathematical function of the location coordinates: $\text{resources} = f(x,y)$, in the case of two dimensional landscapes.
% This approach works equally well for one dimensional landscapes, and these can implement, among others, a resource gradient (i.e., a linear function of landscape extent), or a series of resource peaks and troughs (e.g. using a sine function).
% In the latter case, the largest positive values might represent areas with lots of resources; the lowest negative values, on the other hand, may represent areas of the landscape that are a net drain on individual resources.
% Two options to deal with this are to set negative values to zero, creating intermittent resource vacuums, or to scale the function's values to an acceptable range, such as [0, 1].
% The two implementations may lead to different outcomes for the evolution of perception and movement distances.
% For example, in the first case with intermittent resource vacuums, individuals initialised in these areas will have few or no resource cues upon which to base their decisions.
% In the latter case, the resource gradient is maintained, as at the lowest resource values, individuals can use cues in surrounding areas to move to better locations.

% The same issues apply to two dimensional landscapes; however, the functions used to generate two dimensional spatial heterogeneity can be much more complex.
% Functions that were developed in the visual effects realm to create realistic variation on two-dimensional surfaces, such as \textit{Noise} \citep{perlin}, Simplex Noise \citep{simplexnoise}, and OpenSimplexNoise \citep{opensimplexnoise}, are good candidates for functions that can create smooth spatial variation on two dimensional landscapes.
% Noise has been used to represent [CITE WHERE NOISE HAS BEEN USED].

% Advantages and disadvantages of continuous resource implementations --- advantages --- follow the assumptions of continuous input models, relatively easy to implement other assumptions such as interference competition and variations thereof \citep{tregenza1995,vandermeer1997} --- easy to scale intake as a function of the number of neighbours, or as a function of another layer of the landscape --- Noise and related implementations are almost indefinitely extensible in space --- multiple pre-built implementations are available\\
% --- disadvantages --- difficult to implement resource depletion, which is key to exploitation competition --- including complex noise functions in simulation models is not easy and requires substantial programming skills --- Noise and other two dimensional functions look realistic but should be remembered as abstract representations of natural patterns.

% Models which represent resources as discrete items (as in Chapters~\ref{ch:kleptomove} and \ref{ch:pathomove}) typically treat each item similarly to individuals --- in the parlance of programming, as an object with certain attributes and functions.
% How are item impemented --- how should they be implemented --- what are the options --- how are they different from continuous input/resource models --- what can be done with discrete resources that cannot be done with continuous input --- e.g. exploitation competition due to item depletion --- handling time and regeneration time for easy implementation of theoretical costs (e.g. costs of grouping near resources) --- rgeneration time allows landscape heterogeneity even without specific spatial structure, and allows individuals to shape their landscape --- what are the advantages --- what are the disadvantages [MAYBE split into two paras]

% Combinations of discrete and continuous resource landscapes --- briefly mention with examples, recommend against as they are difficult to combine.

% \subsection*{Modelling time}

% Animal behaviour also has a temporal context, which, similar to the spatial component of models, can be modelled as discrete or continuous.
% Classic theoretical models have considered both continuous time --- in the form of the rate of a process (such as intake -- [CITE], or diffusive movement -- [CITE]) --- and also discrete time in very small units \citep[e.g.][]{hamilton1971}.
% In contemporary individual-based models, continuous time is useful when modelling individuals that have intrinsic rates of some behaviour, such as scanning the landscape for a suitable movement destination \citep{netz2022}, and are hence called event-based models.
% The passage of time in such models is implemented using the Gillespie algorithm, which draws the time until the next `event' --- such as an individual's action --- from an exponential distribution whose mean is $\Sigma~r_i$, where $r_i$ is the individual rate of that action \citep{gillespie198x}.
% An individual is then picked, with the probability of being picked proportional to its rate of engaging in the action, and the picked individual then undergoes a state change, such as movement, finding a food item, or acquiring information \citep{netz2022}.
% In these models, individuals move asynchronously, i.e., not all at the same time.
% This is realistic, as no two individuals can move at exactly the same instant.
% However, because system components change asynchronously, it becomes necessary to recalculate individuals' local environment after each event.
% For example, if an individual in a population moves, the neighbour count of \emph{all} individuals would have to be updated to account for this change in local density.
% Since even-based models can have a variable number of events, each potentially requiring an update of system properties, they can add substantial computational overhead to the speed of a simulation.
% Therefore, these models, while not challenging to implement, do require discretised time-logging if data on individual states (such as energy, or associations with other individuals) are to be returned in tabular format of predictable size.

% On the other hand, discrete time models may be somewhat easier to understand as they are similar to common empirical animal tracking datasets --- the state of the system and its components potentially changes in a series of discrete steps. {\color{red} COME BACK TO THIS.}


% \subsection*{Modelling animal movement}

% What is the classical way of modelling animal movement --- movements between patches --- what are the assumptions (ideal, free) --- what are some other spatially explicit ways in which this can be implemented --- random walks etc. this section will focus on spatially explicit landscapes and not implicit ones

% Modelling random walks --- how to implement them --- parameters of the random walk --- combining two different random walks (e.g. levy flight and brownian motion) into a multi-stage movement process shaped by the environment --- including evolutionary dynamics in random walk parameters as attributes --- including evolutionary dynamics of other parts of the movement process, such as the explore-exploit tradeoff as a switch between random walks, or as the duration of a specific random walk phase (speigel 2017 paper)

% Modelling directed movement --- the optimal movement approach \citep{scherer2020} --- why is this not always suitable: assumes the importance of various cues to animals, and that the most important cues are captured in the model --- alternative: individuals use id-speicfic attributes to assign suitability and move to the most suitable location --- cite own chapters, cite also white et al and amt modelling of utilisation distribution \citep{signer2017,signer2019}. 
% Individual-specific measures of suitability are evolved traits and should be outcomes of natural selection --- include evolution in ecological models (see below).

% Movement decisions in continuous space vs discrete space --- movement in discrete space is relatively constrained (in spatially explicit models) --- movement in the Moore neighbourhood has fixed steps --- movement in larger neighbourhoods is again a simple comparison of a discrete number of choices --- movement on continuous landscapes can be of two types, choosing between fixed locations, and choosing a movement distance and angle --- in both these options, perception distance is important [cite EVO PERCEPTION DISTANCE PAPER, PNAS paper from iain's lab on insect movement?] --- choosing a distance and angle can be done with simple linear functions, but this allows individuals little flexibility --- using neural networks to process local cues into a movement step is more complex, but possibly more realistic \cite*{mueller2011} --- State dependent movement, memory dependent movement can also be included --- maybe should be an entire section

% \subsection*{Modelling ecological interactions}

% Ecological interactions may be between individuals, or between individuals and their environment.
% Interactions between individuals and the environment, including resources, are covered above.
% Inter-individual interactions can be of as many different types as there are real, biological interactions, but for modelling purposes, they fall into two major categories: direct modification of interacting individual attributes, or modification of individual environment which has indirect effects on individual attributes later.

% The classic example is different types of competition, which are covered in chapter \ref{ch:kleptomove}.
% When modelling interactions, the type of landscape implemented imposes constraints on what kinds of interactions are possible (e.g. interactions between individuals on different grid cells in a discrete space model) --- and thus the entire model must be conceptualised in such as a way that the overall implementation suits the study.

% \subsection*{Translating ecological interactions into evolutionary consequences}

% \paragraph*{Intake and energy as proxies of fitness} 

% Theoretical models of animal movement and population distribution have an implicit evolutionary context --- movement and distributions are supposed to maximise fitness --- fitness is usually proxied by intake (chapters chapters).

% Fitness modifiers typically involve negative interactions with environment and other individuals --- environment may have areas where the individual net expends energy (similar to ODBA landscapes or energetic landscapes, see recent JAE paper luca borger).

% Negative interactions with other individuals can be competitive (interference, though it does not directly reduce fitness it does cause a time loss which results in lower intake see ch kleptomove) --- can also be predatory, where fitness goes to zero --- can also be a steady reduction in fitness after the transmission of an infectious pathogen (chapter pathomove).

% Mortality --- how to handle mortality? mortality alters the system and is challenging to implement, and has not been considered in models thus far --- mortality reduces indirect information available in a system and has consequences for social processes --- imposing an intake or energy threshold on survival is usually better suited to detailed predictive models rather than simpler conceptual models, where the link between intake, energy, and survival is better understood and often the focus of the model \citep{stillman2010}.
% this means that most conceptual models will have fixed population sizes --- what are the theoretical and biological bases for this.

% \paragraph*{Translating fitness proxies into trait frequencies}

% genetic Algorithms --- genetic annealing --- discuss \cite*{getz2015,getz2016} --- see kleptomove discussion and especially the part which was cut out from the first version --- approaches from computer science and the study of artificial life --- winner takes all etc.

% the alernative is the weighted lottery approach --- ref chapters kleptomove, pathomove, predator prey.

% \section*{Relating individual-based models with empirical approaches in movement ecology}

\section*{Conclusion: Where are we now, and where is focus necessary?}
