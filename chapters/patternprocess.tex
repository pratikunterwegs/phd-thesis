%************************************************
\chapter{Using a Mechanistic Model to Probe Statistical Methods in Animal Movement}\label{ch:patternprocess}
%************************************************

\noindent \textbf{Pratik R. Gupte} and Franz J. Weissing

\section*{Abstract}

\small{
    Movement ecologists have taken up the challenge of inferring animals' decision-making mechanisms in a spatial context from individual tracking data.
    The implicit assumption is that differences in the movement paths of animals reflect differences in individual decision-making mechanisms.
    However, animal movement takes place in complex and rapidly changing environments, where movement cues are not always available, and animals may differ along multiple axes of behaviour.
    Mechanistic, individual-based modelling of animal decision-making can help investigate whether differences in decision-making mechanisms actually translate into differences in movement paths, and the insights gained by parsing animal tracking data using contemporary statistical methods.
    Here, we examine the movement paths of agents from an evolutionary individual-based model of foraging competition, in which relatively simple movement rules are determined by evolved decision-making weights.
    To show how such a model can be used to investigate statistical methods, we explore a contemporary question in movement ecology: Can individual differences in movement decision-making mechanisms be detected from the emergent properties of the resulting movement paths?
    % First, we examine whether our model individuals' movement types differ in the structure of their movement paths.
    Using data on the movement of evolved model agents, we show how adopting a repeatability framework to quantify individual-differences in movement is sensitive to the evolutionary context in which movement rules evolve.
    We also find that repeatability analysis can yield very different conclusions depending on how individuals' behavioural types are accounted for.
    We also show that step-selection analysis can indicate differences between competition strategies, but rarely captures differences between movement types of the same competition strategy.
    Overall, using a plausible eco-evolutionary model of animal decision-making, we highlight some challenges in using contemporary statistical methods to infer individual differences in animals' decision-making mechanisms from positioning data.

    \medskip

    % \noindent {\large{\color{Maroon}$\Delta$}} Published in the \textit{Journal of Animal Ecology} as Gupte et al. (2021). A guide to pre-processing high throughput tracking data.
}

\clearpage
