%************************************************
\chapter{Rapid Evolution of Movement Strategies Following Novel Pathogen Introduction}\label{ch:pathomove}
\chaptermark{Disease \& Movement}
%************************************************

{\noindent \textbf{Pratik R. Gupte}, Gregory F. Albery\textsuperscript{1}, Jakob R.L. Gismann\textsuperscript{2}, Amy Sweeny\textsuperscript{3} and Franz J. Weissing\textsuperscript{2}}

\graffito{

    {\color{Maroon}\normalsize\headerfont{Co-author Affiliations}}

    \medskip

    \textsuperscript{1} Wissenschaftskolleg zu Berlin, Germany.
    
    \medskip
    
    \textsuperscript{2} University of Groningen, The Netherlands.
    
    \medskip
    
    \textsuperscript{3} University of Edinburgh, U.K.

    \medskip

    {\color{Maroon}\normalsize\headerfont{Funding}}

    \medskip

    Dutch Research Council (NWO)

    \medskip

    European Research Council
}

\section*{Abstract}

{
    % \small
    Animal social interactions are the outcomes of evolved strategies that integrate the costs and benefits of being sociable.
    % Using a novel mechanistic, evolutionary, individual-based simulation model, we examine how animals balance the risk of pathogen transmission against the benefits of social information about resource patches, and how this determines the emergent structure of socio-spatial networks.
    We study a scenario in which a fitness-reducing infectious pathogen is introduced into a population which has initially evolved movement strategies in its absence.
    Within only a few generations, pathogen introduction provokes a rapid evolutionary shift in animals' social movement strategies, and the importance of social cues in movement decisions increases.
    Individuals undertake a dynamic social distancing approach, trading more movement (and less intake) for lower infection risk.
    Pathogen-adapted populations disperse more widely over the landscape, and thus have less clustered social networks than their pre-introduction, pathogen-naive ancestors.
    Running epidemiological simulations on these emergent social networks, we show that diseases do indeed spread more slowly through pathogen-adapted animal societies.
    Finally, the mix of post-introduction strategies is strongly influenced by a combination of landscape productivity, the usefulness of social information, and disease cost.
    Our model suggests that the introduction of an infectious pathogen into a population can trigger a rapid eco-evolutionary cascade, rapidly changing animals' social movement strategies, which alters movement decisions and encounters between individuals. 
    In turn, this changes emergent social structures, and our model informs how such change can make populations more resilient to future disease outbreaks.
    Overall, we offer both a modelling framework and initial predictions for the evolutionary and ecological consequences of wildlife pathogen spillover scenarios.

    \bigskip

    {\noindent \large{$\Delta$}} A preprint submitted to \textit{Nature Comunications}.
}

\clearpage