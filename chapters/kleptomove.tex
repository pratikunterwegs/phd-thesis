%************************************************
\chapter{The Joint Evolution of Animal Movement and Competition Strategies}\label{ch:kleptomove}
%************************************************

\noindent \textbf{Pratik R. Gupte}, Christoph F.G. Netz, and Franz J. Weissing

\section*{Abstract}

\footnotesize{
    Competition typically takes place in a spatial context, but eco-evolutionary models rarely address the joint evolution of movement and competition strategies. 
    Here we investigate a spatially explicit forager-kleptoparasite model where consumers can either forage on a heterogeneous resource landscape, or steal resource items from conspecifics (kleptoparasitism). 
    We consider three scenarios: (1) foragers without kleptoparasites; (2) consumers specializing as foragers or as kleptoparasites; and (3) consumers that can switch between foraging and kleptoparasitism depending on local conditions.
    We model movement strategies as individual-specific combinations of preferences for environmental cues, similar to step-selection coefficients.
    By means of mechanistic, individual-based simulations, we study the joint evolution of movement and competition strategies, and we investigate the implications on the resource landscape and the distribution of consumers over this landscape.
    Movement and competition strategies evolve rapidly and consistently across all scenarios, with marked differences among scenarios, leading to differences in resource exploitation patterns.
    % The evolved movement and resource exploitation patterns differ considerably across the three scenarios.
    In scenario 1, foragers evolve considerable individual variation in movement strategies, while in scenario 2, movement strategy is tightly correlated with competition strategy, with a swift divergence between foragers and kleptoparasites.
    When individuals' competition strategy is conditional on local cues, movement strategies converge to facilitate kleptoparasitism, and individual consistency in competition strategy also emerges.
    Across scenarios, the distribution of consumers over resources differs substantially from `ideal free' predictions. 
    This is related to the intrinsic difficulty of moving effectively on a depleted resource landscape with few reliable cues for movement.
    Our study emphasises the advantages of a mechanistic approach when studying competition in a spatial context, and suggests how evolutionary modelling can be integrated with current work in animal movement ecology.

    \medskip

    \noindent {\large{\color{Maroon}$\Delta$}} Under review at \textit{The American Naturalist} as Gupte et al. The joint evolution of animal movement and competition strategies.
}

\clearpage
