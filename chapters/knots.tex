%************************************************
\chapter{Individual Consistency in Movement Tendencies Across Spatial Scales}\label{ch:knots}
%************************************************

{\noindent \sffamily Selin Ersoy, \textbf{Pratik R. Gupte}, Christine E. Beardsworth, and Allert I. Bijleveld}

\section*{Abstract}

% \graffito{
%     \bigskip

%     {\large{$\Delta$}} Under review at The American Naturalist as Gupte et al. The joint evolution of animal movement and competition strategies.
% }
Foraging is essential for many species and understanding factors affect foraging decision is a core topic in ecology.
Patch-use theory examines how animals make decisions on when to approach and leave a particular foraging patch.
Even though the individual factors associated with patch movement such as searching and processing the food item or digestive costs were studied under this theory, consistent individual differences (also knows as personalities) were understudied.
Here we investigated if exploratory personality measured in controlled settings predicts patch movement in the field on red knots.
We first assayed exploratory tendency of red knots in controlled settings (notably without food) and then release the same birds with transmitters to investigate foraging patch movements in the field.
We asked how exploration speed in the controlled settings relates to within and between patch movement of an individual during low tide.
We found that faster exploring red knots visit more patches during low tide and they move faster and stay shorter within the patch.
The size of the foraging patch did not differ between individuals with different exploratory scores.
Our findings suggest that exploration measured in captivity predicts exploration in the field, and further provides a framework to study and understand how personality variation may be maintained in populations.

\noindent {\color{red} WORK IN PROGRESS}

\clearpage
