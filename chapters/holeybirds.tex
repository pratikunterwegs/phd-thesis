%************************************************
\chapter{Direct Effects of Wing Moult on the Movement and Habitat Selection of Sub-tropical Birds}\label{ch:holeybirds}
%************************************************

\noindent \textbf{Pratik R. Gupte}, Yosef Kiat\textsuperscript{1}, Sivan Toledo\textsuperscript{2}, and Ran Nathan\textsuperscript{3}

\marginpar{
    \textsuperscript{1} University of Haifa, Israel.
    
    \medskip
    
    \textsuperscript{2} Tel Aviv University, Israel.
    
    \medskip

    \textsuperscript{3} The Hebrew University of Jerusalem, Israel.
}

\section*{Abstract}

% \marginpar{ 
%     \bigskip

%     {\large{\color{Maroon}$\Delta$}} \normalfont Published in the \textit{Journal of Animal Ecology} as Gupte et al. (2021). A guide to pre-processing high throughput tracking data.
% }

\small{
    The feathered flight of birds requires regular renewal of the wing surface through moult, the process of shedding worn out feathers and growing fresh ones.
    Moult presents birds with the dilemma of needing to move more to acquire resources for feather growth, at precisely the time that their flight capacity is reduced (due to missing wing feathers) and they are vulnerable to predation.
    Despite this central importance of moult to avian ecology, and especially to movement, we know little about its direct effects on the spatial ecology of birds.
    We combined a range of mechanistic approaches to present a first quantification of the direct, short-term effects of wing moult on the movement and habitat selection of four different non-migratory, sub-tropical birds.
    We quantified landscape visibility from a predator viewpoint, and followed the movement of birds in different stages of moult, with a high-throughput position tracking system.
    We found that birds balanced the needs and risks of moult by adjusting their movement to their wing condition.
    Naturally moulting birds moved more than non-moulting birds, but birds with wing feathers experimentally removed moved less.
    Among similarly vegetated areas, birds preferred low-visibility sheltered sites in our agricultural landscape, regardless of their moult status.
    Intriguingly, our results suggest that birds' cognitive abilities may extend to seeing a landscape from the spatial perspective of another individual.   
}

\clearpage

