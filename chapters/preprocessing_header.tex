    
%************************************************
\chapter{Pre-processing High Throughput Animal Tracking Data}\label{ch:preprocessing}
\chaptermark{Pre-processing Animal Tracking Data}
%************************************************
% 
{\noindent \sffamily\textbf{Pratik R. Gupte}, Christine E. Beardsworth\textsuperscript{1}, Orr Spiegel\textsuperscript{2}, Emmanuel Lourie\textsuperscript{3}, Sivan Toledo\textsuperscript{2}, Ran Nathan\textsuperscript{3}, and Allert Bijleveld\textsuperscript{1}}

\marginpar{
	\sffamily
    \textsuperscript{1} Netherlands Inst. for Sea Research, The Netherlands.
    
    \medskip
    
    \textsuperscript{2} Tel Aviv University, Israel.
    
    \medskip

    \textsuperscript{3} The Hebrew University of Jerusalem, Israel.
}

\section*{Abstract}
\marginpar{ 
    \bigskip

    {\large{$\Delta$}} \normalfont Published in the \textit{Journal of Animal Ecology} as Gupte et al. (2021). A guide to pre-processing high throughput tracking data.
}
{
    \small
    	
    Modern, high-throughput animal tracking increasingly yields `big data' at very fine temporal scales, and 
    % At these scales, location error can exceed the animal's step size, leading to mis-estimation of behaviours inferred from movement. 
    `cleaning' the data to reduce location errors is one of the main ways to deal with position uncertainty. 
    Though data cleaning is widely recommended, inclusive, uniform guidance on this crucial step, and on how to organise the cleaning of massive datasets, is relatively scarce.
    A pipeline for cleaning massive high-throughput datasets must balance ease of use and computationally efficiency, in which location errors are rejected while preserving valid animal movements. 
    % Another useful feature of a pre-processing pipeline is efficiently segmenting and clustering location data for statistical methods, while also being scalable to large datasets and robust to imperfect sampling. 
    Manual methods being prohibitively time consuming, and to boost reproducibility, pre-processing pipelines must be automated.
    We provide guidance on building pipelines for pre-processing high-throughput animal tracking data to prepare it for subsequent analyses. 
    We apply our proposed pipeline to simulated movement data with location errors, and also show how large volumes of cleaned data can be transformed into biologically meaningful `residence patches', for exploratory inference on animal space use. 
    We use tracking data from the Wadden Sea ATLAS system (WATLAS) to show how pre-processing improves its quality, and to verify the usefulness of the residence patch method. 
    Finally, with tracks from Egyptian fruit bats \textit{Rousettus aegyptiacus}, we demonstrate the pre-processing pipeline and residence patch method in a fully worked out example.
    To help with fast implementation of standardised methods, we developed the R package \textit{atlastools}, which we also introduce here. 
    Our pre-processing pipeline and \textit{atlastools} can be used with any high-throughput animal movement data in which the high data-volume combined with knowledge of the tracked individuals’ movement capacity can be used to reduce location errors. 
    % \textit{atlastools} is easy to use for beginners, while providing a template for further development. 
    % The common use of simple yet robust pre-processing steps promotes standardised methods in the field of movement ecology and leads to better inferences from data.
}

\clearpage
