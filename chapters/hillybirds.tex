%************************************************
\chapter{Land-cover, Climate, and Bird Occupancy Within a Tropical Biodiversity Hotspot}\label{ch:hillybirds}
%************************************************
% Using citizen science to parse climatic and land cover influences on

\noindent Vijay Ramesh\textsuperscript{1}, \textbf{Pratik R. Gupte}, Morgan Tingley\textsuperscript{2}, V.V. Robin\textsuperscript{3}, and Ruth S. de Fries\textsuperscript{1}

\marginpar{
    \textsuperscript{1} Columbia University, USA.
    
    \medskip
    
    \textsuperscript{2} University of California --- Los Angeles, USA.
    
    \medskip

    \textsuperscript{3} Indian Institute for Science Education and Research --- Tirupati, India.
}

\section*{Abstract}

\marginpar{ 
    \bigskip

    {\large{\color{Maroon}$\Delta$}} \normalfont Under review at \textit{Ecography} as Ramesh et al. Using citizen science to parse climatic and land cover influences on bird occupancy within a tropical biodiversity hotspot.
}

\small{
    Disentangling associations between species occupancy and its environmental drivers --- climate and land cover --- along tropical mountains is imperative to predict species distributional changes in the future. 
    Previous studies have largely focused on identifying such associations along temperate mountain systems. 
    Using robustly filtered citizen science observations, we examined the role of climatic and landscape variables and its association with species occurrence within a tropical biodiversity hotspot. 
    We used over 1.1 million citizen scientist observations contributed to eBird between 2013 and 2020 for 82 species of birds across the southern Western Ghats in India and modeled the regional distribution of each species within an occupancy modeling framework. 
    Our results show that mean variation in temperature (defined as temperature seasonality), presence of evergreen forests, and presence of deciduous forests were significantly associated with species-specific probabilities of occupancy for 79\% (n=63 birds), 45\% (n=36 birds) and 17\% (n=14 birds) of bird species examined, respectively. 
    Forest specialists were largely sensitive to temperature seasonality and were negatively associated with increasing mean variation in temperature. 
    Human-modified land cover types --- such as the proportion of agriculture/settlements, plantations, and mixed/degraded forests --- were largely negatively associated with the occupancy of forest species, while showing a positive association for many generalist birds. 
    Our study shows that rigorously filtered citizen science observations can be used to identify associations between environmental drivers and species occupancy on tropical mountains. 
    Though current distributions of tropical montane birds of the Western Ghats are strongly driven by climatic factors --- chiefly temperature seasonality --- naturally occurring land cover types including forests are critical to sustain montane avifauna across human-modified landscapes in the long run.
}

\clearpage

