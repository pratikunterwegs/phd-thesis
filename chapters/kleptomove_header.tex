\begin{refsection}
%************************************************
\chapter{The Joint Evolution of Animal Movement and Competition Strategies}\label{ch:kleptomove}
\chaptermark{Competition and Movement}
%************************************************

{\noindent \sffamily\textbf{Pratik R. Gupte}, Christoph F.G. Netz, and Franz J. Weissing}

\section*{Abstract}

\graffito{
    \bigskip

    {\large{$\Delta$}} Under review at The American Naturalist as Gupte et al. The joint evolution of animal movement and competition strategies.
}
{
    \small
    Competition typically takes place in a spatial context, but eco-evolutionary models rarely address the joint evolution of movement and competition strategies.
    Here we investigate a spatially explicit forager-kleptoparasite model where consumers can either forage on a heterogeneous resource landscape, or steal resource items from conspecifics (kleptoparasitism). We consider three scenarios: (1) foragers without kleptoparasites; (2) consumers specializing as foragers or as kleptoparasites; and (3) consumers that can switch between foraging and kleptoparasitism depending on local conditions.
    We model movement strategies as individual-specific combinations of preferences for environmental cues, similar to step-selection coefficients.
    Using mechanistic, individual-based simulations, we study the joint evolution of movement and competition strategies, and we investigate the implications for the distribution of consumers over this landscape.
    Movement and competition strategies evolve rapidly and consistently across scenarios, with marked differences among scenarios, leading to differences in resource exploitation patterns.
    In scenario 1, foragers evolve considerable individual variation in movement strategies, while in scenario 2, movement strategy shows a swift divergence between foragers and kleptoparasites.
    When individuals' competition strategy is conditional on local cues, movement strategies facilitate kleptoparasitism, and individual consistency in competition strategy also emerges.
    Across scenarios, the distribution of consumers differs substantially from `ideal free' predictions.
    This is related to the intrinsic difficulty of moving effectively on a depleted resource landscape with few reliable movement cues.
    Our study emphasises the advantages of a mechanistic approach when studying competition in a spatial context, and suggests how evolutionary modelling can be integrated with current work in animal movement ecology.
}

\subsection*{Data and Code}
{
    \small
    Simulation model: github.com/pratikunterwegs/Kleptomove.\\ %and Zenodo: zenodo.org/record/4905476. 
    \noindent Simulation data (DataverseNL): doi.org/10.34894/JFSC41.\\
    \noindent Data analysis code: github.com/pratikunterwegs/klepto-move-evol.% and on Zenodo: doi.org/10.5281/zenodo.4904497.
}

\clearpage
