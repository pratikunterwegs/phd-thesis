% !TEX program = lualatex

\RequirePackage{silence} % :-\
    \WarningFilter{scrreprt}{Usage of package `titlesec'}
    %\WarningFilter{scrreprt}{Activating an ugly workaround}
    \WarningFilter{titlesec}{Non standard sectioning command detected}
\documentclass[ twoside,openright,titlepage,numbers=noenddot,%1headlines,
                headinclude,footinclude,cleardoublepage=empty,abstract=on,
                BCOR=5mm,paper=b5,fontsize=10pt,dvipsnames
                ]{scrbook}
\renewcommand\textsc[1]{{\fontfamily{put}\fontshape{sc}\selectfont#1}}

% use some packages
\usepackage{multicol}
\usepackage[x11names]{xcolor}
\usepackage{graphbox}

%********************************************************************
% Note: Make all your adjustments in here
%*******************************************************
\input{config}

%********************************************************************
% Bibliographies
%*******************************************************
% \RequirePackage[authoryear,sort]{natbib}

% \usepackage{citation-style-language}
% \cslsetup{style = apa}
\addbibresource{chapters/preprocessing.bib}
\addbibresource{chapters/kleptomove.bib}
% \addbibresource{Bibliography.bib}
% \addbibresource[label=ownpubs]{AMiede_Publications.bib}

%********************************************************************
% Hyphenation
%*******************************************************
%\hyphenation{put special hyphenation here}

%\usepackage{sourcesanspro}   % scaling is recommended
\usepackage{fontspec}
% \usepackage[xcharter]{newtxmath}
% \setmainfont{Vollkorn}[
%   Scale=0.9274,% to match the math fonts
%   Extension=.otf,
%   UprightFont=*-Regular,
%   ItalicFont=*-Italic,
%   BoldFont=*-Bold,
%   BoldItalicFont=*-BoldItalic,
% ]
% \setmainfont{Source Serif Pro}
% \setmainfont{Libre Baskerville}
% \usepackage[usefilenames,RMstyle={Text,Semibold},SSstyle={Text,Semibold},TTstyle={Text,Semibold},DefaultFeatures={Ligatures=Common}]{plex-otf}
\setmainfont{Plantin MT Pro}[Extension = .ttf, UprightFont = *Rg,BoldFont = *Bold,ItalicFont = *RgIt]
\setmathfont{Erewhon Math}
\setsansfont{Oswald}[Extension = .ttf, UprightFont = *-Light,BoldFont = *-Regular]
% \setsansfont{Libre Franklin}
% \setmathfont{TeX Gyre Schola Math}
% \usepackage{unicode-math}
% \setmathfont{XCharter}
\newfontfamily\mysans{Source Sans Pro}

\renewcommand*{\bibfont}{\footnotesize}

\usepackage{setspace}

\setlength{\parindent}{1.5em}
% \setlength{\parskip}{0.5em}

\renewcommand{\chapterNumber}{%
  \color{Black}\fontsize{60}{60} \bfseries}\par

\usepackage{enumitem}
\setlist[description]{leftmargin=\parindent,labelindent=\parindent}

\makeatletter
% fix part heading format
\titleformat{\part}[display]
        {\ct@altfont\centering\LARGE\sffamily}%
        {\thispagestyle{empty}\scshape\sffamily\partname~\MakeTextUppercase{\thepart}}{1em}%
        {\color{Gray}\scshape\bfseries\huge\sffamily}[\bigskip\normalfont\normalsize\color{Black}\begin{quote}\ctparttext@print\end{quote}]
        
% fix chapter heading format
\titleformat{\chapter}[display]%
    % \setbox0=\hbox{\chapterNumber\thechapter\hspace{10pt}\vline\ }%
    {\relax}{\mbox{}\oldmarginpar{\chapterNumber\sffamily\thechapter}}{0pt}%
    {\raggedright\huge\sffamily\bfseries}[\normalsize\vspace*{.8\baselineskip}\titlerule]%

% fix section format
\titleformat{\section}
    {\relax}{\thesection}{1em}{\large\sffamily\bfseries}

% fix section format
\titleformat{\subsection}
    {\relax}{\thesubsection}{1em}{\large\sffamily}

% fix section format
\titleformat{\paragraph}
    {\relax}{\theparagraph}{1em}{\sffamily}

% \usepackage[left]{lineno}
\usepackage[skins,breakable]{tcolorbox}
\usepackage{lipsum}

% parts in toc
\renewcommand{\cftpartfont}{\scshape\sffamily}%
% chapters in toc
\renewcommand{\cftchappresnum}{\sffamily}%
\renewcommand{\cftchapaftersnumb}{\sffamily}%
\renewcommand{\cftchapfont}{\sffamily}%
\renewcommand{\cftchappagefont}{\sffamily}%

% for stylistic elements
\usepackage{lettrine}
\usepackage{adforn}
\usepackage{afterpage}

\makeatother

\DeclareCaptionFont{quack}{\mysans}
\usepackage{caption}
\setcapindent{0pt}

\captionsetup{
  font={footnotesize},
  labelfont=bf,
  justification=justified,
  format=plain
}

% to format short titles
\renewcommand{\chaptermark}[1]{\markboth{\color{gray}\small\sffamily{Chapter \thechapter}}{\color{gray}\small\sffamily{#1}}}
\renewcommand{\sectionmark}[1]{\markright{\small\sffamily\MakeUppercase{\thesection}\enspace{#1}}}

% to format page numbers
\addtokomafont{pagenumber}{\color{gray}\small\sffamily}

% paragraph spacing
% \setlength{\parskip}{1em}

% ********************************************************************
% GO!GO!GO! MOVE IT!
%*******************************************************

\begin{document}

\raggedright
\frenchspacing
\raggedbottom
\setlength{\parindent}{2em}
\selectlanguage{american} % american ngerman
%\renewcommand*{\bibname}{new name}
%\setbibpreamble{}
% \pagenumbering{roman}
\pagestyle{plain}

%********************************************************************
% Frontmatter
%*******************************************************

% \include{FrontBackmatter/DirtyTitlepage}
%*******************************************************
% Cover Titlepage
%*******************************************************

\begin{titlepage}
    % \pagecolor{white}\afterpage{\nopagecolor}
    %\pdfbookmark[1]{\myTitle}{titlepage}
    % if you want the titlepage to be centered, uncomment and fine-tune the line below (KOMA classes environment)
    % \begin{addmargin}[-1cm]{-3cm}
    % \begin{center}
        \linespread{1.15}
        % \large

        % \vfill
        {
          \begin{flushleft}
            {\Huge \displayfont{Animal Movement Strategies}\par}\par
            \vspace{25mm}
            Pratik Rajan Gupte
          \end{flushleft}
        }
    % \end{center}
  \end{addmargin}
\end{titlepage}

% \nopagecolor

\thispagestyle{empty}

\hfill

\vfill

\noindent {\textsf{COLOPHON}}

This document was typeset based on \emph{classicthesis}, developed by Andr\'e Miede and Ivo Pletikosić.
The style was inspired by Robert Bringhurst's seminal book on typography ``\emph{The Elements of Typographic Style}''.
Inspired by Wouter Vahl's PhD thesis ``\emph{Interference Competition in Foraging Waders}'', Matthew Carter's Charter is used for the main text; Vernon Adams' Oswald is used for headings.
The cover design is inspired by the colours of the Japanese edition of Theodore M. Porter's ``\emph{Trust the Numbers}''.

\bigskip

\noindent\finalVersionString

\noindent\myName. \textit{\myTitle.}% \mySubtitle, %\myDegree,
\\
\textcopyright\ \today

%\bigskip
%
%\noindent\spacedlowsmallcaps{Supervisors}: \\
%\myProf \\
%\myOtherProf \\
%\mySupervisor
%
%\medskip
%
%\noindent\spacedlowsmallcaps{Location}: \\
%\myLocation
%
%\medskip
%
%\noindent\spacedlowsmallcaps{Time Frame}: \\
%\myTime

%*******************************************************
% RUG defined Titlepage
%*******************************************************
\thispagestyle{empty}
\begin{titlepage}
    \thispagestyle{empty}
    %\pdfbookmark[1]{\myTitle}{titlepage}
    % if you want the titlepage to be centered, uncomment and fine-tune the line below (KOMA classes environment)
    \begin{addmargin}[-1cm]{-3cm}
    \begin{center}
        % \linespread{1.5}
        % \large

        \hfill
        \begin{figure*}
            \includegraphics[width=7.38cm,height=2.03cm]{figures/rug_logo.eps}
        \end{figure*}

        \vspace{12mm}

        {
           {\fontsize{30}{30} \bfseries{Animal Movement\\Strategies}\par}\par
        }

        \vspace{21mm}

        {
           {\fontsize{15}{15} \bfseries{PhD thesis}\par}\par
        }

        \vspace{12mm}

        {\fontsize{14}{15}
            {to obtain the degree of PhD at the\\
            University of Groningen\\
            on the authority of the\\
            Rector Magnificus Prof. C. Wijmenga\\
            and in accordance with\\
            the decision by the College of Deans.}

            \vspace{3mm}

            This thesis will be defended in public on

            \vspace{3mm}

            19 August 2022 at 11:00 hours

            \vspace{12mm}

            by

            \vspace{12mm}

            \textbf{Pratik Rajan Gupte}

            \vspace{3mm}

            born on 22 September 1993\\
            in Hyderabad, India
        }

    \end{center}

    \pagebreak
    \thispagestyle{empty}

    \textbf{Supervisors}
    \begin{description}
        \item Prof. Dr. Franz J. Weissing
    \end{description}

    \vspace{6mm}

    \textbf{Co-supervisor}
    \begin{description}
        \item {\color{red} ADD HERE}
    \end{description}

    \vspace{6mm}

    \textbf{Assessment committee}
    \begin{description}
        \item {\color{red} ADD HERE}
        \item {\color{red} ADD HERE}
        \item {\color{red} ADD HERE}
    \end{description}

  \end{addmargin}

\end{titlepage}


    %\pdfbookmark[1]{\myTitle}{titlepage}
    % if you want the titlepage to be centered, uncomment and fine-tune the line below (KOMA classes environment)
    


\clearpage%*******************************************************
% Propositions
%*******************************************************
%\renewcommand{\abstractname}{Abstract}
\pdfbookmark[1]{Propositions}{Propositions}
% \addcontentsline{toc}{chapter}{\tocEntry{Abstract}}
\begingroup
% \let\clearpage\relax
% \let\cleardoublepage\relax
% \let\cleardoublepage\relax

\chapter*{Propositions}

\begin{onehalfspace}
    
    \begin{enumerate}
        \item \textit{What are birds if not dinosaurs persevering.\\--- Twitter, paraphrasing WandaVision}
        \item Animal movement ecology needs to adopt best-practices from other big-data disciplines.\\ --- \textit{Chapter \ref{ch:preprocessing}}.
        \item Animals' movement decisions incorporate not only what they can see, but what they think other individuals can see.\\ --- \textit{Chapter \ref{ch:holeybirds}}.
        \item Mechanistic, individual-based simulation modelling of movement decisions opens the door to the evolutionary ecology of animal movement. \\ --- \textit{Chapter \ref{ch:kleptomove}}.
        \item Statistical tools in movement ecology are sensitive to the scale and mechanisms of individual variation.\\ ---\textit{Chapter \ref{ch:patternprocess}}.
        \item Rapid evolution in movement strategies can drastically reshape the structure of animal societies.\\ --- \textit{Chapter \ref{ch:pathomove}}.
        \item \textit{Chapter Sometimes it's better to light a flamethrower than curse the darkness.\\--- Terry Pratchet}
    \end{enumerate}

\end{onehalfspace}

\endgroup

\vfill

\clearpage

\cleardoublepage%*******************************************************
% Dedication
%*******************************************************
\thispagestyle{empty}
\phantomsection
\pdfbookmark[1]{Dedication}{Dedication}

\vspace*{3cm}

\begin{center}
    \large\emph{What are birds, if not dinosaurs persevering?}\\
    \medskip
    -- \small{Paraphrased from \textit{WandaVision}, 2021.}
\end{center}

% \medskip

% \begin{center}
%     Dedicated to the loving memory of Rudolf Miede. \\ \smallskip
%     1939\,--\,2005
% \end{center}

% \cleardoublepage\include{frontbackmatter/Colophon}
\cleardoublepage
\cleardoublepage%*******************************************************
% Table of Contents
%*******************************************************
\begingroup

\pagestyle{scrheadings}
% \thispagestyle{empty}
%\phantomsection
\pdfbookmark[1]{\contentsname}{tableofcontents}
\setcounter{tocdepth}{0} % <-- 2 includes up to subsections in the ToC
\setcounter{secnumdepth}{2} % <-- 3 numbers up to subsubsections
\manualmark
\markboth{\spacedlowsmallcaps{\contentsname}}{\spacedlowsmallcaps{\contentsname}}
\begin{doublespace}
    
    \raggedright
    \tableofcontents
    
    \automark[section]{chapter}
    \renewcommand{\chaptermark}[1]{\markboth{\spacedlowsmallcaps{#1}}{\spacedlowsmallcaps{#1}}}
    \renewcommand{\sectionmark}[1]{\markright{\textsc{\thesection}\enspace\spacedlowsmallcaps{#1}}}
\end{doublespace}

\endgroup

%*******************************************************
% List of Figures and of the Tables
%*******************************************************
% \pagestyle{empty} % Uncomment this line if your lists should not have any headlines with section name and page number
% \begingroup
%     \let\clearpage\relax
%     \let\cleardoublepage\relax
%     %*******************************************************
%     % List of Figures
%     %*******************************************************
%     %\phantomsection
%     %\addcontentsline{toc}{chapter}{\listfigurename}
%     % \pdfbookmark[1]{\listfigurename}{lof}
%     % \listoffigures

%     % \vspace{8ex}

%     %*******************************************************
%     % List of Tables
%     %*******************************************************
%     %\phantomsection
%     %\addcontentsline{toc}{chapter}{\listtablename}
%     \pdfbookmark[1]{\listtablename}{lot}
%     \listoftables

%     \vspace{8ex}
%     % \newpage

%     %*******************************************************
%     % List of Listings
%     %*******************************************************
%     %\phantomsection
%     %\addcontentsline{toc}{chapter}{\lstlistlistingname}
%     \pdfbookmark[1]{\lstlistlistingname}{lol}
%     \lstlistoflistings

%     \vspace{8ex}

%     %*******************************************************
%     % Acronyms
%     %*******************************************************
%     %\phantomsection
%     \pdfbookmark[1]{Acronyms}{acronyms}
%     \markboth{\spacedlowsmallcaps{Acronyms}}{\spacedlowsmallcaps{Acronyms}}
%     \chapter*{Acronyms}
%     \begin{acronym}[UMLX]
%         \acro{DRY}{Don't Repeat Yourself}
%         \acro{API}{Application Programming Interface}
%         \acro{UML}{Unified Modeling Language}
%     \end{acronym}

% \endgroup


\setcounter{page}{0}
\cleardoublepage%*******************************************************
% Abstract
%*******************************************************
%\renewcommand{\abstractname}{Abstract}
\pdfbookmark[1]{Abstract}{Abstract}
% \addcontentsline{toc}{chapter}{\tocEntry{Abstract}}
\begingroup
% \let\clearpage\relax
% \let\cleardoublepage\relax
% \let\cleardoublepage\relax

\chapter*{Abstract}

% \begin{center}
%     \emph{What are birds, if not dinosaurs persevering?}\\
%     \medskip
%     -- \small{Paraphrased from \textit{WandaVision}, 2021.}
% \end{center}

% Competition typically takes place in a spatial context, but eco-evolutionary models rarely address the joint evolution of movement and competition strategies. 
% Here we investigate a spatially explicit forager-kleptoparasite model where consumers can either forage on a heterogeneous resource landscape, or steal resource items from conspecifics (kleptoparasitism). 
% We consider three scenarios: (1) foragers without kleptoparasites; (2) consumers specializing as foragers or as kleptoparasites; and (3) consumers that can switch between foraging and kleptoparasitism depending on local conditions.
% We model movement strategies as individual-specific combinations of preferences for environmental cues, similar to step-selection coefficients.
% By means of mechanistic, individual-based simulations, we study the joint evolution of movement and competition strategies, and we investigate the implications on the resource landscape and the distribution of consumers over this landscape.
% Movement and competition strategies evolve rapidly and consistently across all scenarios, with marked differences among scenarios, leading to differences in resource exploitation patterns.
% % The evolved movement and resource exploitation patterns differ considerably across the three scenarios.
% In scenario 1, foragers evolve considerable individual variation in movement strategies, while in scenario 2, movement strategy is tightly correlated with competition strategy, with a swift divergence between foragers and kleptoparasites.
% When individuals' competition strategy is conditional on local cues, movement strategies converge to facilitate kleptoparasitism, and individual consistency in competition strategy also emerges.
% Across scenarios, the distribution of consumers over resources differs substantially from `ideal free' predictions. 
% This is related to the intrinsic difficulty of moving effectively on a depleted resource landscape with few reliable cues for movement.
% Our study emphasises the advantages of a mechanistic approach when studying competition in a spatial context, and suggests how evolutionary modelling can be integrated with current work in animal movement ecology.

% \begin{center}
% % \url{https://plg.uwaterloo.ca/~migod/research/beckOOPSLA.html}
% \end{center}

% \vfill

% \begin{otherlanguage}{ngerman}
% \pdfbookmark[1]{Zusammenfassung}{Zusammenfassung}
% \chapter*{Zusammenfassung}
% Kurze Zusammenfassung des Inhaltes in deutscher Sprache\dots
% \end{otherlanguage}

\endgroup

\vfill

\clearpage


%********************************************************************
% Mainmatter
% *******************************************************
\clearpage 
% \titleformat{\section}{\relax}{\textsf{\thesection}}{1em}{}
\pagestyle{scrheadings}
\pagenumbering{arabic}
% use \cleardoublepage here to avoid problems with pdfbookmark

\cleardoublepage 
\phantomsection
% \addtocontents{toc}{\protect\vspace{\beforebibskip}}%
\addcontentsline{toc}{chapter}{\tocEntry{\color{Maroon}\scshape\bfseries{Introduction}}}%
\chapter*{Introduction}

{\sffamily{Pratik R. Gupte}}


% \ctparttext{
%     Large datsets are revolutioning our understanding of animal movement, but preparing these data for inference, and integrating that preparation with existing methods, is key.
    
%     \medskip

%     \textit{Chapter 1} covers how data from a novel, high-throughput, reverse-GPS tracking system should be prepared, calling for a substantially more reproducible and automated approach. 
%     This guide introduces the summarising of data into `residence patches', a fast and simple way to cluster and segment tracking data.

%     \medskip

%     \textit{Chapter 2} takes a mechanistic look at the movement of moulting birds, which is surprisingly poorly understood.
%     This analysis combines the residence patch method of \textit{Chapter 1} with step-selection and viewshed analysis, to show how birds use sheltered habitats.
% }
% \part{Developing and Applying Methods for High-Resolution Tracking}

\cleardoublepage %************************************************
\chapter{High-throughput Tracking in Animal Movement Ecology}\label{ch:htme}
%************************************************



\cleardoublepage     
%************************************************
\chapter{Pre-processing High Throughput Animal Tracking Data}\label{ch:preprocessing}
\chaptermark{Pre-processing Animal Tracking Data}
%************************************************
% 
{\noindent \sffamily\textbf{Pratik R. Gupte}, Christine E. Beardsworth\textsuperscript{1}, Orr Spiegel\textsuperscript{2}, Emmanuel Lourie\textsuperscript{3}, Sivan Toledo\textsuperscript{2}, Ran Nathan\textsuperscript{3}, and Allert Bijleveld\textsuperscript{1}}

\marginpar{
	\sffamily
    \textsuperscript{1} Netherlands Inst. for Sea Research, The Netherlands.
    
    \medskip
    
    \textsuperscript{2} Tel Aviv University, Israel.
    
    \medskip

    \textsuperscript{3} The Hebrew University of Jerusalem, Israel.
}

\section*{Abstract}
\marginpar{ 
    \bigskip

    {\large{$\Delta$}} \normalfont Published in the \textit{Journal of Animal Ecology} as Gupte et al. (2021). A guide to pre-processing high throughput tracking data.
}
{
    \small
    	
    Modern, high-throughput animal tracking increasingly yields `big data' at very fine temporal scales, and 
    % At these scales, location error can exceed the animal's step size, leading to mis-estimation of behaviours inferred from movement. 
    `cleaning' the data to reduce location errors is one of the main ways to deal with position uncertainty. 
    Though data cleaning is widely recommended, inclusive, uniform guidance on this crucial step, and on how to organise the cleaning of massive datasets, is relatively scarce.
    A pipeline for cleaning massive high-throughput datasets must balance ease of use and computationally efficiency, in which location errors are rejected while preserving valid animal movements. 
    % Another useful feature of a pre-processing pipeline is efficiently segmenting and clustering location data for statistical methods, while also being scalable to large datasets and robust to imperfect sampling. 
    Manual methods being prohibitively time consuming, and to boost reproducibility, pre-processing pipelines must be automated.
    We provide guidance on building pipelines for pre-processing high-throughput animal tracking data to prepare it for subsequent analyses. 
    We apply our proposed pipeline to simulated movement data with location errors, and also show how large volumes of cleaned data can be transformed into biologically meaningful `residence patches', for exploratory inference on animal space use. 
    We use tracking data from the Wadden Sea ATLAS system (WATLAS) to show how pre-processing improves its quality, and to verify the usefulness of the residence patch method. 
    Finally, with tracks from Egyptian fruit bats \textit{Rousettus aegyptiacus}, we demonstrate the pre-processing pipeline and residence patch method in a fully worked out example.
    To help with fast implementation of standardised methods, we developed the R package \textit{atlastools}, which we also introduce here. 
    Our pre-processing pipeline and \textit{atlastools} can be used with any high-throughput animal movement data in which the high data-volume combined with knowledge of the tracked individuals’ movement capacity can be used to reduce location errors. 
    % \textit{atlastools} is easy to use for beginners, while providing a template for further development. 
    % The common use of simple yet robust pre-processing steps promotes standardised methods in the field of movement ecology and leads to better inferences from data.
}

\clearpage
 
% 
\newrefcontext[sorting=ynt]

    \lettrine{A}{nimal} movement is an adaptive, integrated response to multiple drivers, including internal state, life-history traits and capacities, biotic interactions, and other environmental factors \citep{nathan2008a, holyoak2008}.
    % Movement has both beneficial and detrimental consequences for individual fitness, and 
    The movement ecology framework links the drivers, processes, and fitness outcomes of animal movement \citep{nathan2008a}, and remotely tracking individual animals in the wild is the methodological mainstay of movement ecology \citep{wikelski2007,nathan2008a,hussey2015,kays2015}.
    A key challenge with observed tracks is to extract information on the behavioural, cognitive, social, ecological and evolutionary processes that shape animal movement.
    Addressing this challenge requires investigating the relationships between movement and its drivers at the fine scales at which animals sense and respond to variation in their environment. 
    Tracking data, which are observations of a continuous process (animal movement) at discrete timesteps, reveal useful information about the movement process when the tracking interval is considerably shorter than the typical duration of a movement mode \citep{nathan2008a, noonan2019, getz2008}.
    This can be accomplished by wildlife tracking systems that collect position data from many individuals at high temporal and spatial resolution (i.e., high-throughput tracking) relative to the scale of the movement mode of interest \citep{getz2008}.

    \graffito{
        Data repositories such as Movebank have much more data than can be processed, except in large, synthetic studies such as \citep{tucker2018}.
        See the Discussion for ideas for these pre-existing datasets.
    }
    High-throughput tracking technologies include GPS tags \citep{strandburg-peshkin2015, papageorgiou2019, harel2016, klarevas-irby2021}, tracking radars \citep{horvitz2014}, and computer vision methods for tracking entire groups of animals from video recordings \citep{rathore2020, perez-escudero2014}. 
    Furthermore, high-throughput wildlife tracking is routinely provided by terrestrial reverse-GPS systems such as ATLAS \citep[Advanced Tracking and Localization of Animals in real-life Systems:][]{toledo2014, weiser2016, toledo2016,toledo2020} --- see also \citep{maccurdy2009, maccurdy2019} --- and underwater acoustic reverse-GPS tracking of aquatic animals \citep{baktoft2019, baktoft2017, jung2015, aspillaga2021, aspillaga2021a}.
    Finally, low resolution tracking over a long duration may also capture important aspects of animal behaviour at certain time-scales \citep[e.g. migration, long-range dispersal;][]{getz2008}, thereby being `relatively' high-throughput.

    Although high-throughput tracking provides a massive amount of data on the path of a tracked animal, these data present a challenge to ecologists.
    When tracking animals at a high temporal resolution, the location error of each position may approach or exceed the true movement distance of the animal, compared to low-resolution tracking with the same measurement error.
    This leads to an over-estimation of the true distance moved by an animal between two discrete time-points, leading to unreliable behavioural metrics ultimately derived from movement distance, such as speed and tortuosity \citep[see][]{ranacher2016, noonan2019, hurford2009, calenge2009}.
    Additionally, the location error around a position introduces uncertainty when studying the relationship between animal movements and either fixed landscape features (e.g. roads), or mobile elements (e.g. other tracked individuals), as well as confounding estimates of habitat selection.

    Users have two main options to improve data quality, \textit{(i)} making inferences after modelling the system-specific location error using a continuous time movement model \citep{fleming2014a, fleming2020, jonsen2003, jonsen2005, johnson2008, patterson2008, aspillaga2021}, or \textit{(ii)} pre-processing data to clean it of positions with large location errors \citep{bjorneraas2010}.
    The first approach may be limited by the animal movement models that can be fitted to the data \citep{fleming2014a, noonan2019, fleming2020}, may result in unreasonable computation times, or may be entirely beyond the computational capacity of common hardware, leading users to prefer data cleaning instead.
    \graffito{
        Methods from \citet{fleming2014a} are technically very advanced, and that puts them beyond the reach of many animal ecologists.
    }
    Data cleaning reveals another challenge of high-throughput tracking: the large number of observations make it difficult for researchers to visually examine each animal's track for errors \citep{weiser2016, toledo2020}.
    With manual identification and removal of errors from individual tracks prohibitively time consuming, data cleaning can benefit from automation based on a protocol.

    Pre-processing of movement data --- defined as the set of data management steps executed prior to data analysis --- must reliably discard large location errors, also called outliers, from tracks (analogous to reducing false positives) while avoiding the overzealous rejection of valid animal movements (analogous to reducing false negatives).
    How well researchers balance these imperatives has consequences for downstream analyses \citep{stine2001}.
    For instance, small-scale resource selection functions can easily infer spurious preference and avoidance effects when there is uncertainty about an animal's true position \citep{visscher2006}.
    Ecologists recognise that tracking data are imperfect observations of the underlying movement process, yet they implicitly consider cleaned data equivalent to the ground-truth.
    This assumption is reflected in popular statistical methods in movement ecology such as Hidden Markov Models (HMMs) \citep{langrock2012}, stationary-phase identification methods \citep{patin2020a}, or step-selection functions (SSFs) \citep{barnett2008, signer2017, avgar2016}, which expect minimal location errors relative to real animal movement (i.e., a high signal-to-noise ratio).
    This makes the reproducible, standardised removal of location errors crucial to any animal tracking study.
    While gross errors are often removed by positioning-system algorithms in both GPS and reverse-GPS setups, `reasonable' errors often remain to confront end users \citep{fischler1981, weiser2016, ranacher2016}.
    Further, as high-throughput tracking is deployed in more regions and for more species, standardised pre-processing steps should be general enough to tackle animal movement data recovered from a range of environments, so as to enable sound comparisons across species and ecosystems.

    Despite the importance and ubiquity of reducing location errors in tracking data, movement ecologists lack formal guidance on this crucial step.
    Pre-processing protocols are not often reported in the literature, or may not be easily tractable for mainstream computing hardware and software.
    Some tracking data, such as GPS, are autonomously pre-processed without user access to the raw data \citep[using error estimates and Kalman smooths;][and substantial location errors may yet persist]{kaplan2005}.
    However, filtering out positions using estimates of location error alone may not be sufficient to exclude outliers which represent unrealistic movement but have low error measures \citep{weiser2016, ranacher2016}.
    When tracking systems do make their raw data available to researchers, this can enable users to better control the data pre-processing stage, and to substantially improve data quality while ensuring that cleaning does not itself lead to unrealistic movement tracks \citep[e.g. Kalman smooths which distort tracks,][]{kaplan2005}.
    This makes identifying and removing biologically implausible locations from a track an important component of recovering true animal movement \citep{bjorneraas2010}.

    Even after removing unrealistic movement, a track may be comprised of positions that are randomly distributed around the true animal location \citep{noonan2019}.
    The large data-volumes of high-throughput tracking allow for a neat solution: tracks can be `median smoothed' to reduce small location errors that have remained undetected \citep[e.g.][]{bijleveld2016}.
    Large data volumes may also need to be thinned, for example, examining environmental covariates as predictors of prolonged residence in an area  \citep[see e.g.][]{bracis2018, aarts2008, bijleveld2016, oudman2018, harel2016} might require thinning of high-resolution movement data to match the lower spatial resolution of environmental measurements. 
    Data thinning and clustering are also required to avoid non-independent observations due to strong spatio-temporal autocorrelation, or to examine the effect of sampling scale on movement metrics and resource-selection \citep{fleming2014a,noonan2019}.

    When dealing with datasets that contain many millions of positions, reseachers may run into computational limits when trying to apply pre-processing steps to their full dataset.
    For instance, the size of working memory (RAM) limits the size of datasets that can be loaded into \textit{R}, the programming and statistical language of choice in movement ecology \citep{r2020,joo2020,joo2020b}.
    Data-rich fields such as genomics inspire a possible solution: to break very large data into smaller subsets, and pass these subsets through automated computational `pipelines' \citep{schadt2010,peng2011}.
    Pre-processing pipelines for animal tracking data --- the set of steps that users apply to prepare the data for a specific analysis --- come with some additional concerns: \textit{(i)} identifying which pre-processing steps are necessary, and \textit{(ii)} ensuring that these steps reproducibly operate on the data as expected, and as efficiently as possible.

    While exploratory data analysis and visualisation can help determine how to pre-process the data to maximise the signal to noise ratio \citep{slingsby2016}, standardising implementations of pre-processing techniques into robust, version controlled software packages \citep[e.g. in \textit{R}, see]{wickham2015}, can increase the reliability and reproducibility of animal movement ecology \citep{haddaway2015,archmiller2020,powers2019,lewis2018}.
    Overcoming hard computational constraints on speed and memory usage for very large data will often require a combination of programming strategies, such as using tools optimised for tabular data, or parallelised processing.

    Here, we present guidelines for reproducibly pre-processing high-throughput animal tracking data (Fig.~\ref{preproc_fig_01}), with a focus on simple, widely generalisable steps that help improve data quality (Fig.~\ref{preproc_fig_02}).
    We take two important considerations into account, that \textit{(i)} the pre-processing steps should be easily understood and reproduced, and \textit{(ii)} our implementations must be computationally efficient and reliable.
    Consequently, formalising tools as functions in an \textit{R} package would improve portability and reproducibility \citep{marwick2018, wickham2015}.
    Using simulated movement tracks, we demonstrate simple yet robust implementations of the pre-processing steps we recommend, conveniently wrapped into the \textit{R} package \textit{atlastools} \citep{gupte2020a}, with a discussion of features that make these steps more reproducible, and more efficient.
    We also suggest one potential application of high-throughput tracking in studies of animal movement and space use, illustrated by the first-principles based synthesis of `residence patches' from clusters of spatio-temporally proximate positions \citep[\textit{sensu}][]{bijleveld2016, oudman2018, barraquand2008}.

    \graffito{
        The version of this manuscript published in JAE includes useful code snippets as a hands-on guide to using `atlastools'.
    }
    In two fully worked out examples using our package on real tracking data, we show how to apply basic spatio-temporal and data quality filters, how to filter out unrealistic movement, and how to reduce the effect of location error with a median smooth.
    In the first example, using calibration data from an ATLAS system, we show how the residence patch segmentation-clustering method can be used to accurately identify areas of prolonged residence under real field conditions.
    Finally, in our second example, we use ATLAS data from Egyptian fruit bats (\textit{Rousettus aegyptiacus}) tracked in the Hula Valley, Israel, to show a fully worked out example of the pre-processing pipeline and the residence patch method.
    While our approach to high-throughput tracking data, and our package of pre-processing functions was developed with reverse-GPS ATLAS systems in mind, both are broadly suitable to a wide range of high-throughput animal tracking data sources, from underwater acoustic reverse-GPS \citep{baktoft2019, baktoft2017, jung2015, aspillaga2021, aspillaga2021a}, high-resolution GPS \citep{strandburg-peshkin2015, papageorgiou2019, harel2016, klarevas-irby2021}, tracking radars \citep{horvitz2014}, and visual video tracking \citep{rathore2020, perez-escudero2014}.

    \section*{Best-Practices for Pre-Processing Workflows}

    \begin{figure}[ht!]
        \centering
        \includegraphics[width=0.9\textwidth]{figures/preprocessing/fig_01.png}
        \caption{
            \textbf{Some best-practices for pre-processing high-throughput tracking data.}
            Simple pre-processing of animal tracking data can improve the quality of animal tracking data, and the inferences that are drawn from it.
            % The organisation of pre-processing workflows into a `pipeline' --- a set of steps that users apply to prepare the data for a specific analysis --- can help make research more reproducible and reliable.
            Exploratory data analysis of representative subsets of the data can help to identify common issues with data quality, and to determine which pre-processing, steps such as filters and smooths, might be necessary (\textit{see also Fig.~\ref{preproc_fig_02}}).
            Pre-processing steps implemented as programming code can be made reproducible and shareable by following best-practices for software development: (i) tracking changes to the steps, and the software used, using version control (e.g. \textit{git}, \textit{renv}), (ii) preferring pre-existing tools, such as \textit{R} packages, which are well documented and tested, (iii) encapsulating custom-written code as functions, and bundling related functions into a package, and (iv) checking the quality of both custom-written code (e.g. by testing functions), and the overall pipeline (e.g. data visualisation).
            The efficiency of pre-processing steps can be increased by using strategies for dealing with large datasets, such as batch processing, or using a computing cluster.
            The use of existing tools optimised for large datasets, or by writing code in a `fast' language such as \textit{C++}, can also speed up the pre-processing of large datasets (see main text for examples).
            % See the \textit{Worked Out Example} on Egyptian fruit bats, as well as Supplementary Material 1, for more details on implementing pipelines.
            % Fig.~\ref{preproc_fig_02} shows an example of such a pipeline.
        }
        }
        \label{preproc_fig_01}
    \end{figure}

    Exploratory data analysis should be the first step towards pre-processing movement data \citep[see Fig.~\ref{preproc_fig_01};][]{slingsby2016}.
    Researchers with very large datasets of perhaps millions of rows should ideally select a representative subset of these data for exploratory data analysis, including individuals of different species, sexes, or seasonal cohorts.
    \graffito{
        Regardless of the level of automation, there is no replacement for human supervision at certain stages, which inherently makes it a subjective process. This both normal and desirable.
    }
    Examples of exploratory data analysis include plotting heatmaps of the number of observations per unit area across the study site (Fig.~\ref{preproc_fig_01}).
    Histograms of the location error estimates, plotting the linear approximations of animal paths between observations, and histograms of the sampling interval can help determine how data need to be treated so as to minimise location errors and improve computational tractability (Fig.~\ref{preproc_fig_01}).
    While pre-processing steps required for datasets will differ between studies and tracking technologies, we elaborate upon candidate steps and their parameterisation in following sections (see also Fig.~\ref{preproc_fig_02}).

    Following exploratory data analysis and the parameterisation of data cleaning steps, the specific implementation of these steps should be made reliable and reproducible.
    Since reproducing pre-processing steps can be challenging when using only written descriptions from published articles, providing the code to implement pre-processing steps reduces ambiguity and increases reproducibility \citep{haddaway2015}.
    For technically advanced users, the best-practices here are \textit{(i)} to implement pre-processing steps as `functions', \textit{(ii)} to collect related functions --- e.g. for similar kinds of data --- into a software `package', \textit{(iii)} to `test' that the functions handle input as expected, and \textit{(iv)} implement `version control' throughout, such that the process of development is documented \citep[Fig.~\ref{preproc_fig_01};][]{wickham2015,alston2020,perez-riverol2016}.

    As an example, our \textit{atlastools} package incorporates these best-practices, and may be used as a reference \citep[][]{gupte2020a}.
    We have written each pre-processing step as a separate function, and each of these functions is tested, usually on simulated data, but in some cases also on empirical data \citep[][see the directory \textit{tests/} in the associated Zenodo repository]{wickham2015}.
    Finally, logging error messages is crucial when passing data through a pipeline, helping determine which data subsets could not be handled as expected, and why.
    Users who would prefer to rely on pre-existing toolsets and methods can use \textit{R} packages that follow these best-practices, such as \textit{move} \citep{kranstauber2011}, and \textit{sftrack} \citep{boone2020}.
    \graffito{
        Every so often, movement ecologists develop `one package to rule them all'. `sftrack' seemed to be one such --- but then the developing lab wound up operations, possibly because of the pandemic --- and `sftrack' has not seen widespread uptake.
    }
    The large size of modern, high-throughput animal tracking data means that the computational challenge can often be \textit{the} main challenge in working with these data.
    For beginning users, organising their workflows so that they process subsets of the data (such as one individual) at a time can help overcome limitations on working memory.
    Animal tracking data stored in a relational database \citep[e.g. SQL databases][]{codd1970}, for example, can be broken into meaningful subsets based on individual identity and tracking season.
    These smaller subsets can then be loaded into working memory, pre-processed, and saved in a separate location (see Supplementary Material 1, Section 2 for a worked out example on an SQL database).
    Using existing tools optimised for tabular data, such as the \textit{R} package \textit{data.table} \citep{dowle2020}, can also speed up computation; \textit{atlastools} is built using \textit{data.table} for this reason.

    More advanced users seeking substantial speed gains might wish to look into parallel-processing, and process each subset of the data independently of the full dataset, for example by using a computing cluster \citep[see also][for an alternative]{zjdai2021}.
    Finally, another advanced method, used by popular packages such as \textit{move} \citep{kranstauber2011} and \textit{recurse} \citep{bracis2018}, is to write one's own methods in a `fast' low-level language, such as \textit{C++}, and link these to \textit{R} \citep[][]{eddelbuettel2013}; see also \textit{adehabitatLT}, which is written partially in \textit{C} \citep{calenge2006}.
    Beginning practitioners can organise their workflows around these packages to benefit from the features they incorporate.

    \section*{Pre-processing Steps, Usage, and Simulating Data}

    \subsection*{An Overview of Pre-processing Steps and \textit{atlastools}}

    In the sections that follow, we lay out pre-processing techniques for raw high-throughput tracking data, and demonstrate working examples of these techniques, which we have collected in the \textit{R} package \textit{atlastools} (see Fig.~\ref{preproc_fig_02}).
    Our package is aimed at getting `raw data' to the `analysis' stage identified by Joo et al. (2020) in their review of \textit{R} packages in movement ecology.
    The package is based on \textit{data.table}, a fast implementation of data frames; thus it is compatible with a number of data structures from popular packages including \textit{move}, \textit{sftrack}, and \textit{ltraj} objects, which can be converted to data frames \citep[][]{kranstauber2011,boone2020,calenge2009}.
    Our package functions are suitable for use with both regularly sampled data, as well as data with missing observations.

    We cover, first, the use of simple \textit{\textbf{Spatio-Temporal Filters}} to select positions within a certain time or area.
    Next, we show how users can \textit{\textbf{Reduce Location Errors}} by removing unreliable positions based on a system-specific error measure, or by the plausibility of associated movement metrics, such as speed and turning-angle \citep{seidel2018, calenge2009}.
    We then show how users can tackle small-scale location errors by applying a \textit{\textbf{Median Smooth}}, and users who need uniformly sampled data, can undertake \textit{\textbf{Data Thinning}} by either aggregation or subsampling.
    At this stage, the data are ready for a number of popular statistical treatments such as Hidden Markov Model-based classification \citep{michelot2016,langrock2012}.
    \graffito{
        Neither \citet{barraquand2008} nor \citet{bijleveld2016} had reproducible code --- I cannot really be sure how my implementation corresponds to theirs.
    }
    Finally, we show how users wishing simple, efficient segmentation-clustering of points where the animal showed prolonged residence, can classify their data into `residence patches' \citep{barraquand2008, bijleveld2016} based on the movement ecology of their study species, after filtering out travelling segments (see \textit{\textbf{System-Specific Pre-Processing Tools}}).

    These pre-processing techniques and package were designed with ATLAS systems in mind, motivated to meet the rapid growth of studies using this high-throughput system worldwide: in Israel \citep{toledo2014, toledo2016, toledo2020, corl2020, vilk2021}, the UK \citep{beardsworth2021a, beardsworth2021b}, and the Netherlands \citep[][Bijleveld et al. \textit{in prep.}]{beardsworth2021}. 
    However, the principles and functions presented here are ready for use with other massive high-resolution data collected by GPS \citep[e.g.][]{papageorgiou2019}, reverse-GPS \citep[e.g.][]{aspillaga2021} or any other high-throughput tracking system .
    Users may construct a pre-processing pipeline comprising of all the techniques we cover, or implement the modules most suitable for their data.
    Users are advised to visualise their data throughout their workflow, and especially to perform thorough exploratory data analysis, to check for evident location errors or other issues \citep{slingsby2016}.

    \begin{figure}[ht!]
        \centering
        \includegraphics[width=0.9\textwidth]{figures/preprocessing/fig_02.png}
        \caption{
            \textbf{An example of a modular pipeline for pre-processing high-throughput tracking data from raw localisations to cleaned data, and optionally into residence patches.}
            Users should apply the appropriate pre-processing modules and the steps therein until the data are suitable for their intended analysis, some of which are suggested here.
            The \textit{atlastools} function that may be used to implement each pre-processing step is shown in the grey boxes underneath each step.
            Popular statistical methods are shown underneath possible analyses (yellow boxes).
            Users are strongly encouraged to visualise their data and scan it for location errors as they work through the pipeline, always asking the question, could the animal plausibly move this way?
        }
        \label{preproc_fig_02}
    \end{figure}

    \subsection*{Simulating Data to Demonstrate Pre-Processing Steps}

    To demonstrate pre-processing steps, we simulated a realistic movement track of 5,000 positions using an unbiased correlated velocity model (UCVM) implemented via the \textit{R} package \textit{smoove} \citep[][see Fig.~\ref{preproc_fig_03}.a]{gurarie2017}.
    We added four kinds of error to the simulated track: (i) normally distributed small-scale offsets to the X and Y coordinates (small-scale error), (ii) normally distributed large-scale offsets to a random subset (0.5 \%) of the positions (spikes), (iii) large-scale displacement of a continuous sequence of 300 of the 5,000 positions (prolonged spikes; indices 500 -- 800), and (iv) we removed 10\% of the canonical track to simulate missing data (see Fig.~\ref{preproc_fig_03}.a).
    To demonstrate the residence patch method, we obtained data, in the form of 1,000 positions, from a mechanistic, individual-based simulation model, in which agents move using simple decision making rules, and can find high-productivity patches using only ephemeral cues, such as the density of prey-items and other competitors \citep{gupte2021a, netz2021}.
    \graffito{
        Using movement tracks from Chapter~\ref{ch:kleptomove} to demonstrate the residence patch method led to the idea of using them to investigate empirical methods generally; this then became Chapter~\ref{ch:patternprocess}.
    }
    The emergent, complex track structure is analogous to the foraging movements of animals, and provides a suitable challenge for the residence patch method and helps to demonstrate its generality.

    \section*{Spatio-Temporal Filtering}

    \subsection*{Spatial Filtering Using Bounding Boxes and Polygons}

    First, users should exclude positions outside the spatial bounds of a study area by comparing position coordinates with the range of acceptable coordinates (the bounding box), and removing positions outside them (Fig.~\ref{preproc_fig_03}.a). 
    A bounding box filter does not require a geospatial representation such as a shapefile, and can help remove unreliable data from a tracking system that is less accurate beyond a certain range \citep[][]{beardsworth2021}.
    In some special cases, users may wish to remove positions \textit{inside} a bounding box, either because movement behaviour within an area is not the focus of a study, or because positions recorded within an area are known to be erroneous.
    An example of the former is studies of transit behaviour between features which can be approximated by their bounding boxes. 
    Instances of the latter are likely to be system specific, but are known from ATLAS systems. 
    Bounding boxes are typically rectangular, and users seeking to filter for other geometries, such as a circular or irregularly-shaped study area, need a geometric intersection between their data and a spatial representation of the area of interest (e.g. shapefile, geopackage, or \textit{sf}-object in \textit{R}).
    The \textit{atlastools} function \textit{atl\_filter\_bounds} implements both bounding box and explicit spatial filters, and accepts X and Y coordinate ranges, an \textit{sf}-polygon or multi-polygon object \citep{pebesma2018}, or any combination of the three to filter the data.
    When both coordinate ranges and a polygon are provided, the data is first filtered by the ranges and then the polygon.
    The boolean function argument \textit{remove\_inside} determines whether positions inside the bounds are retained (\textit{FALSE}) or removed (\textit{TRUE}).

    % \begin{lst{\color{red} Listing}}[float,floatplacement=h!,language=R, style=customR, caption = {
    %     The \textit{atl\_filter\_bounds} function filters on an area defined by coordinate ranges, a polygon, or all three; it can remove positions outside (\textit{remove\_inside = FALSE}), or within the area (\textit{remove\_inside = TRUE}).
    %     The arguments \textit{x} and \textit{y} determine the X and Y coordinate columns, \textit{x\_range} and \textit{y\_range} are the filter bounds in a coordinate reference system in metres, and the data can be filtered by an \textit{sf-(MULTI)POLYGON}, which can be passed using the \textit{sf\_polygon} argument. 
    %     The output is a \textit{data.table}, which must be saved as an object (here, \textit{filtered\_data}).}]
    % filtered_data <- atl_filter_bounds(
    %                         data = data,
    %                         x = "X", y = "Y",
    %                         x_range = c(x_min, x_max),
    %                         y_range = c(y_min, y_max),
    %                         sf_polygon = your_polygon,
    %                         remove_inside = FALSE
    %                     )
    % \end{lst{\color{red} Listing}}

    \begin{figure}[ht!]
        \centering
        \includegraphics[width=0.9\textwidth]{figures/preprocessing/fig_03.png}
        \caption{
            \textbf{Simulated movement data showing four kinds of artificially added errors}.
            (i) Normally distributed small-scale error on each position, (ii) large-scale error added to 0.5\% of positions, (iii) 10\% of positions removed to simulate missing data, and (iv) 300 consecutive positions displaced to simulate a gross distortion affecting a continuous subset of the track.
            \textbf{(a)} Tracks can be quickly filtered by spatial bounds (dashed grey lines) to exclude broad regions (green = retained; grey = removed).
            \textbf{(b)} location error may affect single observations resulting in point outliers or `spikes' (red crosses and track segments), or continuous subsets of a track, called a `prolonged spike' (purple circles, top right), and both represent unrealistic movement.
            \textbf{(c)} Histograms of speed for the track (grey = small-scale errors, red = spikes), and the prolonged spike (purple) show that while spikes could be removed by filtering out positions with both high incoming and outgoing speeds and turning angles, prolonged spikes cannot be removed in this way, and should be resolved by conceptualising algorithms that find the bounds of the distortion instead.
            Users should frequently check the outputs of such algorithms to avoid rejecting valid data.
        }
        \label{preproc_fig_03}
    \end{figure}

    \subsection*{Temporal and Spatio-temporal Filters}

    Tracking data might fail to properly represent an animal's movement at certain times, for instance, data recorded before release, or data from shortly after release when the animal is still influenced by the stress of capture and handling.
    Periods of poor tracking quality may result from system malfunctions and unusual disturbances, and users may wish to exclude these data as well.
    Temporal filtering can exclude positions from intervals when data are expected to be unreliable for ecological inference, either due to abnormal movement behaviour or system-specific issues.  
    Temporal filters can be combined with spatial filters to select specific time-location combinations. 
    For example, studies of foraging behaviour of a nocturnal animal would typically exclude tracking data from the animal's daytime roosts (see \textit{Worked Out Example}).
    Users should apply filters in sequence rather than all at once, and visualise the output after each filtering step (`sanity checks'; see Supplementary Material Section 2).
    The atlastools function \textit{atl\_filter\_covariates} allows convenient filtering of a dataset by any number of logical statements, including querying data within a spatio-temporal range.
    % This function can be used to easily filter timestamps in a range, as well as combine simple spatial and temporal filters.
    % It accepts a character vector of \textit{R} expressions that each return a logical vector (i.e., \textit{TRUE} or \textit{FALSE}; {\color{red} Listing} 2).
    The function keeps only those data which satisfy each of the filter conditions, and users must ensure that the filtering variables exist in their dataset in order to avoid errors.

    % \begin{lst{\color{red} Listing}}[float, language=R, style=customR, caption = {
    %     Data can be filtered by a temporal or a spatio-temporal range using \textit{atl\_filter\_covariates}. 
    %     Filter conditions are passed to the \textit{filters} argument as a character vector. 
    %     Only rows in the data satisfying \textit{all} the conditions are retained. 
    %     Here, the first example shows how nighttime data can be retained using a predicate that determines whether the value of `hour' is between 6 and 18, and also within a range of X coordinates.
    %     The second example retains ATLAS locations calculated using $>$ 3 base stations (\textit{NBS}), with location error (\textit{SD}) $<$ 100, and data between an arbitrary day 5 and day 8.
    %     }]
    % night_data <- atl_filter_covariates(
    %                     data = dataset,
    %                     filters = c(
    %                         "!inrange(hour, 6, 18)",
    %                         "between(x, x_min, x_max)"
    %                     )
    %                 )

    % filtered_data <- atl_filter_covariates(
    %                         data = data,
    %                         filters = c(
    %                             "NBS > 3",
    %                             "SD < 100",
    %                             "between(day, 5, 8)"
    %                         )
    %                     )                            
    % \end{lst{\color{red} Listing}}

    \section*{Filtering to Reduce Location Errors}

    \subsection*{Filtering on Data Quality Attributes}

    Tracking data attributes can be good indicators of the reliability of positions calculated by a tracking system \citep{beardsworth2021}.
    \graffito{
        I wasn't fully convinced that ATLAS systems solve for X and Y coordinates independently. The assumption of independence underlies the formula for `SD' --- but \citet{weiser2016} does not make it perfectly clear whether it is correct.
    }
    GPS systems provide direct measures of location error during localisation \citep[Horizontal Dilution of Precision, HDOP in GPS]{ranacher2016}, while  in reverse-GPS systems, a measure referred to as Standard Deviation (SD in many datasets), can be calculated from the variance-covariance matrix of each position as: $\text{SD} = \sqrt{\text{Var X} + \text{Var Y} + \text{Cov XY}}$ \citep[see details in][]{maccurdy2009, maccurdy2019, weiser2016, ranacher2016}.
    Tracking data can also include indirect indicators of data quality.
    For instance, GPS systems' location error may be indicated indirectly by the number of satellites involved in the localisation.
    In reverse-GPS systems too, the number of base stations involved in each localisation is an indirect indicator of data quality, and positions localised using more receivers are usually more reliable \citep[the minimum required for an ATLAS localisation is 3; see][]{weiser2016, beardsworth2021}.
    % A location error measure associated with each coordinate pair (similar to GPS HDOP) can be calculated and assigned to a new column \textit{SD} using the formula for the sum of correlated random variables
    % \begin{linenomath*}
    %     \begin{equation*}
    %         SD = \sqrt{{VARX} + {VARY} + 2 \times {COVXY}}
    %      \end{equation*}
    % \end{linenomath*}
    Unreliable positions can be removed by filtering on direct or indirect measures of quality using \textit{atl\_filter\_covariates}.
    While filtering on direct quality attributes and unrealistic movement speeds (see below) will often be sufficient, filtering on indirect quality indicators is a strategy to consider when direct error measures are not available.

    \subsection*{Filtering Unrealistic Movement}

    Filtering on system-generated attributes may not remove all erroneous positions, and the remaining data may still include biologically implausible movement.
    Users are encouraged to visualise their tracks before and after filtering point locations, and especially to `join the dots' and connect consecutive positions with lines (Fig.~\ref{preproc_fig_03}.b).
    Whether the resulting track looks realistic is ultimately a subjective human judgement, but any decision to filter-out data must remain independent of the hypothesised movement behavior.
    This basic principle does not preclude explicitly integrating prior knowledge of the movement ecology of the study species to ask, `Does the animal move this way?'.
    \graffito{
        \citeauthor{bjorneraas2010}'s 2010 paper is surprisingly not as well known as it should be.
    }
    Segments which appear to represent unrealistic animal movement are often obvious to researchers with extensive experience of the study system \citep[the non-movement approach; see][]{bjorneraas2010}.
    Since it is both difficult and prohibitively time consuming to exactly reproduce expert judgement when dealing with large volumes of tracking data from multiple individuals, some automation is necessary.
    Users should first manually examine a representative subset of tracks and attempt to visually identify problems --- either with individual positions, or with subsets of the track --- that persist after filtering on system-generated attributes.
    Once such problems are identified, users can conceptualise algorithms that can be applied to their data to resolve them.

    A common example of a problem with individual positions is that of point outliers or `spikes' \citep{bjorneraas2010}, where a single position is displaced far from the track (see Fig.~\ref{preproc_fig_03}.b).
    Point outliers are characterised by artificially high speeds between the outlier and the positions before and after \citep[called incoming and outgoing speed, respectively;][]{bjorneraas2010}, lending a `spiky' appearance to the track.
    Removing spikes is simple: remove positions with extreme incoming and outgoing speeds.
    Users must first define plausible upper limits of the study species' speed \citep{calenge2009, seidel2018}.
    Here, it is important to remember that speed estimates are scale-dependent; high-throughput tracking typically overestimates the speed between positions where the animal is stationary or moving slowly, due to small-scale location errors \citep{ranacher2016, noonan2019}. 
    Even after data with large location errors have been removed, it is advisable to begin with a liberal (high) speed threshold that excludes only the most unlikely speeds.
    Estimates of maximum speed may not always be readily obtained for all species, and an alternative is to use a data-driven threshold such as the 90\textsuperscript{th} percentile of speeds from the track.
    Once a speed threshold $S$ has been chosen, positions with incoming \textit{and} outgoing speeds $> S$ may be identified as spikes and removed.

    Some species can realistically achieve speeds $> S$ in fast transit segments when assisted by their environment, such as birds with tailwinds, and a simple filter on incoming and outgoing speeds would exclude this valid data.
    To avoid removing valid, fast transit segments while still excluding spikes, the speed filter can be combined with a filter on the turning angles of each position \citep[see][]{bjorneraas2010, calenge2009}.
    This combined filter assumes that positions in high-throughput tracking with both high speeds and large turning angles are likely to be due to location errors, since most species are unable to turn sharply at very high speed.
    Users can then remove those positions whose incoming and outgoing speeds are both $> S$, and where $\theta > A$ (sharp, high-speed turns), where $\theta$ is the turning angle, and $A$ is the turning angle threshold.
    Many other track metrics may be used to identify implausible movement and to filter data \citep{seidel2018}.
    At this early stage in pre-processing, track metrics should be considered provisional --- it is not until after smoothing and potentially resampling to a regular interval (see below), that calculated track metrics should be used for ecological inference.
    % We show an implementation of spike removal using the \textit{atl\_filter\_covariates} function.

    % \begin{lst{\color{red} Listing}}[float, language=R, style=customR, caption = {
    %     Filtering a movement track on incoming and outgoing speeds, and on turning angle to remove unrealistic movement.
    %     The functions \textit{atl\_get\_speed} and \textit{atl\_turning\_angle} are used to get the speeds and turning angles before filtering, and assigned to a column in the data (assignment of \textit{speed\_out} is not shown).
    %     The filter step only retains positions with speeds below the speed threshold $S$ \textit{or} angles above the turning angle threshold $\theta$, i.e., positions where the animal is slow but makes sharp turns, and data where the animal moves quickly in a relatively straight line.}]
    % data$speed_in <- atl_get_speed(
    %                     data,
    %                     x = "x", y = "y",
    %                     time = "time", 
    %                     type = c("in")
    %                 )

    % data$angle <- atl_turning_angle(
    %                     data,
    %                     x = "x", y = "y", 
    %                     time = "time"
    %                 )

    % filtered_data <- atl_filter_covariates(
    %                     data = data,
    %                     filters = c(
    %                         "(speed_in < S & speed_out < S) | angle < A"
    %                     )
    %                 )
    % \end{lst{\color{red} Listing}}
    \graffito{
        This issue was especially evident in WATLAS tracking data from 2018. ATLAS positioning errors are an interesting combination of failures of the physical tracking device, as well as failures of the maximum likelihood localisation algorithm.
    }
    Sometimes, entire subsets of the track may be affected by the same large-scale location error.
    For instance, multiple consecutive positions may be roughly translated (geometrically) away from the real track and form `prolonged spikes', or `reflections' (see Fig.~\ref{preproc_fig_03}.b).
    These cannot be corrected by targeted removal of individual positions, as in Bjørneraas et al.'s approach (2010), since there are no positions with both high incoming and outgoing speeds, as well as sharp turning angles, that characterise spikes.
    Since filtering individual positions will not suffice, algorithms to correct such errors must take a track-level view, and target the displaced sequence overall.
    Track-subset algorithms are likely to be system-specific, and may be challenging to conceptualise or implement.
    In the case of prolonged spikes, one relatively simple solution is identifying the bounds of displaced segments, and removing positions between them.
    This identification can be based on relatively simple rules --- for example, the beginning of a prolonged spike could be identified as a position with a high \textit{incoming} speed, but a low \textit{outgoing} speed, while the end of such a spike would have a low incoming, but a high outgoing speed.
    We have implemented an illustrative example of such an algorithm in the form of track-subset filtering for prolonged spikes using the \textit{atlastools} function \textit{atl\_remove\_reflections} (see the \textit{atlastools} documentation for details on the algorithm).
    Users are strongly encouraged to visualise their data before and after applying such algorithms; as these methods are not foolproof, and data that are heavily distorted by errors affecting entire track-subsets should be used with care when making further inferences.

    % % \begin{lst{\color{red} Listing}}[float, language=R, style=customR, caption = {
    % %     Removing unrealistic movement in the form of prolonged spikes from a movement track. 
    % %     The important function arguments here are \textit{point\_angle\_cutoff} ($A$), \textit{reflection\_speed\_cutoff} ($S$), and \textit{est\_ref\_len}, the maximum number of positions after the inner bound that are candidates for the end of the prolonged spike, i.e., the outer bound. 
    % %     If the prolonged spike ends after less than $N$ positions, the true end point is used as the outer bound of the spike.
    % %     However, the algorithm behind this function fails when the prolonged spike ends after more than $N$ positions. 
    % %     Users are advised to use a liberally large value of N in the \textit{est\_ref\_len} argument; 1,000 may be appropriate for 3s interval data.
    % %     Further, users are cautioned against relying on such algorithms for severely distorted data.}]
    % % filtered_data <- atl_remove_reflections(data = track_data,
    % %                        x = "x", y = "y", time = "time",
    % %                        point_angle_cutoff = A,
    % %                        reflection_speed_cutoff = S,
    % %                        est_ref_len = N)
    % % \end{lst{\color{red} Listing}}

    \section*{Smoothing and Thinning Data}

    \subsection*{Median Smoothing}

    After filtering out large location errors, the track may still look `spiky' at small scales, and this is due to smaller location errors that are especially noticeable when the individual is stationary or moving slowly \citep{noonan2019}.
    These smaller errors are challenging to remove since their attributes (such as speed and turning angles) are within the expected range of movement behaviour for the study species. 
    The large data volumes of high-throughput tracking allow users to resolve this problem by smoothing the positions. 
    The most basic `smooths' work by approximating the value of an observation based on neighbouring values.
    For a one-dimensional series of observations, the neighbouring values are the $K$ observations centred on each index value $i$.
    The range ${i - (K-1)/2} \ldots {i + (K-1)/2}$ is referred to as the moving window as it shifts with $i$, and $K$ is the moving window size.
    A common smooth is nearest neighbour averaging, in which the value of an observation $x_i$ is the average of the moving window $K$.
    The median smooth is a variant of nearest neighbour averaging which uses the median rather than the mean, and is more robust to outliers (\citeauthor{tukey1977} 1977).
    The median smoothed value of the X coordinate, for instance, is
    %
    % \begin{linenomath*}
        % \begin{equation*}
        $$
            X_i = \text{Median}(X_{i - (K-1)/2} \ldots X_{i + (K-1)/2}).
        $$
            % \end{equation*}
    % \end{linenomath*}
    \graffito{
        Median smoothing is a classic method \citep{tukey1977}, that has found applications across fields, including for example, recovering corrupted digital images.
    }
    Users can apply a median smooth with an appropriate $K$ independently to the X and Y coordinates of a movement track to smooth it (see Fig.~\ref{preproc_fig_04}.a -- e).
    The median smooth is robust to even very large temporal and spatial gaps, and does not interpolate between positions when data are missing. 
    Thus it is not necessary to split the data into segments separated by periods of missing observations when applying the filter (see Fig.~\ref{preproc_fig_04}).

    Some data sources, such as GPS, provide tracks that have already been smoothed in quite sophisticated ways, such as with a Kalman filter, making a median smooth unnecessary \citep{kaplan2005}.
    Furthermore, smoothing is not a panacea for data quality issues, and has its drawbacks.
    Smoothing does not change the number of observations, but does decouple the coordinates from some of their attributes.
    For instance, smoothing breaks the relationship between a coordinate and the location error estimate around it (VARX, VARY, and SD in ATLAS systems).
    Since the X and Y coordinates are smoothed independently, the smoothed coordinates of an observation will likely differ from all the coordinates used to compute the smoothed value.
    Any position covariates (e.g. environmental values such as landcover or elevation) obtained before smoothing should be replaced with the covariates obtained at the smoothed coordinates.
    Similarly, instantaneous track metrics, such as speed and turning angle, should also be updated at this stage to reflect the smoothed coordinates.
    Furthermore, the location error estimate around each coordinate, and around the localisation overall, become invalid and should be ignored.
    This makes subsequent filtering on measures of data quality unreliable, and smoothed data are unsuitable for use with methods that model location uncertainty \citep{noonan2019, fleming2014a, fleming2020, calabrese2016}.
    Thus, when applying location error modelling methods, users should ensure that the error measure bears a mechanistic relationship with the location estimate \citep[see][ for more details]{fleming2020, noonan2019}.
    Additionally, excessively large $K$ may result in a loss in detail of the individual's small-scale movement (compare Fig.~\ref{preproc_fig_04}.e with ~\ref{preproc_fig_04}.a).
    Users must themselves judge how best to balance large-scale and small-scale accuracy, and choose $K$ accordingly.
    Median smoothing is provided by the \textit{atlastools} function \textit{atl\_median\_smooth}, with the only option being the moving window size, which must be an odd integer.

    % \begin{lst{\color{red} Listing}}[float, language=R, style=customR, caption = {
    %     Median smoothing a movement track using the function \textit{atl\_median\_smooth} function with a moving window \textit{K = 5}. 
    %     Larger values of $K$ yield smoother tracks, but $K$ should always be some orders of magnitude lower than the number of observations.}]
    % atl_median_smooth(
    %     data = track_data,
    %     x = "x", y = "y",
    %     time = "time",
    %     moving_window = 5
    % )
    % \end{lst{\color{red} Listing}}

    \begin{figure}[ht!]
        \centering
        \includegraphics[width=0.9\textwidth]{figures/preprocessing/fig_04.png}
        \caption{
            \textbf{Median smoothing position coordinates reduces small-scale location error in tracking data.}
            The goal of this step is to approximate the simulated canonical track (black line, \textbf{(a)}), given positions with small-scale error that remains after filtering in previous steps (green points).
            \textbf{(b)} Median smoothing the position coordinates (green points, in \textbf{(a)}) over a moving window ($K$) of 21 positions gives a good approximation (blue line) of the canonical track, and is a significant improvement on the unsmoothed track (grey lines and points).
            While $K$ should usually be at least two orders of magnitude less than the number of positions in the track, users are cautioned that there is no correct $K$, and they must subjectively choose a $K$ which most usefully trades small-scale details of the track for large-scale accuracy.
            Here, smoothing with a $K$ of \textbf{(c)} 5 (dark grey line) and \textbf{(d)} 11 (blue line), leads to a jagged track, compared to the true path in (a), and the distance moved by the animal would be overestimated.
            \textbf{(e)} Using extremely large values of $K$ (101) may lead to a loss of both large and small scale detail (red line).
            Across panels, grey lines and points show the track without smoothing.
        }
        \label{preproc_fig_04}
    \end{figure}

    \subsection*{Thinning Movement Tracks}

    Most data at this stage are technically ‘clean', yet the volume alone may pose challenges for lower-specification or older hardware and software if these are not optimised for efficient computation.
    Thinning data i.e., reducing their volume, need not compromise researchers' ability to answer ecological questions; for instance, proximity-based social interactions lasting 1 -- 2 minutes would still be detected on thinning from a sampling interval of 1 second to 1 minute \citep[][]{aspillaga2021a}.
    Thinning data also does not imply that efforts to collect high-throughput movement data are ‘wasted', as rich movement datasets enable more detailed and more accurate representation of the true track, as elaborated above. 
    Indeed, some analyses require that temporal auto-correlation in the data be broken by subsampling the data to a lower resolution; these include traditional kernel density estimators for animal home-range, as well as resource selection functions \citep{fleming2014a,manly2007,dupke2017}.
    Furthermore, a number of powerful methods in movement ecology, including Hidden Markov Models and integrated Step-Selection Analysis recommend uniform sampling intervals \citep{avgar2016,langrock2012,michelot2016}.
    Finally, subsampling data may be an important strategy in exploratory data analysis; for instance, it allows researchers to determine whether computationally intensive methods, such as distance and speed estimates from continuous time movement model fitting, are required for their data, or whether the movement metrics stabilise at a certain time scale \citep[][]{noonan2019}.
    Two plausible approaches here are subsampling and aggregation, and both approaches begin with identifying time-interval groups (e.g. of 1 minute).
    Subsampling picks one position from each time-interval group while aggregation involves computing the mean or median of all system-generated attributes for positions within a time-interval group.
    Here again, users should repeat the extraction of any environmental covariates for the thinned data, and may wish to obtain the mean values in a radius aroung the locations, rather than point estimates alone.
    Both approaches yield one position per time-interval group (Fig.~\ref{preproc_fig_05}.a).
    Categorical variables, such as the habitat type associated with each position, can be aggregated using a suitable measure such as the mode.
    We caution users that thinning causes an extensive loss of small-scale detail in the data, and should be used carefully.

    Both aggregation and subsampling have their relative advantages. 
    The aggregation method is less sensitive to selecting point outliers by chance than subsampling.
    However, to account for location error with methods such as state-space models \citep{jonsen2003, jonsen2005, johnson2008} or continuous time movement models \citep{fleming2014a, noonan2019, gurarie2017, calabrese2016, fleming2020}, correctly propagating the location error is important, and subsampling directly propagates these errors without further processing.
    In reverse-GPS systems systems the location error is calculated from the variance-covariance matrix of the coordinates of candidate positions considered by the location solver \citep{weiser2016}; this is equivalent to GPS systems' HDOP \citep{ranacher2016}.
    In the aggregation method, the location error around each coordinate provided by either GPS or reverse-GPS systems can be propagated --- assuming the errors are normally distributed --- to the averaged position as the sum of errors divided by the square of the number of observations contributing to each average ($N$):
    % \begin{linenomath*}
        $$
            \text{Var}(X)_\text{agg} = \left( \sum_{i=1}^{i=N} \text{Var}(X)_i \right) / N ^ 2
        $$
    % \end{linenomath*}
    Similarly, the overall location error estimate for the average of $N$ positions in a time-interval can be calculated by treating it as a variance. For instance, the ATLAS error and GPS error measures (SD and HDOP, respectively) can be aggregated as:
    % \begin{linenomath*}
        $$
            SD_\text{agg} \ or \ HDOP_\text{agg} = \sqrt{ \left( \sum_{i=1}^{i=N} SD_i^2 \ or \ HDOP_i^2 \right) / N ^ 2  }
        $$
    % \end{linenomath*}

    Users may question why thinning, which can obtain consensus positions over an interval and also reduce data-volumes should not be used directly on the raw data.
    %%
    We caution that thinning prior to excluding unrealistic movement and smoothing (Fig 5.b) can lead to preserving artefacts in the data, and estimates of essential metrics --- such as straight-line displacement (and hence, speed) --- that are substantially different from the true value \citep[see Fig.~\ref{preproc_fig_05}.c;][]{noonan2019}.
    In our example, the data with errors would have to be thinned to $\frac{1}{30}$\textsuperscript{th} of its volume for the median speed of the thinned data to be comparable with the overall median speed --- this is an undesirable step if the aim is fine-scale tracking.
    Additionally, the optimal level of thinning can be difficult to determine, especially if there is wide individual variation in movement behaviour, and the mis-estimation of track metrics from inappropriately thinned data could have consequences for the implementation of subsequent filters based on detecting unrealistic movement.
    However, thinning before data-cleaning has its place as a useful step before exploratory visualisation of the movement track, since reduced data-volumes are easier to handle for plotting software.
    %
    Thinning is implemented in \textit{atlastools} using the \textit{atl\_thin\_data} function, with either aggregation or subsampling (specified by the \textit{method} argument) over an interval using the \textit{interval} argument.
    Grouping variable names (such as animal identity) may be passed as a character vector to the \textit{id\_columns} argument.

    % \begin{lst{\color{red} Listing}}[float, language=R, style=customR, caption = {Code to thin data by aggregation in \textit{atlastools}. The method can be either "aggregate" or "subsample". 
    % The time interval is specified in seconds, while the \textit{id\_columns} allows a character vector of column names to be passed to the function, with these columns used as identity variables.
    % Both methods return a dataset with one rows per time-interval.}]
    % thinned_data <- atl_thin_data(
    %                     data,
    %                     interval = 60,
    %                     id_columns = c("animal_id"),
    %                     method = "aggregate"
    %                 )
    % \end{lst{\color{red} Listing}}

    % % figure for aggregation thinning
    \begin{figure}[ht!]
        \centering
        \includegraphics[width=0.9\textwidth]{figures/preprocessing/fig_05.png}
        \caption{
            \textbf{Thinning tracking data can aid computation but must be approached carefully}.
            Aggregating a filtered and smoothed movement track \textbf{(a)} preserves track structure while reducing data-volume, but \textbf{(b)} aggregating before filtering gross location errors and unrealistic movement leads to the persistence of large-scale errors (such as prolonged spikes).
            \textbf{(c)} Thinning before data cleaning can lead to significant misestimations of essential movement metrics such as speed at lower intervals.
            Boxplots show the median and interquartile ranges for speed estimates of tracks aggregated over intervals of 3, 10, 30, and 120 seconds.
            For comparison, the median and 95\textsuperscript{th} percentile of speed of the canonical track are shown as solid and dashed horizontal lines, respectively.
            % The unfiltered data would have to be thinned to \sfrac{1}{30}\textsuperscript{th} volume to correctly estimate median speed.
        }
        \label{preproc_fig_05}
    \end{figure}

    \section*{System-Specific Pre-processing Tools}

    When researchers' pre-processing requirements exceed the functionalities of existing tools, they might have to conceptualise and implement their own methods.
    For instance, an important and common analysis with animal tracking data is to link space use with environmental covariates.
    This is difficult even with smoothed and thinned high-throughput data, as these may be too large for statistical packages, or have strong autocorrelation.
    Users aiming for such analyses can benefit from segmenting and clustering the data into spatio-temporally independent bouts of different behavioural modes \citep{patin2020a}.
    Treating these as the unit of observation also conveniently sidesteps pseudo-replication and reduces computational requirements.
    While numerous methods of segmenting and clustering data are in use, they may not be scalable to very large or gappy datasets \citep{patin2020a, langrock2012, michelot2016}.
    \graffito{
        I actually came up with the residence patch method because Lavielle segmentation \citep{lavielle} and `segclust2d' \citep{patin2020a} just did not seem to be working on WATLAS data.
    }
    As an alternative, a first-principles approach that segments data based on the movement capacity (top speed, etc.) of tracked animals, could provide a fast, yet useful way to cluster data.
    Here, as a working example that may be suitable for some systems, we present a simple segmentation-clustering algorithm to make `residence patches', identified as bouts of relatively stationary behaviour \citep[][]{barraquand2008,bijleveld2016,oudman2018}.
    Details of the implementation may be found in the package code, and examples are provided in the Supplementary Material.

    \subsection*{Conceptualising a Simple Segmentation-Clustering Algorithm: The Residence-Patch Example}

    Before implementing the algorithm, users should identify positions where the animal is relatively stationary, for instance on its speed or first-passage time \citep{bracis2018,barraquand2008}.
    Our suggested algorithm begins by assessing whether consecutive stationary positions are spatio-temporally independent, and clusters them together into a residence patch if they are not.
    This clustering could be based on a simple proximity threshold --- points farther apart than some threshold distance are likely to represent two different residence patches.
    In cases where animals visit multiple sites in sequence \citep[such as traplining:][]{thomson1997}, and which researchers might wish to consider as a single residence patch, a larger-scale distance threshold can help cluster nearby residence patches together, and this can also be applied to cluster together patches artificially separated due to missing data.
    Our algorithm separates two observations at a similar location, but at two very different time points, by comparing the intervening time-lag against a time-difference threshold, which can also apply to patches that would otherwise be clustered by the large-scale distance threshold.
    Users are encouraged to base these thresholds on the movement habits of their study species (see the \textit{Worked Out Example}).

    We have implemented a working example of the simple clustering concept presented here as the function \textit{atl\_res\_patch} (see Fig.~\ref{preproc_fig_06}.b), which requires three parameters: (i) the distance threshold between positions (called \textit{buffer\_size}), (ii) the large-scale distance threshold between clusters of positions (called \textit{lim\_spat\_indep}), and (iii) the time-difference threshold between clusters (called \textit{lim\_time\_indep}).
    Clusters formed of fewer than a minimum number of positions can be excluded.
    Our algorithm performs well when movement modes are clearly separated, and is capable of correctly separating positions that are close together in space and time, but which comprise different behavioural sequences (see Fig.~\ref{preproc_fig_06}).
    While the algorithm may not cover all possible use-cases and study species, we provide it here as an example of a user-built exploratory method for animal tracking data.
    It is important to systematically test such custom-made algorithms, to ensure reproducibility and reliability \citep{wickham2015, marwick2018}.
    Simple examples of such tests for the residence patch method and other functions in \textit{atlastools} may be found in the \textit{tests/} directory in the associated Github repository.

    % \begin{lst{\color{red} Listing}}[float, language=R, style=customR, caption ={The \textit{atl\_res\_patch} function can be used to classify a track into residence patches. The arguments \textit{buffer\_radius} and \textit{lim\_spat\_indep} are specified in metres, while the \textit{lim\_time\_indep} is provided in minutes. In this example, specifying \textit{summary\_variables = c("speed")}, and \textit{summary\_functions = c("mean", "sd")} will provide the mean and standard deviation of instantaneous speed in each residence patch. The \textit{atl\_patch\_summary} function is used to access the classified patch in one of three ways, here using the \textit{summary} option which returns a table of patch-wise summary statistics.}]
    % patches <- atl_res_patch(
    %                 data = track_data,
    %                 buffer_radius = 10,
    %                 lim_spat_indep = 100,
    %                 lim_time_indep = 30,
    %                 min_fixes = 3,
    %                 summary_variables = c("speed"),
    %                 summary_functions = c("mean", "sd")
    %             )
    % \end{lst{\color{red} Listing}}

    % % patch_summary <- atl_patch_summary(
    % %                     patch_data = patches,
    % %                     which_data = "summary",
    % %                     buffer_radius = 10
    % %                 )

    % % first residence patch figure
    \begin{figure}[ht!]
        \centering
        \includegraphics[width=0.9\textwidth]{figures/preprocessing/fig_06.png}
        \caption{
            \textbf{Movement tracks can be classified into residence patches, while leaving out the transit between them.}
            \textbf{(a)} A simulated animal movement track from \citealt{gupte2021a}, where an agent uses local cues to make movement decisions to maximise intake.
            The agent tends to stop (solid circles) on high-productivity areas of the landscape, as these are more likely to generate prey-items.
            Transit points between stationary phases are shown as crosses.
            \textbf{(b)} Our simple, first-principles based clustering algorithm classifies the track into five residence patches. 
            Some transit points are erroneously classified as being part of a residence patch (top, yellow), illustrating why is it important to remove such data before applying this method.
            Simultaneously, some points where the animal is not stationary for long are not picked up by the method.
            While the large purple patch (bottom) is composed almost entirely of consecutive positions, the subsequent patches are composed of multiple parts.
            This is because our method was designed to be robust to missing data from empirical tracks; the spatial and temporal limits of splitting and lumping can be controlled using the arguments passed to \textit{atl\_res\_patch}, and can be adjusted to fit the study system.
            Users are cautioned that there are no `correct' options, and the best guide is the behavioural biology of the tracked individual.
        }
        \label{preproc_fig_06}
    \end{figure}

    \subsection*{A Real-World Test of User-Built Pre-Processing Tools}

    We applied the pre-processing pipeline using \textit{atlastools} functions described above to an ATLAS dataset to verify that the residence patch method could correctly identify known stopping points (see Fig.~\ref{preproc_fig_07}).
    We collected the data (n = 50,816) on foot and by boat, with a hand-held WATLAS tag (sampling interval = 1s) around the island of Griend (53.25$^{\circ}$N, 5.25$^{\circ}$E) in August 2020 \citep[WATLAS: Wadden Sea ATLAS system][Bijleveld et al. \textit{in prep.}]{beardsworth2021}.
    Since the data were intended to test the accuracy of the WATLAS system, we were able to log stops in the track as waypoints using a handheld GPS device, and manually annotate the WATLAS data with the timestamp of each waypoint (Garmin Dakota 10; see \citealt{beardsworth2021}).
    We estimated the real duration of each stop as the time difference between the first and last position recorded within 50m of each waypoint, within a 10 minute window before and after the waypoint timestamp (to avoid biased durations from revisits).
    Stops had a median duration of 10.28 minutes (range: 1.75 minutes -- 20 minutes; see Supplementary Material).
    We cleaned the data before constructing residence patches by (i) removing a single outlier ($>$ 15 km away), removing unrealistic movement ($\geq$ 15 m/s), smoothing the data ($K$ = 5), and (iv) thinning the data by subsampling over a 30 second interval.
    The cleaning steps retained 37,324 positions (74.45\%), while thinning reduced these to 1,803 positions (4.8\% positions of the smoothed track).
    Details and code are provided in the Supplementary Material (see \textit{Validating the Residence Patch Method with Calibration Data}).

    We began by visualising the data to check for location errors, and found a single outlier position approx. 15km away from the study area (Fig.~\ref{preproc_fig_07}.a).
    This outlier was removed by filtering data by the X coordinate bounds using the function \textit{atl\_filter\_bounds}; X coordinate bounds $\leq$ 645,000 in the UTM 31N coordinate reference system were removed (n = 1; remaining positions = 50,815).
    We then calculated the incoming and outgoing speed, as well as the turning angle at each position using the functions \textit{atl\_get\_speed} and \textit{atl\_turning\_angle} respectively, as a precursor to targeting large-scale location errors in the form of point outliers.
    We used the function \textit{atl\_filter\_covariates} to remove positions with incoming and outgoing speeds $\geq$ the speed threshold of 15 m/s (n = 13,491, 26.5\%; remaining positions = 37,324, 73.5\%; Fig.~\ref{preproc_fig_07}.b).
    This speed threshold was chosen as 5 m/s faster than the known boat speed, 10 m/s.
    Finally, we targeted small-scale location errors by applying a median smooth with a moving window size $K$ = 5 using the function \textit{atl\_median\_smooth} (Fig.~\ref{preproc_fig_07}.c).
    This step does not reduce the number of positions.

    \graffito{
        The residence patch method obviously works best when there are clear speed differences between stationary and travelling positions.
    }
    We identified stationary positions as those where the median smoothed speed ($K$ = 5) was $<$ 2m/s, as people or a boat moving any faster are likely to be in transit.
    We clustered these positions into residence patches with a buffer radius of 5m, spatial independence limit of 50m, temporal independence limit of 5 minutes, and a minimum of 3 positions per patch.
    Inferred residence patches corresponded well to the locations of stops (see Fig.~\ref{preproc_fig_07}.c).
    However, the residence patch algorithm detected seven more stops (n = 28) than there were waypoints (n waypoints = 21).
    One of these was the field station on Griend where the tag was stored between trips (red triangle, Fig.~\ref{preproc_fig_07}.c), while another patch was formed of positions recorded while waiting for the boat; such unintended stops, not recorded as waypoints, likely accounted for the remaining five `extra' residence patches.
    Our analysis also did not detect two stops of 105 and 563 seconds (1.75 and 9.4 minutes) since they were data poor and were cleaned away during pre-processing (n positions = 6, 15), highlighting that the quality of the raw data (as in the rest of the track) is still a limiting factor on the inferences that are possible after pre-processing.
    To determine whether the residence patch method correctly identified the duration of detected stops in the calibration track, we first extracted the patch attributes using the function \textit{atl\_patch\_summary}.
    We then matched the patches to the waypoints by their median coordinates (rounded to 100 metres).
    We assigned the inferred duration of the stop as the duration of the spatially matched residence patch.
    We compared the inferred duration with the real duration using a linear model with the inferred duration as the only predictor of the real duration.
    % We excluded a single waypoint (WP080) since we had accidentally stopped there for $>$ 60 minutes.
    Inferred duration was a good predictor of the real duration of a stop (linear model estimate = 1.021, t-value = 12.965, $p <$ 0.0001, $R^2$ = 0.908; see Supplementary Material Fig.~\ref{preproc_fig_01}.7).
    This translates to a 2\% underestimation of the stop duration at a tracking interval of 30 seconds.
    Finally, any classification algorithm will present users with a trade-off between over-sensitivity (erroneously finding stops where there were none), and under-sensitivity (missing stops where they are not local or long enough) --- users should balance between these based on the broader questions sought to be answered.

    % % calibration figure
    \begin{figure}[ht!]
        \centering
        \includegraphics[width=0.9\textwidth]{figures/preprocessing/fig_07.png}
        \caption{
            \textbf{Pre-processing steps for WATLAS calibration data showing filtering on speed, median smoothing and thinning by aggregation, and making residence patches.}
            \textbf{(a)} Positions with incoming and outgoing speed $>$ 15 m/s are removed (grey crosses = removed, green points = retained).
            \textbf{(b)} Raw data (grey crosses), median smoothed positions (green circles; moving window $K$ = 5), and the smoothed track thinned by aggregation to a 30 second interval (purple squares).
            Square size corresponds to the number of positions used to calculate the averaged position during thinning.
            \textbf{(c)} Clustering thinned data into residence patches (green polygons) yields robust estimates of the location of known stops (purple triangles).
            The algorithm identified all areas with prolonged residence, including those which we had not intended to be recorded, such as stops at the field station (n = 12; red triangle).
            Our analysis could not find two stops of 105 and 563 seconds duration (6 and 15 fixes, respectively), since these were lost in the data thinning step; one of these is shown here (purple triangle without green polygon).
        }
        \label{preproc_fig_07}
    \end{figure}

    \section*{A Worked-Out Example on Animal Tracking Data}

    We present a fully worked-out example of our pre-processing pipeline and residence patch method using movement data from three Egyptian fruit bats (\textit{Rousettus aegyptiacus}) tracked using the ATLAS system in the Hula Valley, Israel (33.1$^{\circ}$N, 35.6$^{\circ}$E) (\citealt{toledo2020, lourie2021}).
    Code and data can be found in the Supplementary Material and Zenodo repository (see \textit{Processing Egyptian Fruit Bat Tracks}). 
    Data selected for this example were collected over three nights (5\textsuperscript{th} -- 7\textsuperscript{th} May, 2018), with an average of 13,370 positions (SD = 2,173; range = 11,195 -- 15,542; interval = 8 seconds) per individual.
    Plotting the tracks revealed potential location errors (see Fig.~\ref{preproc_fig_01}, see also Supplementary Material Fig.2.1), which we filtered out by removing observations with ATLAS SD $>$ 20 (see Supplementary Material Section 2.5), as well as removing observations calculated using fewer than four base stations, altogether trimming 22\% of the raw data (mean positions remaining = 10,447 per individual).
    % We converted the ATLAS time format (milliseconds since the UNIX Epoch) to the more common UNIX format (seconds since the Epoch) by taking the floored value of the time divided by 1000.
    Then, we removed unrealistic movement represented by positions with incoming and outgoing speeds $>$ 20 m/s that exceed the maximum flight speed recorded in this species (15 m/s; \citealt{tsoar2011}), leaving 10,337 positions per individual on average (98\% of previous step).
    We median smoothed the data with a moving window $K$ size = 5, and no observations were lost.

    We aimed to study bats' night-time foraging on fruit trees by quantifying the duration of bats' residence patches.
    We began the construction of residence patches by finding the residence time within 50 metres of each position; this is the maximal radius of a `cloud of points' around fruit trees \citep{bracis2018}.
    Foraging bats repeatedly traverse the same routes (\citealt{toledo2020, tsoar2011, lourie2021}) and this could artificially inflate the residence time of positions along these routes.
    To avoid confusing revisits with residence, we limited the summation of residence times at each position to the period until the first departure of 5 minutes or more.
    Thus, two nearby locations ($\leq$ 50m apart) each visited for one minute at a time, but separated by an interval of some hours would not be clustered together as a residence patch. 
    To focus on bats' night-time foraging behaviour, we also excluded positions during the day (5 AM -- 8 PM), and at or near the roost-cave (see Fig.~\ref{preproc_fig_08}a) to focus on night-time foraging behaviour; 22,910 of 31,012 positions remained (73.9\%).
    % Since bats departed and returned to their roost at different times each night, we also excluded locations with a residence time $>$ 200 minutes (approx. 3.3 hours), as this was more likely to represent daytime roosting than nighttime foraging; of 31,012 smoothed positions, 18,677 remained (60\%).
    % From these positions, we calculated that between leaving the roost to forage, and returning, bats had a mean residence time at each position of 95.64 minutes (SD = 119.23) --- this value is still likely to be biased by some positions at the roost.
    
    \graffito{
        These and similar data are further analysed in \citet{lourie2021}.
    }
    To determine the true duration of foraging, we opted for a first-principles approach and first selected only locations with a residence time $>$ 5 minutes, reasoning that a flying animal stopping for $>$ 5 minutes at a location should plausibly indicate resource use or another interesting localised behaviour.
    This step retained 5,736 positions per bat on average (17,208 total), or 72.4\% of the nighttime positions.
    We then constructed residence patches with a buffer distance of 25m, a spatial independence limit of 100m, a temporal independence limit of 30 minutes, and rejected patches with fewer than three positions.
    These values are meant as examples; users should determine the sensitivity of their results to parameter choices.
    Bats spent 56.95 minutes at foraging sites (SD = 62.20), and were stationary in particular fruit trees and roosting trees during 83.8\% of their foraging time (Fig.~\ref{preproc_fig_08}).
    Although all three bats roosted at the same cave during the day, and all their tracks are within the typical foraging area of bats roosting in this cave \citep{lourie2021}, they used distinct foraging sites across the area at night (Fig.~\ref{preproc_fig_08}.a). The lack of overlap among individuals in tree use, obtained with the residence patch algorithm, shows that although co-roosting bats share the same cave-specific foraging area \citep{lourie2021}, they often forage on different trees.
    Contrasting the actual movement path with the linear path between residence patches can help reveal details of how animal cognition affects space use \citep{toledo2020}.
    Bats tended to show prolonged residence near known food sources (fruit trees), but also where no fruit trees were recorded (Fig.~\ref{preproc_fig_08}.b, ~\ref{preproc_fig_08}.c), in line with previous evidence for their use of non-fruiting trees to rest, to handle and digest food, and presumably for social interactions \citep{tsoar2011}.

    \begin{figure}[ht!]
        \centering
        \includegraphics[width=0.9\textwidth]{figures/preprocessing/fig_08.png}
        \caption{
            \textbf{Synthesising animal tracks into residence patches can reveal movement in relation to landscape features, prior exploration, and other individuals.}
            \textbf{(a)} Linear approximations of the paths (coloured straight lines) between residence patches (circles) of three Egyptian fruit bats (\textit{Rousettus aegyptiacus}), tracked over three nights in the Hula Valley, Israel.
            Real bat tracks are are shown as thin lines below the linear approximations, and colours show bat identity. The grey hexagon represents the roost-cave at Gar Hershom.
            Black points represent known fruit trees.
            Background is shaded by elevation at 30 metre resolution.
            \textbf{(b)} Spatial representations of an individual bat's residence patches (green polygons) can be used to study site-fidelity by examining overlaps between patches, or to study resource selection by inspecting overlaps with known resources such as fruit trees (black circles).
            In addition, the linear approximation of movement between patches (straight green lines) can be contrasted with the estimated real path between patches (irregular green lines), for instance, to determine the efficiency of movement between residence patches.
            \textbf{(c)} Fine-scale tracks (thin coloured lines), large-scale movement (thick lines), residence patch polygons, and fruit tree locations show how high-throughput data can be used to study movement across scales.
            Patches and lines are coloured by bat identity.
        }
        \label{preproc_fig_08}
    \end{figure}

    \section*{Discussion and Perspective}

    Recent technical advances in wildlife tracking have already yielded exciting new insights from massive high-resolution movement datasets \citep{aspillaga2021, aspillaga2021a, baktoft2017, baktoft2019, harel2016, harel2018, oudman2018, papageorgiou2019, tsoar2011, strandburg-peshkin2015, toledo2020, beardsworth2021a, beardsworth2021b, corl2020, vilk2021, lourie2021}, and high-throughput animal tracking is expected to become increasingly more common in the near future.
    \graffito{
        I also had the opportunity to test the residence patch method on pike tracked using YAPS \citet{baktoft2017}, and found it worked very well.
    }
    Tackling the very large datasets that high-throughput tracking generates requires a different approach from that used for traditionally smaller volumes of data.
    We foresee that movement ecologists will have to adopt ever more practices from fields accustomed to dealing with `big data', and that the field will become increasingly computational \citep{peng2011}.

    Researchers have long used some of these approaches \textit{ad hoc}, such as exploratory data analysis on small subsets before applying methods to the full data, using efficient tools, and basic batch-processing. 
    Yet formally prescribing these steps can help practitioners avoid pitfalls and implement techniques that make their analyses quicker and more reliable.
    Standardised principles, implemented a basic pipeline, for approaching data cleaning promote reproducibility across studies, making comparative inferences more robust.
    While massive datasets make reliance on standardised pipelines necessary, the output of such pipeline should periodically manually double-checked to ensure `realistic' output.
    The open-source \textit{R} package \textit{atlastools} serves as a starting point for methodological collaboration among movement ecologists, and as a simple working example on which researchers may wish to model their own tools.
    Efficient location error modelling approaches \citep{fleming2020, aspillaga2021} may eventually make data-cleaning optional.
    Yet cleaning tracking data even partially before modelling location error is faster than error-modelling on the full data, and the removal of large location errors may improve model fits.
    Thus we see our pipeline as complementary to these approaches \citep{fleming2014a, fleming2020}.

    Finally, we recognise that the diversity and complexity of animal movement and data collection techniques often requires system-specific, even bespoke, pre-processing solutions.
    Though the principles outlined here are readily generalised to numerous data sources (including terrestrial radio-based reverse-GPS: e.g. \citealt{toledo2020}, and marine acoustic reverse-GPS: e.g. \citealt{aspillaga2021}; high-resolution GPS such as \citealt{strandburg-peshkin2015}, and video-tracking: \citealt{rathore2020}), users' requirements will eventually exceed the particular tools we provide.
    % For instance, relational databases are the standard for storing very large datasets, and extending pre-processing pipelines to deal with various data sources is relatively simple, as we show in our Supplementary Material.
    We see the diversity of animal tracking datasets and studies as an incentive for more users to be involved in developing methods for their systems.
    We offer our approach to large tracking datasets, and our pipeline and package as a foundation for system-specific tools in the belief that simple, robust concepts are key to methods development that balances system-specificity and broad applicability.

    \newrefcontext[sorting=nyt]
    \printbibliography[title=Literature~Cited,heading=none]
\end{refsection}


\cleardoublepage 
\phantomsection
\pagestyle{plain}

\begingroup

% \let\clearpage\relax
% \let\cleardoublepage\relax
% \let\cleardoublepage\relax


\interlude{Mapping Animal Movement in \textit{R}}\label{box:mapping}
\noindent \textbf{Pratik R. Gupte}

\medskip

\setlist[description]{font=\scshape\bfseries\space}
% \footnotesize
\begin{description}
	\item[Mapping as Exploratory Data Analysis] Mapping animal movements is a key component of exploratory data analysis. 
	It is important to `join the dots' of animal positions. Large tracking datasets can contain errors that are only evident to researchers when they look at an approximation of the animal's path and ask, ``Does the animal move this way?''
	This map shows `jumps': long, linear segments between points, indicating missing data for some periods. 
	
	Mapping can also reveal interesting behaviours that can only be observed after significant effort in the field.
	The `looping' behaviour of \emph{AM253} to water sources is the focus of this map. 
	Seeing this looping behaviour allowed us to focus our study on elephant movements between visits to water sources.
	
	\item[Mapping as Art] Growing up in early 2000s India, I read hard copies of National Geographic Magazine, which has long had fantastic graphics. \emph{Where the Animals Go}\footfullcite{cheshire2017} was a source of inspiration as well. I built up the image in layers, used colours that don't clash, and highlighted the phenomenon of interest. These approaches chime with the `grammar of graphics' approach of \textit{ggplot}, which I used to make this map.
	
	\item[Mapping in R] R's great advantage over other languages is visualisaton, specifically the popular \emph{ggplot} package. \textit{ggplot}'s emergence as a mainstay of spatial visualisation is due to its \textit{geom\_sf} function, which can handle sf spatial objects.

	One of \emph{ggplot}'s advantages is its many extensions. Here, I used the \emph{ggspatial} and \emph{ggtext} extension packages to add the scale bars and north arrow, and to add the text box, respectively.
		
	Plotting rasters is not straightforward in \emph{ggplot}. There are two main options: the stars package and its associated \textit{geom\_stars}, or converting a raster dataset into a dataframe with regular coordinate intervals and using \textit{geom\_tile}.
		
	Here, I chose the second approach because I'm an infrequent stars user; since making the map I've tried \textit{geom\_stars} which works just as well, and is very convenient.

	\item[Reproducibility in R] I adopted a relatively relaxed understanding of reproducibility: given the data, the code would be reproducible if it could produce the map I had entered for this contest. To do this, I set up a continuous integration pipeline using Github Actions (GHA).

	Using the \textit{usethis} package, I created a `DESCRIPTION' file, which is usually reserved for packages. This file tricks GHA into reading its contents, especially the dependencies, i.e., the R packages required by the project.

	GHA automatically reads the dependencies and installs them, as well as the programs required by those dependencies. For instance, GDAL (the Geospatial Data Abstraction Library) is key to nearly all spatial analyses, and is installed as a requirement of the \textit{rgdal} package, which is itself key to \textit{sf} and \textit{raster}.

	I used the R package \textit{renv} to make sure that the packages (and the package versions) I used are available to the pipeline. \textit{renv} creates a lockfile, a registry of packages the current project uses, from which those packages can be installed.
	Finally, to check whether the entire pipeline works, I used \textit{bookdown} to sequentially execute the series of Rmarkdown files. An obvious alternative is \textit{rmarkdown}.

	GHA runs this pipeline and reports whether the code ran successfully, and if not, where it failed (you can see these reports here). GHA runs the pipeline on Linux, and Windows containers (Mac OS-x is also supported). This means that though I use Linux, I'm pretty sure that this code works for Windows users.

	\item[The Limits of Reproducibility] Reproducibility inevitably breaks down at certain scales in an ecological study. For instance, it would be impossible to reproduce the primary data collection of the study, such as which elephants were captured and fitted with transmitters. These data are taken on faith from the original researchers, highlighting the role of trust in the scientific community.

	In ten years, code in R or another language may no longer be reproducible due to software and hardware changes, as many researchers found in the 10-year reproducibility challenge. Finally, entire services might become unavailable; for example, the raster processing using Google Earth Engine is dependent on Google maintaining this service.

	Researchers then, should be pragmatic about reproducibility. Who is it for — the researchers themselves, the reviewers of their manuscript, their students, their funders?
	To whom this effort is owed, and by whom, and how the additional work required can be prevented from becoming a gatekeeping mechanism \footfullcite{finley2017}\textsuperscript{;}\footfullcite{murphy2020}, are are issues that the ecology and evolution community will have to address.
\end{description}

% { \begin{center} \barfont{-.-} \end{center} }

\subsection*{About This Map}

This map and text is adapted from a submission to the \textit{Methods in Ecology and Evolution} blog, after my entry won the BES' Mapping Animal Movements Contest (2020 -- 2021), in the reproducible ``R Map'' category. The map shows the movement of 14 female savanna elephants \textit{Loxodonta africana} tagged in Kruger National Park, South Africa, with a focus on the elephant \emph{AM253}.
The study that inspired this map was published as \citet{thaker2019} \citetitle{thaker2019}.

I coloured the temperature raster using the \textit{scico} package's `VikO' palette. I tried out a number of palettes from \textit{scico}, pals (providing the Kovesi palettes), \textit{RColorBrewer}, and \textit{colorspace} packages. I chose a diverging palette to show heterogeneity in the thermal landscape, but this approach is not to be recommended for material that will be printed in grayscale.

Map text is set in two related typefaces designed by the Dutch type foundry \emph{Bold Monday} for IBM: \emph{Plex Serif} --- for text on the map --- and \emph{Plex Sans} --- for text in the box.
While aiming to be text typefaces, I think both perform much better as `display' faces; Plex Serif especially so.

{ \begin{center} \barfont{-.-} \end{center} }

\includepdf{figures/boxes/elemove.pdf}

\endgroup

\afterpage{\nopagecolor}
\pagestyle{scrheadings}


\cleardoublepage %************************************************
\chapter{Individual Consistency in Movement Tendencies Across Spatial Scales}\label{ch:knots}
%************************************************


%*****************************************
%*****************************************
%*****************************************
%*****************************************
%*****************************************


\cleardoublepage %************************************************
\chapter{Contrasting Movement Strategies of Moulting Sub-tropical Birds}\label{ch:holeybirds}
%************************************************

\noindent \textbf{Pratik R. Gupte}, Yosef Kiat\textsuperscript{1}, Sivan Toledo\textsuperscript{2}, and Ran Nathan\textsuperscript{3}

\marginpar{
    \textsuperscript{1} University of Haifa, Israel.
    
    \medskip
    
    \textsuperscript{2} Tel Aviv University, Israel.
    
    \medskip

    \textsuperscript{3} The Hebrew University of Jerusalem, Israel.
}

\section*{Abstract}

\footnotesize{
    
    \medskip

    % \noindent {\large{\color{Maroon}$\Delta$}} Under review at \textit{The American Naturalist} as Gupte et al. The joint evolution of animal movement and competition strategies.
}

\clearpage



\cleardoublepage \begin{refsection}

%************************************************
\chapter{Land-cover and Climate Shape Bird Distributions in a Tropical Biodiversity Hotspot}\label{ch:hillybirds}
\chaptermark{Citizen Science \& Bird Distributions}
%************************************************
% Using citizen science to parse climatic and land cover influences on

{\noindent Vijay Ramesh\textsuperscript{1}, \textbf{Pratik R. Gupte}, Morgan Tingley\textsuperscript{2}, V.V. Robin\textsuperscript{3}, and Ruth S. de Fries\textsuperscript{1}}

\marginpar{
    \textsuperscript{1} Columbia University, USA.
    
    \medskip
    
    \textsuperscript{2} University of California --- Los Angeles, USA.
    
    \medskip

    \textsuperscript{3} Indian Institute for Science Education and Research --- Tirupati, India.
}

\section*{Abstract}

\small{
    Disentangling associations between species occupancy and its environmental drivers --- climate and land cover --- along tropical mountains is imperative to predict species distributional changes in the future.
    Previous studies have primarily focused on identifying such associations in temperate mountain systems.
    Using 1.29 million robustly processed citizen science observations contributed to eBird between 2013 and 2021, we examined the role of climatic and landscape variables and its association with bird species occurrence within a tropical biodiversity hotspot, the southern Western Ghats in India.
    Using an occupancy modeling framework, we found that temperature seasonality, precipitation seasonality, and the proportion of evergreen forests were significantly associated with species-specific probabilities of occupancy for 78\% (n = 43 birds), 38\% (n = 21 birds), and 27\% (n = 15 birds) of bird species examined, respectively.
    Our study shows that several forest birds (n = 18 species) were negatively associated with temperature seasonality, highlighting narrow thermal niches for such species.
    The probability of occupancy of six forest species and eight generalist species was positively associated with precipitation seasonality, indicating potential associations between rainfall and resource availability, and thereby, species occurrence.
    A smaller number of largely generalist species (n = 9 birds) were positively associated with human-modified land cover types --- including the proportion of agriculture/settlements and plantations.
    Our study shows that rigorously filtered citizen science observations can be used to identify associations between environmental drivers and species occupancy on tropical mountains.
    Though current distributions of tropical montane birds of the Western Ghats are strongly associated with climatic factors (mainly, temperature seasonality), naturally occurring land cover types (forests) are critical to sustaining montane avifauna across human-modified landscapes in the long run.

    \bigskip

    {\noindent \large{$\Delta$}} \normalfont Under review at \textit{Ecography} as Ramesh et al. ``Using citizen science to parse climatic and land cover influences on bird occupancy within a tropical biodiversity hotspot''.
}

\clearpage

 
\newrefcontext[sorting=ynt]

\lettrine{T}{ropical} montane ecosystems are hotspots of biological diversity and are home to over 70\% of the world's avian diversity in less than 10\% of global terrestrial area \citep{myers2000,davies2007,quintero2018}.
However, tropical mountains are under tremendous anthropogenic pressures of habitat modification and climate change, which can both have negative consequences for bird species \citep{nogues-bravo2007,newbold2015}.
In addition to directly affecting bird populations, climate change and changes in land cover can also affect species distributions in montane areas worldwide \citep{nogues-bravo2007,rahbek2019}.
For example, mountain birds in California tracked changes in temperature and precipitation over a century of climate change, illustrating the long-term role of climate in driving range shifts \citep{tingley2009}.
The movement of temperature bands upslope can eliminate the conditions to which high-altitude species are adapted, leading to local extinction \citep{freeman2018,urban2018}.
A combination of changes in climate and land cover best explains the colonization and extinction probabilities of North American birds \parencite{yalcin2018}.
However, few studies have disentangled the role of these two drivers on species' current distributions \citep{sirami2017}.
Furthermore, species distributions in tropical mountains especially are poorly studied despite being `escalators to extinction' for montane birds \citep{elsen2017,freeman2018,srinivasan2019,srinivasan2021}.
Understanding the contemporary drivers of species' distributions in tropical mountains can help predict future species ranges as the climate changes \citep{guo2018,srinivasan2021}.

The drivers of bird distributions in tropical montane ecosystems are poorly understood because data on species distributions in these regions are limited \citep{payne2017,peters2019}.
Citizen science efforts offer a solution: initiatives such as \textit{eBird} are growing in popularity and scale and make the observation data readily available to researchers \citep{sullivan2014}.
\textit{eBird} combines many thousands of decentralized, \textit{ad hoc}, organized, or semi-organized bird observations to form representative samples of species' occurrence over vast scales \citep{sullivan2009,sullivan2014,wood2011a}.
The standardization of the reporting infrastructure (e.g., the \textit{eBird} mobile app or website) allows observations to be reproducibly processed to achieve a high standard of reliability.
For example, one can filter out short observation sessions that might not accurately capture a location's bird community or weight observations by the observer's effort \citep{kelling2015a,johnston2018,johnston2021}.
Including data from citizen scientist observations can significantly improve species distribution models \citep{robinson2020}, and enable a wide range of research, including mapping species elevational movements \citep{tsai2020} and prioritizing conservation efforts \citep{vanstrien2013,fink2014,johnston2015}.
India reports one of the largest numbers of \textit{eBird} checklists from a tropical country, as birdwatchers have contributed to \textit{eBird} in a concerted and growing effort since 2014 \citep{viswanathan2020}.
Coordinated citizen science efforts have led to successfully mapping the distribution and abundance of birds across multiple regions in India \citep[e.g. the Mysore Bird Atlas, and the Kerala Bird Atlas][]{praveenj2021}.
As of March 2021, the \textit{eBird} India dataset has grown to a total of over 14 million observations across 1,342 species of birds.

We set out to examine the role of climate and land cover and its association with bird occupancy in a tropical montane region, the Western Ghats of southern India.
The Western Ghats mountain ecosystem is part of the Western Ghats-Sri Lanka biodiversity hotspot and is home to numerous species of endemic plants and animals \citep{myers2000,das2006}.
We examined observations from \textit{eBird} between 2013 and 2021 for 79 species (later reduced to 55, following model fitting) of birds across the two largest hill ranges in the southern Western Ghats --- the Nilgiri and the Anamalai-Palani hills (see Fig.~\ref{hilly_fig_01}a).
Specifically, we tested associations between climatic variables, land cover, and bird occupancy.
We binned species according to their habitat preference prior to hypothesis testing; a species could either be a forest species (species found in forested/woodland habitats as well as forest edges) or generalist species (widespread species found across a range of habitat types) \citep{ali1983}.

First, we examined the direction of association between species-specific probability of occupancy and climatic predictors.
Temperature seasonality: We tested the hypothesis that the probability of occupancy of forest specialist birds should be negatively associated with temperature seasonality (coefficient of variation) \citep{srinivasan2018}.
Tropical forest species are often associated with a narrow range of temperatures leading to the expectation that the probability of occupancy will decrease with increasing variation in temperatures \citep{janzen1967,stevens1989,frishkoff2016,chan2016,srinivasan2018}.
However, we expected that the occupancy of generalist species may be positively associated with temperature seasonality.
In other words, we expected that generalist species have broader thermal niches and occur in climatically variable regions when compared to their forest counterparts.
Precipitation seasonality: the `hygric' niche hypothesis states that species often occur within an optimal range of rainfall conditions \citep{boyle2020}.
Across our study area, we expected that precipitation seasonality (coefficient of variation) would be positively associated with species occupancy for forest birds and negatively associated with generalist bird species.
Forest species in the Western Ghats are largely seen in wetter habitats relative to generalist species that are more often found in drier habitats \citep{raman2006}.
Finally, we examined the direction of association between species-specific probability of occupancy and land cover.
We expected the occupancy of forest species to be positively associated with naturally occurring land cover types such as evergreen forests and deciduous forests.
We expected that human-modified land cover types, including agriculture, settlements, and plantations would be positively associated with species-specific probability of occupancy of generalist birds.

\section*{Preparing Data to Model Bird Species Occupancy}
% \addcontentsline{toc}{chapter}{\tocEntry{\color{black}\itshape{General Introduction: The Current Frontiers of Animal Movement Ecology}}}%

\subsection*{The Southern Western Ghats}

The Nilgiri and the Anamalai-Palani hills (hereafter, Anamalai hills) (Fig.~\ref{hilly_fig_01}) are part of the Western Ghats, an ancient region of differentiation of flora and fauna in South Asia \citep{mani1974,myers2000,vijayakumar2016}.
These hill ranges host a diversity of land cover types, possess a wide climatic gradient, and several bird species \citep{ali1983,das2006}.
The elevational range across these hill ranges varies from 40m in the plains to 2,625m in the higher elevations.
(Fig.~\ref{hilly_fig_01}a).
These two hill ranges are home to a multitude of habitats, ranging from high elevation grasslands (>1400m; Fig.~\ref{hilly_fig_01}b) to mid-elevation evergreen forests (>700m and <1400m; Fig.~\ref{hilly_fig_01}b).
These hill ranges interact strongly with the annual south-west monsoon resulting in orographic rainfall on the western slopes ($\sim$3000 mm) and a relative rain-shadow on the leeward eastern slopes ($\sim$2000 mm) that in turn influences the distribution of endemic flora and fauna \citep{gadgil1986,pascal1988,robin2015}.

\begin{figure}[h!]
    \centering
    \includegraphics[width=0.8\textwidth]{figures/hillybirds/fig_01.png}
    \caption{
        \textbf{The Nilgiri and Anamalai Hills in southern India provide a convenient geography for studying the interplay of land cover and climate on the distributions of bird species.}
        \textbf{(a)} The Nilgiri and Anamalai Hills of the Southern Western Ghats are topographically complex, with maximum elevations > 2,000 m, and are separated by the very low-lying Palghat Gap, which serves as a natural barrier to the dispersal of many hill birds. 
        \textbf{(b)} Lower elevations are primarily covered by agriculture and settlements, reflecting the intense human pressure on this region, while mid- and higher elevations show a mix of natural and human-modified land cover types (see Fig.~\ref{hilly_fig_02} for details). 
        \textbf{(c)} The coastal edge of the area, and the windward hill slopes show limited temperature seasonality across the December -- May period; this seasonality increases with distance from the coast but is lower at higher elevations inland. 
        \textbf{(d)} Higher elevations also show limited precipitation seasonality than both low-lying coastal and inland regions. 
        Our study area (bounds shown as dashed lines) includes multiple combinations of elevation, land cover type, and temperature and rainfall seasonality, resulting in a naturally occurring crossed-factorial design that allows us to study the effects of climate and land cover on bird occupancy. 
        Representative forest-restricted and habitat-generalist birds from the study area are shown between panels (all images were obtained from Wikimedia commons and credit is assigned for each species in brackets). 
        Elevation is from 30m resolution SRTM data (Farr et al. 2007), land cover, at 1km resolution, is reclassified from Roy et al. (2015), while climatic variation is represented by CHELSA seasonality layers (temperature: BIOCLIM 4a, rainfall: BIOCLIM 15), at 1km resolution (Karger et al. 2017). All layers were resampled to 1km resolution for analyses.
    }
    \label{hilly_fig_01}
\end{figure}

\begin{figure}[h!]
    \centering
    \includegraphics[width=0.9\textwidth]{figures/hillybirds/fig_02.png}
    \caption{
        \textbf{Climate and land cover vary strongly along the elevation gradient in the Nilgiri and Anamalai Hills.}
        Both \textbf{(a)} temperature seasonality and \textbf{(b)} precipitation seasonality, between the months of December and May, declines with increasing elevation across the Nilgiri and Anamalai Hills. 
        Climatic variation is not very strongly associated with land cover type, as both natural habitats such as forests, and human-associated habitat types such as plantations show low seasonality in \textbf{(c)} temperature, and \textbf{(d)} precipitation.
        \textbf{(e)} Most elevations host a range of land cover types: while human-associated habitats such as agriculture are concentrated at lower elevations, and more natural types such as grasslands and forests are associated with higher elevations, each of these types is also found outside their characteristic elevational bands.
        We calculated climate seasonalities (BIOCLIM 4a and 15: temperature and precipitation, respectively) using CHELSA data over 1979 -- 2013, from December to May (Karger et al, 2017), and present mean seasonality values (vertical bars show standard deviation) for every 200m elevational band
        Land cover types were taken from a reclassification of Roy et al
        (2015; see main text) at 100m elevational bands.
        Land cover types covering < 1\% of an elevational band are shaded grey
        All landscape layers were first resampled to 1km resolution.
    }
    \label{hilly_fig_02}
\end{figure}

\subsection*{Filtering eBird Data}

Data from \textit{eBird} is available in the form of a `checklist' submitted by an observer or a group of observers.
Each checklist includes a wide range of information that includes species identity, latitude, longitude, date of observation, distance traveled, time spent observing etc.
`Complete' checklists indicate that the observer(s) recorded all the birds detected and identified.
We obtained bird detections from such complete checklists contributed to \textit{eBird} for nine years (2013 to 2021) across the Nilgiri and Anamalai hill ranges.
Only checklists recorded during December to May (non-rainy months) were included in our study because detecting birds during the rainy months is difficult due to poor weather.
Restricting our data to complete checklists also allowed us to interpret the absence of a species on a checklist as a non-detection \citep[called zero-filling][]{johnston2021}.
Even when restricting analysis to only `complete' checklists, the semi-structured, flexible nature of databases like \textit{eBird} results in large variation in effort across checklists as a result of the often opportunistic nature of data collection \citep{kelling2019}.
Complete checklists are marked as `Stationary' or `Traveling' based on the distance traveled by an observer while recording detections.
To reduce variation in observer effort, we first considered only those complete checklists with a duration ≤ 300 mins (5 hrs), and distance ≤ 5km (for traveling checklists), and with fewer than 10 observers \citep[following][]{johnston2019}.
Since stationary birdwatchers can detect birds up to 100m away, we set all stationary checklists to a distance of 100 m.
In many cases, checklists are submitted by a single observer for a group of birdwatchers; in such cases, the group checklist only occurs once in the dataset.
We used only checklists recorded between 5:00 AM and 7:00 PM to avoid sightings in low-light conditions.

\subsection*{Selecting Study Species}

We limited our study to 79 species of terrestrial, diurnal birds that occur in our study region (see list of species at end; see Fig.~\ref{hilly_fig_01} for representative species).
We selected these species using inclusion criteria adapted from the State of India's Birds Report 2020 \citep[SoIB][]{viswanathan2020}.
We intended these criteria to ensure uniform sampling of each species across our study area, and to reduce erroneous associations between environmental drivers and species distributions.
Beginning with 3.37 million observations of 684 species in \textit{eBird} that occurred within the outlines of our study area (Fig.~\ref{hilly_fig_01}a), over the years 2013 -- 2021, we retained only those species that had a minimum of 1,000 detections each between 2013 and 2021 (347 species remaining; 3.33 million observations).
Next, we divided the study area into 25 $\times$ 25km grid cells (42 unique cells; see supplementary material).
We kept only those species that occurred in at least 5\% of all checklists across at least 27 unique grid cells (50\% of the study area).
We further manually removed raptors (\textit{Accipitriformes} and \textit{Falconidae}), swifts (\textit{Apodiformes}), and swallows (\textit{Hirundinidae}) since these birds are usually observed in flight when species identification can be prone to errors.
This filtering process resulted in a total of 1.29 million observations (presences) across our study area.

\subsection*{Spatio-Temporal Bias in Occurrence Data}

Sampling bias can be introduced into citizen science observations due to the often opportunistic nature of data collection \citep{sullivan2014}.
For \textit{eBird}, this translates into checklists reported when convenient, rather than at regular or random points in time and space, leading to non-independence in the data if observations are spatio-temporally clustered \citep{johnston2021}.
For example, sites near roads are easier to reach and maybe sampled more frequently.
The spatial clustering of observations can be reduced by sub-sampling at an appropriate spatial resolution \citep{aiello-lammens2015}; however, thinning the data over-zealously can result in very few presence records compared to absence records \citep[i.e., class imbalance][]{steen2019}.
Consequently, when there are many more absence records than presence records, presences and absences should be handled separately when spatially thinning the data.

We first estimated two simple measures of spatial clustering: the distance from each site to the nearest road \citep[road data from OpenStreetMap:][]{openstreetmapcontributors2017} and the nearest-neighbor distance for each site.
Sites were strongly tied to roads (see Fig.~\ref{hilly_fig_03}a.; mean distance to road $\pmin$ SD = 390.77 $\pmin$ 859.15m; range = 0.28m -- 7.64km) and were on average only 297m away from another site (SD = 553m; range = 0.14m -- 12.85km).
This is understandable, as roads and trails provide access, and particular well-known areas are visited often.
On average, across species, presences comprised only 8.5\% of all observations.
We followed \textcite{steen2021} in choosing to spatio-temporally thin only the absences, and not the presences, for each species --- a methodology called `thin majority' that can improve model performance \citep{steen2021}.
To do this, we divided the study area into a grid of 500m wide square cells, and from within each cell, we chose the site with the most visits (checklists) over the sampling period.
From each of the remaining sites, we selected a maximum of 10 random absence checklists to reduce temporal clustering, keeping all absence checklists for sites with ≤ 10 checklists.
We retained all presences for each species without any spatial or temporal thinning \citep{steen2021}.
As a result of class balancing, in our final dataset, presences made up 29.3\% of observations on average across species.

\afterpage{
    \begin{sidewaysfigure}[p]
        \centering
        \includegraphics[width=0.9\textwidth]{figures/hillybirds/fig_03.png}
        \caption{
            \textbf{Distribution of sampling effort in the form of \textit{eBird} checklists in the Nilgiri and Anamalai Hills between 2013 and 2021.}
            \textbf{(a)} Sampling effort across the Nilgiri and Anamalai Hills, in the form of \textit{eBird} checklists reported by birdwatchers, mostly takes place along roads, with the majority of checklists located $<$ 1km from a roadway (see distribution in inset), and therefore, only about 300m, on average, from the location of another checklist.
            \textbf{(b)} \textit{eBird} checklists are also strongly clustered in time, with some of the most sampled areas over the study period visited at intervals of $>$ 1 week, and with some less intensively sampled areas visited frequently, at intervals of $<$ 1 week.
            Overall, most checklists are reported only a day after the previous checklist at that location (see inset).
            Both spatial and temporal clustering make data thinning necessary.
            Both panels show counts or mean intervals in a 2.5km grid cell; the study area is bounded by a dashed line, and roads within it are shown as (a) blue or (b) red lines.
        }
        \label{hilly_fig_03}
    \end{sidewaysfigure}
}

\subsection*{Adjusting for Spatial Precision}

Every checklist on \textit{eBird} is associated with a latitude and longitude.
However, the coordinates entered by an observer may not accurately depict the location at which a species is detected.
Such an error can occur for two reasons: first, traveling checklists are associated with a single location along the route travelled by observers.
Second, checklist locations could be assigned to a `hotspot' --- a location that is automatically marked by \textit{eBird} as being frequented by multiple observers --- even though the observation was not made at the precise location of the hotspot \citep{praveenj.2017}.
Since a large proportion of observations occur within 3km of the observation effort's starting point, we adjusted for the spatial precision of \textit{eBird} records by considering a buffer radius of 2.5km around each site when sampling environmental covariate values.

\subsection*{Calibrating Observations Across Observers}

Differences in bird identification skills among citizen scientists can lead to biased species detection when compared with data collected by a consistent set of trained observers \citep{vanstrien2013}.
Including observer calibration (that accounts for observer-specific differences in identification) as a detection covariate in occupancy models using \textit{eBird} data can help account for this variation \citep{johnston2018}.
Observer-specific calibration in local avifauna was calculated following \textcite{kelling2015a} as the normalized predicted number of species reported by an observer after 60 minutes of sampling across the most common land cover type within the study area (in our case, deciduous forests).
This score was calculated by examining checklists from anonymized observers across the study area.
We modified the \citep{kelling2015a} formulation by including only observations of the 79 species of interest in our calculations.
An observer with a higher number of species of interest reported within 60 minutes would have a higher observer-specific calibration score, with respect to the study area.
We then estimated a checklist calibration index (CCI) from observer-specific calibration scores associated with each checklist.
The CCI was the lone observer's calibration score for single-observer checklists and the highest calibration score among observers for group checklists.
CCI is predicted from a generalized linear mixed effects model:
\begin{multline*}
    \text{n} \sim \text{duration} + \sqrt{\text{duration}} + \text{landcover} + \sqrt{\text{time of day}} + \sqrt{\text{time of day}}^2 + \\ \log({\text{julian date})} + \log({\text{julian date})}^2 + (1 | \text{observer}) + \\(0 + \text{duration} | \text{observer})
\end{multline*}
where, $n$ is the number of species observed in that checklist, \textit{duration} is the time spent observing birds for the checklist, \textit{landcover} refers to the land cover type, \textit{time~of~day} refers to the time of the day that observations were made, and \textit{julian~date} refers to the ordinal day of the year.

\subsection*{Preparing Occupancy Predictors}

We prepared a suite of climatic and land cover variables to be modeled as covariates of species-specific probabilities of occupancy within our full study region (Figs.~\ref{hilly_fig_01} -- \ref{hilly_fig_02}).
Among climatic predictors, we chose to examine the effects of temperature and precipitation seasonality on species occupancy, and we obtained these predictors at a spatial scale of 1km \citep[Climatologies at High resolution for the Earth's Land Surface Areas; CHELSA:][]{karger2017}.
Temperature seasonality is defined as the amount of temperature variation over a given time period based on the ratio of the standard deviation of the monthly mean temperatures to the mean of the monthly temperatures \citep{odonnell2012}.
In other words, temperature seasonality is the coefficient of variation and captures the dispersion in relative terms because standard deviation can produce two similar values while the means may be different.
Larger values of temperature seasonality imply higher variability in temperature, relative to the average temperature.
It is important to calculate variability relative to the mean because the same amount of statistical variability (e.g., variance) in a dry area as a wet area would have a much bigger `seasonality' impact on a dry area.
Similarly, we defined precipitation seasonality as the ratio of the standard deviation of the monthly total precipitation to the mean monthly total precipitation \citep{odonnell2012}.
The above calculations of seasonality were made using temperature and precipitation data from CHELSA for the non-monsoon months of December to May for our study area.

While data from global databases such as WorldClim have been used for modeling species distributions, CHELSA data has shown greater predictive power \citep{karger2017} and hence we used the latter in this study.
Other bioclimatic predictors such as mean annual temperature, mean annual precipitation, mean temperature of the coldest or driest quarter, or the precipitation of driest or coldest quarter were equally well suited for our study; however, they were highly correlated ($|r|$ $>$ 0.5) with temperature and precipitation seasonality.
We obtained land cover over our study site from a high-resolution vegetation type map generated by \citep{roy2015}, using medium resolution IRS-LISS III (Indian Remote Sensing Satellite - Linear Imaging Self Scanner) images (http://bis.iirs.gov.in/).
This classification was originally generated at a scale of $\sim$23m and with 22 land cover classes for our study area.
We aggregated these 22 classes into seven broad, ecologically relevant land cover types: evergreen forests, deciduous forests, mixed/degraded forests, agriculture/settlements, plantations, grasslands, and water bodies (see Supplementary Material).
We resampled the reclassified land cover layer using a nearest neighborhood approach to 1km to match the 1km resolution of the climatic layers.

Testing for collinearity among the climatic and land cover predictors did not result in the removal of any predictors as the correlations were low ($|r| <$ 0.5).
We then pooled the climatic (n = 2) and land cover (n = 7) predictors and calculated mean values for the two climatic predictors (temperature seasonality and precipitation seasonality) and calculated the proportion of each of the seven land cover types within the 2.5km buffer radius around each spatio-temporally thinned locality for each species.

\subsection*{Estimating Species Occupancy}

Occupancy models estimate the probability of occurrence of a given species while controlling for imperfect detection and allow us to model the factors affecting occurrence and detection independently \citep{mackenzie2017,johnston2018}.
The flexible \textit{eBird} observation process contributes to the largest source of variation in the likelihood of detecting a particular species \citep{johnston2021}; hence, we included six continuous covariates that influence the probability of detection for each checklist: ordinal day of year, duration of observation, distance travelled, time of day of observations, number of observers, and the checklist calibration index (CCI).
We converted calendar date into a linear, continuous predictor by extracting ordinal days of the year (julian date) for December to May and scaling them between 1 and 183 (dates in December subtracted from 333, and 31 added to dates between January and May).
This time period essentially includes winter and summer seasons (loosely defined) in the Western Ghats where detectability of bird species is high.
Our breeding season is often toward the end of this window (late April to early May) when resident species begin to breed while migratory birds travel back to their breeding grounds.
We modeled time of day so as to allow detectability to be highest at dawn and dusk when birds often sing and are easily detected, and to be lower in the middle of the day, when birds are least active and thus less likely to be detected.

Using a multi-model information-theoretic approach, we tested how strongly our occurrence data fit our candidate set of environmental covariates \citep{burnham2002}.
We fitted single-species occupancy models for each species, to simultaneously estimate a probability of detection ($p$) and a probability of occupancy ($\psi$) \citep{mackenzie2002,fiske2011}.
For each species, we fit 512 models, each with a unique combination of the (climate and land cover) occupancy covariates and all detection covariates (the six detection covariates are present in every model).

\begin{multline*}
    \text{logit}(p) \sim \text{julian date} + \text{duration} + \text{distance} + \text{obs. started} + \\
    \text{number observers} + CCI    
\end{multline*}

where \textit{julian date} refers to the ordinal day of the year, \textit{duration} refers to the time spent observing birds in minutes, \textit{distance} is the distance travelled by the observer(s) in kilometers, \textit{obs. started} is the time of day when observations were recorded, \textit{number observers} refer to the number of observers, and $CCI$ is the checklist calibration index.

\begin{multline*}
    \text{logit}(\psi) \sim \text{BIO 4a} + \text{BIO 15} + p(\text{Evergreen}) + p(\text{Deciduous}) + \\
    p(\text{MixedDegraded}) + p(\text{AgricultureSettlements}) + p(\text{Plantations}) + \\
    p(\text{Grasslands}) + p(\text{WaterBodies})
\end{multline*}

Previously, we explored the non-linear effects of temperature seasonality and precipitation seasonality.
However, our occupancy models showed a poor fit to the data when temperature and precipitation seasonality were included as non-linear terms and hence, we did not explore this further.
Adequate model fit was assessed using a chi-square goodness-of-fit test using 1,000 parametric bootstrap simulations on a global model that included all occupancy and detection covariates (MacKenzie and Bailey 2004).
Across the 512 models tested for each species, the model with highest support was determined using AICc scores.
However, across the majority of the species, no single model had overwhelming support.
Hence, for each species, we examined those top models which had a difference in AICc of $<$ 4, as these top models were considered to explain a large proportion of the association between the species-specific probability of occupancy and environmental drivers \citep{burnham2011}.
Using these restricted model sets for each species; we created a model-averaged coefficient estimate for each predictor and assessed its direction and significance \citep{barton2009}.
These model-averaged coefficients include zeros when a predictor is absent in one of the top models.
In addition, we estimated a model-averaged standard error using which we calculated a 95\% confidence interval \citep{burnham2002}.
We considered a predictor to be significantly associated with occupancy if the range of the 95\% confidence interval around the model-averaged coefficient did not contain zero.

Prior to further inference, all 79 birds in our study were classified as forest species or generalist species following \citep{ali1983}.
Forest species are those that are typically found in wet evergreen, semi-evergreen, deciduous, moist deciduous forests, and other woodland habitats as well as forest edges.
This classification encompasses specialist endemic birds, species that occur in woodland habitats as well as those species found along the edges of forested areas.
Generalist species are those that are typically found across a range of habitat types such as forests, agricultural lands, settlements, etc.
All continuous covariates were standardized prior to analysis, allowing for the comparison of model-averaged coefficients between species.
We used the R packages unmarked, and MuMIn for occupancy modeling and model averaging; the code provided in the supplementary material and on our Github repository shows the full set of R and Python packages used in this work \citep{barton2009,fiske2011,r2020}.

\subsection*{Data Deposition}

All data were downloaded from \textit{eBird} (version 1.13) and can be accessed via: http://\textit{eBird}.org/data/download.
The complete analysis is available as Supplementary Material (https://github.com/vjjan91/\textit{eBird}Occupancy) and is archived on Zenodo (https://doi.org/10.5281/zenodo.6025640).

\section*{Species Occupancy in the Southern Western Ghats}

Following spatio-temporal thinning of observations, we relied on 315,428 curated citizen scientist observations (including both presences and non-detections) across 79 species of birds between 2013 and 2021 for modeling occupancy.
The number of detections varied from a minimum of 224 observations to a maximum of 7,725 observations per species (following spatio-temporal thinning).
Chi-square goodness-of-fit tests suggested a poor model fit for twenty-four species ($p <$ 0.05) and hence these species were removed before further analysis (resulting in a total of 55 species).
Of the list of 55 species, six species were migratory species (long-distance/altitudinal) that are present in our study area during the focal seasonal time period: Blyth's reed warbler \textit{Acrocephalus dumetorum}, Brown shrike \textit{Lanius cristatus}, Chestnut-headed bee-eater \textit{Merops leschenaulti}, Grey wagtail \textit{Motacilla cinerea}, Eurasian hoopoe \textit{Upupa epops} and Ashy drongo \textit{Dicrurus leucophaeus}.

\subsection*{Bird-Climate Associations}

The probability of occupancy of $\sim$78\% (n = 43 out of 55) of species examined was significantly ($p <$ 0.05) associated with temperature seasonality.
18 bird species (n = 14 generalist birds and 4 forest birds) showed a positive association with temperature seasonality, while 25 bird species (n = 7 generalist birds and 18 forest birds) were negatively associated (Fig.~\ref{hilly_fig_04}).
The probability of occupancy of $\sim$38\% of (n = 21 out of 55) species examined had a significant association with precipitation seasonality.
14 bird species (n = 8 generalist birds and 6 forest birds) showed a positive association, while seven bird species (n = 5 generalist birds and 2 forest birds) were negatively associated with precipitation seasonality.

\begin{figure}[h!]
    \centering
    \includegraphics[width=0.9\textwidth]{figures/hillybirds/fig_05.png}
    \caption{
        \textbf{Probability of occupancy as a function of temperature seasonality.}
        Predicted probability of occupancy curves as a function of temperature seasonality for four forest species are shown here. 
        Temperature seasonality is negatively associated with the probability of occupancy of several forest species including the Asian fairy-blu\textit{eBird} (\textit{Irena puella}), the crimson-backed sunbird (\textit{Leptocoma minima}), the chestnut-headed bee-eater (\textit{Merops leschenaulti}) and the Malabar whistling-thrush (\textit{Myophonus horsfieldii}).
    }
    \label{hilly_fig_05}
\end{figure}

\begin{figure}[h!]
    \centering
    \includegraphics[width=0.9\textwidth]{figures/hillybirds/fig_04.png}
    \caption{
        \textbf{Environmental predictors and species-specific associations}.
        The direction of association between species-specific probability of occupancy and climatic and landscape predictors is shown here (as a function of habitat preference). 
        Blue colors show the number of species that are positively associated with a climatic/landscape predictor while red colors show the number of species that are negatively associated with a climatic or landscape predictor.
    }
    \label{hilly_fig_04}
\end{figure}

\subsection*{Bird-Land Cover Associations}

Twenty-seven percent of species (n = 15 out of 55) were significantly associated with the proportion of evergreen forests.
Of these species, eight forest birds were positively associated.
Among generalist birds that showed a significant association with the proportion of evergreen forests, three species were positively associated while four were negatively associated.
A fewer number of species (n = 4) were significantly associated with the proportion of deciduous forests (positive association with two forest species and a negative association with two generalist bird species).
Six bird species showed a significant association with the proportion of grasslands.
Of these species, three forest bird species and three generalist birds showed a negative association (Fig.~\ref{hilly_fig_04}).
Five bird species were significantly and positively associated with the proportion of water bodies (n = 3 generalist birds and 2 forest birds).

\begin{figure}[h!]
    \centering
    \includegraphics[width=0.9\textwidth]{figures/hillybirds/fig_06.png}
    \caption{
        \textbf{Predicted area of occurrence for four forest species.} 
        The probability of occupancy of the Asian fairy-blu\textit{eBird} (\textit{Irena puella}), the crimson-backed sunbird (\textit{Leptocoma minima}) and the chestnut-headed bee-eater (\textit{Merops leschenaulti}) is higher across the western slopes and at mid-elevations across our study area. The Malabar whistling-thrush (\textit{Myophonus horsfieldii}) has a higher probability of occupancy across mid-elevations throughout the study area examined.
    }
    \label{hilly_fig_06}
\end{figure}

33\% (n = 18 out of 55) of species examined were significantly associated with human-modified land cover types --- including the proportion of agriculture or settlements, plantations, and mixed or degraded forests.
One forest species showed a negative association, and one generalist species was positively associated with the proportion of mixed or degraded forests.
Five bird species showed a significant association with the proportion of agriculture or settlements.
Of these species, two generalist bird species showed a positive association while one forest species and two generalist bird species showed a negative association.

Eleven bird species showed a significant association with the proportion of plantations.
Of these species, one forest bird species showed a negative association, and one forest bird species showed a positive association.
Among generalist birds that showed a significant association with the proportion of plantations, seven birds were positively associated while two birds showed a negative association.

\section*{Interpreting Species Occupancy Models for the Southern Western Ghats}

Our study shows that rigorously filtered and curated citizen science observations can be used within a robust statistical framework to inform our understanding of how environmental drivers are associated with species distributions.
We highlight the role of climate and land cover and its associations with bird occurrences along a tropical montane gradient in a biodiversity hotspot, the southern Western Ghats.

\subsection*{Role of Temperature}

Tropical montane birds are especially vulnerable to ongoing changes in climate \citep{sekercioglu2007,perez2016,freeman2018,srinivasan2018}.
As a result of reduced temperature seasonality in the tropics relative to temperate regions, montane species in particular exhibit narrow thermal niches and hence, are likely to be unable to shift their distributions to track future climate changes \citep{janzen1967,deutsch2008,tewksbury2008,jankowski2013}.
Previous work in tropical areas across the globe have demonstrated that forest species are adapted to thermally aseasonal environments, while generalist species are more adapted to thermally variable, seasonal environments \citep{frishkoff2016,chan2016}.
In line with previous work, our study showed that several forest bird species (n = 18) were negatively associated with temperature seasonality.
Species such as the crimson-backed sunbird \textit{Leptocoma minima}, Asian fairy-blu\textit{eBird} \textit{Irena puella} and the chestnut-headed bee-eater \textit{Merops leschenaulti} for example showed a negative association (Fig.~\ref{hilly_fig_05}; Fig.~\ref{hilly_fig_06}).
The above result suggests that forest birds across the elevational gradient are potentially associated with narrow thermal niches.
Similar results have been demonstrated in the Western Himalayas, where birds occurring in forested habitats have narrow thermal niches relative to species in other land cover types \citep{srinivasan2019}.

In line with our hypothesis, the probability of occupancy of several generalist bird species (n = 14) was positively associated with temperature seasonality.
For example, the red-vented bulbul \textit{Pycnonotus cafer}, purple sunbird \textit{Cinnyris asiaticus}, and the spotted dove \textit{Streptopelia chinensis} showed a positive association.
Our result suggests that such generalist species occupy areas that show large variation in temperatures - including drier open habitats such as mixed or degraded forests and agricultural lands.
In fact, temperatures across tropical agricultural lands have been shown to be 7.6{\textdegree}C higher than temperatures within tropical primary forests \citep{senior2017}.
Generalist bird species that showed a positive association likely possess broad thermal niches, relative to their forest counterparts.
However, our study also reported a negative relationship with temperature seasonality for seven generalist bird species, including the red-whiskered bulbul \textit{Pycnonotus jocosus} and the Oriental magpie-robin \textit{Copsychus saularis}.
Future studies need to consider climate-land cover interactions to explore patterns seen for generalist species.

\subsection*{Role of Precipitation}

The significant association with precipitation seasonality suggests the importance of the `hygric niche', which has been seldom explored empirically \citep{boyle2020}.
In other words, species' occupancy is often governed by a range of precipitation regimes which vary in turn by land cover type and topographic complexity \citep{nowakowski2018}.
Several forest and generalist species showed a positive association with precipitation seasonality.
Research from the Australian tropical rainforests suggests that precipitation seasonality was strongly associated with bird abundance \citep{williams2008}.
In addition, precipitation seasonality has been reported as a crucial factor influencing resource availability (e.g., insects) for bird populations \citep{loiselle1991}.
The positive association between precipitation seasonality and species occupancy (for forest and generalist birds) reported in this study can be explained by the cascading effect of rainfall on food availability and, thereby survival of birds \citep{butt2015,boyle2020}.

In our study, forest species such as the southern hill myna \textit{Gracula indica} and the crimson-backed sunbird \textit{Leptocoma minima}, and generalist species such as the rose-ringed parakeet \textit{Psittacula krameri} and the Indian white-eye \textit{Zosterops palpebrosus} showed a positive association with precipitation.
On the other hand, we found that generalist species like the coppersmith barbet \textit{Psilopogon haemacephalus} and the red-vented bulbul \textit{Pycnonotus cafer} were negatively associated with precipitation seasonality.
Many of these generalist bird species that showed a negative association is associated with drier habitats across our study area.
Similar results have been reported from the neotropics where bird species largely associated with open habitats tend to prefer drier climates \citep{frishkoff2016}.
The above result merits further exploration that tests the interaction between precipitation seasonality and habitat structure and floristics in determining habitat use.
With increasing variability in rainfall patterns, it remains to be seen whether forest, as well as generalist bird species, adapt to such changes in the near future.
For instance, models have predicted reduced rainfall across regions in the Western Ghats as a result of future climatic changes \citep{rajendran2012}.

\subsection*{Role of Naturally Occurring Vegetation and Landscape Transformation}

Apart from climate, certain land cover types are hypothesized to be crucial for many species, as they offer resources necessary for survival, breeding, and other activities \citep{sunarto2012}.
For insectivorous birds in central Jamaica, the landscape matrix and habitat type were vital in determining occupancy \citep{kennedy2011}.
Our study suggests a positive relationship for several forest species across naturally occurring land cover types --- evergreen and deciduous forests.
Few generalist species such as the Blyth's reed warbler \textit{Acrocephalus dumetorum} and gray wagtail \textit{Motacilla cinerea} were positively associated with the proportion of evergreen forests.
The above association can be attributed to the fact that the Blyth's reed warbler and the gray wagtail have been reported from forest edges as well as plantations and agricultural areas in the vicinity of evergreen forests.
It is also likely that our minimum spatial scale of 2.5km was coarse and resulted in sampling multiple land cover types.

As expected, several generalist bird species showed a positive association with human-modified land cover types.
This association highlights the role of habitat transformation.
The southern Western Ghats have undergone a drastic transformation in the last two decades, with the replacement of mid- and high-elevation forests and grasslands with exotic trees and plantations \citep{arasumani2018}.
In the Nilgiris alone, the area covered by exotic trees has almost doubled, from approx.
140 sq.km to 277 sq.km in the 44-year period between 1973 and 2017.
Generalist birds such as the jungle myna \textit{Acridotheres fuscus} and the red-whiskered bulbul \textit{Pycnonotus jocosus} were positively associated with the proportion of plantations.
On the other hand, we did see forest species like the Malabar whistling thrush \textit{Myophonus horsfieldii} showing a positive association with the proportion of plantations, which could be an artifact of this species often being reported in not only forested areas but forest edges and plantations as well.
In a complex matrix that is the Western Ghats, our results further lend support to the role of natural vegetation within these human-modified landscapes in sustaining biodiversity in the long term \citep{anand2010,ranganathan2010}.
For example, windbreaks, which are often thin slivers of natural vegetation present in tea plantations in our study area, have been shown to possess similar bird species richness compared to adjacent primary forests \citep{sreekar2013}.
In a similar vein, data from the Anamalai hills suggests that native shade trees within tea plantations bolster avian species richness almost two-fold compared to tea plantations without native shade trees \citep{raman2021}.
Furthermore, the type of human-modified land cover type matters too, and coffee, rubber, and areca plantations across the Western Ghats have been shown to support more bird species than tea plantations \citep{sidhu2010,karanth2016}.

\subsection*{Caveats and Conclusions}

Our analysis was carried out using semi-structured data derived from a large citizen science project.
The lack of experimental and sampling design of this study is a persistent criticism of citizen science research.
For example, a large proportion of checklists were reported within 200m of a road, which are relatively more accessible (Fig.~\ref{hilly_fig_03}a).
This pervasive spatial bias in sampling could impact results in ways that cannot be corrected via spatio-temporal filtering of data.
While citizen science observations are often seen as supplementary to (presumably) more rigorous, methodical sampling by trained observers, such sampling designs are often not logistically feasible at large spatial scales.
In under-studied or under-sampled regions, citizen scientists and their observations are first-class data sources with significant exploratory and explanatory power \citep{devictor2010,ellwood2017,robinson2020}.

Recent evidence also suggests that a species' response to environmental gradients or to drivers such as land cover and climate will vary as a function of biological traits \citep{mcgill2006}.
Our study classified species as forest species or generalist species \citep{ali1983}.
Other traits might better explain associations between climatic and land cover predictors and species' occupancy.
For example, body mass is often considered an indicator of thermoregulation, and has been shown to be strongly associated with thermal niches of species, particularly temperate species, and high elevation tropical species \citep{barve2021}.
Similarly, functional traits such as trophic niches, that explain dietary preferences of a particular species are often associated with the use of a particular habitat \citep{pigot2020}.
Himalayan birds --- which encounter a comparable, if wider, range of temperatures --- have been shown to use forest and agriculture habitats to cope with resource scarcity in winter, possibly indicating greater dietary generalization than previously thought \citep{elsen2018}.
Including functional traits is a promising avenue to better understand species' response to environmental change across human-modified landscapes in the Western Ghats, and tropical mountains more generally.

Over 60\% of mountainous landscapes across the planet are under tremendous anthropogenic pressures, and yet host some of the highest biodiversity in the world \citep{lasorte2010,elsen2020}.
The southern Western Ghats is one such human-dominated mountainous landscape, where understanding the role of climatic and landscape predictors in structuring species occupancy can inform conservation.
In this study, we show that species have differential responses to climate (temperature and precipitation) and natural and human-modified land cover types.
If species need to adapt to environmental changes, they need to be able to track their suitable climatic and habitat niche space, which may only be possible through the creation of climate corridors \citep{freeman2018}.

\subsection*{Species in Figure~\ref{hilly_fig_01}}

From left to right: (1) Malabar grey hornbill (by Koshy), (2) Crimson-backed sunbird (by Mandar Godbole), (3) Asian emerald dove (by Selvaganesh), (4) Black-and-orange flycatcher (by LKanth), (5) Grey-headed canary flycatcher (by David Raju), (6) Greater-racket tailed drongo (by MD Shahanshah Bappy), (7) Eurasian hoopoe (by Zeynel cebeci), (8) Chestnut-headed bee-eater (by Mik\textit{eBird}s), (9) Coppersmith barbet (by Raju Kasambe), (10) Red-vented bulbul (by TR Shankar Raman), (11) Pied bushchat (by TR Shankar Raman), (12) Ashy prinia (by Rison Thumboor). All figures from Wikimedia Commons, and credited to creators in parentheses.

{\centering\barfont{-.-}}

% \newrefcontext[sorting=nyt]
% \section*{Literature Cited}
% \printbibliography[title=Literature~Cited,heading=none]
% \end{refsection}


% \ctparttext{
%     Mechanistic, individual-based models can capture the complexity and diversity of animal behaviour with spatial contexts --- simultaneously, they can be used to investigate methods used in empirical studies.
    
%     \medskip

%     \textit{Chapter 3} presents a translation of the step-selection paradigm into an individual-based model to study the evolution of movement and competition strategies.
%     Feedbacks between ecology and evolution lead to unexpected outcomes, including substantial individual variation, and the correlation of movement and competition strategies.

%     \medskip

%     \textit{Chapter 4} takes the evolved populations from \textit{Chapter 3} and parses them through the lens of two methods in animal behaviour: repeatability analysis, and step-selection analysis.
%     Applying these methods to scenarios with known mechanisms explains aspects of their performance on real data.
% }
% \part{Mechanistic Modelling and Investigating Empirical Methods}

\cleardoublepage 
%************************************************
\chapter{The Joint Evolution of Animal Movement and Competition Strategies}\label{ch:kleptomove}
\chaptermark{Competition and Movement}
%************************************************

{\noindent \textbf{Pratik R. Gupte}, Christoph F.G. Netz\textsuperscript{1}, and Franz J. Weissing\textsuperscript{1}}

    \medskip

    {\normalsize\headerfont{Co-author Affiliations}}
    
    \begin{enumerate}
        \item University of Groningen, The Netherlands.
    \end{enumerate}
    
    \medskip

    {\normalsize\headerfont{Funding}}

    European Research Council

    \bigskip

    {\noindent \large{$\Delta$}} A manuscript under review at \textit{The American Naturalist}.

% \subsection*{Data and Code}
% {
%     \small
%     Simulation model: github.com/pratikunterwegs/Kleptomove.\\ %and Zenodo: zenodo.org/record/4905476. 
%     \noindent Simulation data (DataverseNL): doi.org/10.34894/JFSC41.\\
%     \noindent Data analysis code: github.com/pratikunterwegs/klepto-move-evol.% and on Zenodo: doi.org/10.5281/zenodo.4904497.
% }

\clearpage
 
% 
\begin{refsection}
\section*{Introduction}
    Intraspecific competition is an important driver of population dynamics and the spatial distribution of organisms \citep{krebs1978}, and has two main types, `exploitation' and `interference'.
    In exploitation competition, individuals compete indirectly by depleting a common resource, while in interference competition, individuals compete directly by interacting with each other \citep{birch1957,case1974,keddy2001}.
    A special case of interference competition which is widespread among animal taxa is `kleptoparasitism', in which an individual steals a resource from its owner \citep{iyengar2008}.
    Since competition has an obvious spatial context, animals should account for the locations of competitors when deciding where to move \citep{nathan2008}.
    % Experimental work shows that indeed, competition, as well as the pre-emptive avoidance of competitive interactions, affects animal movement decisions in taxa as far apart as shorebirds \citep[][see also \citealt{rutten2010,bijleveld2012}]{goss-custard1980,vahl2005b,rutten2010a}, and fish \citep[][]{laskowski2013}.
    This is expected to have downstream effects on animal distributions across spatial scales, from resource patches \citep{fretwell1970}, to species distributions \citep{duckworth2007,schlagel2020}.
    Animal movement strategies are thus likely to be adaptive responses to landscapes of competition, with competitive strategies themselves being evolved responses to animal distributions.
    Empirical studies of this joint evolution are nearly impossible at large spatio-temporal scales.
    This makes models linking individual movement and competition strategies with population distributions necessary.

    Contemporary individual-to-population models of animal space-use \citep[reviewed in][]{deangelis2019} and competition, however, are only sufficient to represent very simple movement and prey-choice decisions.
    % , and struggle to adequately represent more complex systems of consumer-resource interactions.
    For example, models including the ideal free distribution \citep[IFD;][]{fretwell1970}, information-sharing models \citep[][]{giraldeau1999,folmer2012}, and producer-scrounger models \citep[][]{barnard1981,vickery1991,beauchamp2008}, often treat foraging competition in highly simplified ways.
    Most IFD models consider resource depletion unimportant or negligible \citep[continuous input models, see][]{tregenza1995, vandermeer1997}, make simplifying assumptions about interference competition, or even model an \textit{ad hoc} benefit of grouping \citep[e.g.][]{amano2006}.
    Meanwhile, producer-scrounger models primarily examine the benefits of choosing either a producer or scrounger strategy given local conditions, such as conspecific density \citep{vickery1991}, or the order of arrival on a patch \citep{beauchamp2008}.
    Overall, these models simplify the mechanisms by which competition decisions are made, and downplay spatial structure \citep[see also][]{holmgren1995, garay2020, spencer2018}.

    On the contrary, spatial structure is key to foraging (competition) decisions \citep{beauchamp2008}.
    % , making resource abundance and conspecific density of obvious importance to animal movement decisions \citep[e.g. step selection, \textit{sensu}][]{fortin2005, avgar2016}.
    How animals are assumed to integrate the costs (and potential benefits) of competition into their movement decisions has important consequences for theoretical expectations of population distributions \citep{vandermeer1997,hamilton2002,beauchamp2008}.
    In addition to short-term, ecological effects, competition also likely has evolutionary consequences for individual \textit{movement strategies}, setting up feedback loops between ecology and evolution.
    Modelling competition and movement decisions jointly is thus a major challenge.
    Some models take an entirely ecological view, assuming that individuals move or compete ideally, or according to fixed strategies \citep{vickery1991,holmgren1995,tregenza1995,amano2006}, but see \citep{hamilton2002}.
    Models that include evolutionary dynamics in movement \citep{dejager2011,dejager2020} and foraging competition strategies \citep{beauchamp2008,tania2012} are more plausible, but they too make arbitrary assumptions about the functional importance of environmental cues to individual decisions.

    Mechanistic, individual-based models are well suited to capturing the complexities of spatial structure, animal decision-making, and evolutionary dynamics \citep{guttal2010,kuijper2012,getz2015,getz2016,white2018,long2020,netz2021}; for conceptual underpinnings see \textcite{huston1988,mueller2011,deangelis2019}.
    Individual-based models can incorporate the often significant variation in movement and competition preferences found in populations, allowing individuals to make different decisions given similar cues \citep[][]{laskowski2013}.
    % Capturing these differences in models is likely key to better understanding how individual decisions scale to population- and community-level outcomes \citep{bolnick2011}.
    Individual-based models also force researchers to be explicit about their modelling assumptions, such as \textit{how exactly} competition affects fitness.
    Similarly, rather than taking a purely ecological approach and assuming individual differences \citep[e.g. in movement rules:][]{white2018}, allowing movement strategies to evolve in a competitive landscape can reveal whether individual variation emerges in plausible ecological scenarios \citep[as in][]{getz2015}.
    This allows the functional importance of environmental cues for movement \citep[see e.g.][]{scherer2020} and competition decisions in evolutionary models to be joint outcomes of selection, and lets different competition strategies to be associated with different movement strategies \citep[][]{getz2015}.

    Here, we present a spatially-explicit, mechanistic, individual-based model of intraspecific foraging competition, where movement and competition strategies jointly evolve on a resource landscape with discrete, depletable food items that need to be processed (`handled') before consumption.
    % As foraging and movement decisions are taken by individuals, we study the joint evolution of both types of decision-making by means of an individual-based simulation model.
    % Such models are well suited to modelling the ecology and evolution of complex behaviours .
    % This allows us to both focus more closely on the interplay of exploitation and interference competition, and to examine the feedback between movement and foraging behaviour at ecological and evolutionary timescales.
    % In our model, foraging individuals move .
    In our model, foragers make movement decisions using inherited, evolvable preferences for local ecological cues, such as resource and competitor densities; the combination of preferences for each cue forms individuals' movement strategy \citep[similar to relative step-selection:][]{fortin2005, avgar2016}.
    % After each move, individuals choose between two foraging strategies: whether to search for a food item or steal from another individual; 
    Lifetime resource consumption is our proxy for fitness; more successful individuals produce more offspring, transmitting their movement and foraging strategies to future generations (with small mutations).
    We consider three scenarios: in the first scenario, we examine only exploitation competition.
    In the second scenario, we introduce kleptoparasitic interference as an inherited strategy, fixed through an individual's life.
    In the third scenario, we model kleptoparasitism as a behavioural strategy conditioned on local environmental and social cues; the mechanism underlying this foraging choice is also inherited.

    Our model allows us to examine the evolution of individual movement strategies, population-level resource intake, and the spatial structure of the resource landscape.
    The model enables us to take ecological snapshots of consumer-resource dynamics (animal distributions, resource depletion, and competition) proceeding at evolutionary time-scales.
    Studying these snapshots allows us to check whether, when, and to what extent the spatial distribution of competitors resulting from the co-evolution of competition and movement strategies corresponds to standard IFD predictions.
    We investigate \textit{(1)} which movement strategies evolve in our three competition scenarios, \textit{(2)} whether movement strategies differ within and between competition strategies in our scenarios, and \textit{(3)} whether the emergent spatial distributions of consumers corresponds to `ideal free' expectations.

    {\printbibliography[heading=subbibliography]}

\end{refsection}



\cleardoublepage %************************************************
\chapter{Using a Mechanistic Model to Probe Statistical Methods in Animal Movement}\label{ch:patternprocess}
%************************************************

\noindent \textbf{Pratik R. Gupte} and Franz J. Weissing

\section*{Abstract}

\small{
    Movement ecologists have taken up the challenge of inferring animals' decision-making mechanisms in a spatial context from individual tracking data.
    The implicit assumption is that differences in the movement paths of animals reflect differences in individual decision-making mechanisms.
    However, animal movement takes place in complex and rapidly changing environments, where movement cues are not always available, and animals may differ along multiple axes of behaviour.
    Mechanistic, individual-based modelling of animal decision-making can help investigate whether differences in decision-making mechanisms actually translate into differences in movement paths, and the insights gained by parsing animal tracking data using contemporary statistical methods.
    Here, we examine the movement paths of agents from an evolutionary individual-based model of foraging competition, in which relatively simple movement rules are determined by evolved decision-making weights.
    To show how such a model can be used to investigate statistical methods, we explore a contemporary question in movement ecology: Can individual differences in movement decision-making mechanisms be detected from the emergent properties of the resulting movement paths?
    % First, we examine whether our model individuals' movement types differ in the structure of their movement paths.
    Using data on the movement of evolved model agents, we show how adopting a repeatability framework to quantify individual-differences in movement is sensitive to the evolutionary context in which movement rules evolve.
    We also find that repeatability analysis can yield very different conclusions depending on how individuals' behavioural types are accounted for.
    We also show that step-selection analysis can indicate differences between competition strategies, but rarely captures differences between movement types of the same competition strategy.
    Overall, using a plausible eco-evolutionary model of animal decision-making, we highlight some challenges in using contemporary statistical methods to infer individual differences in animals' decision-making mechanisms from positioning data.

    \medskip

    % \noindent {\large{\color{Maroon}$\Delta$}} Published in the \textit{Journal of Animal Ecology} as Gupte et al. (2021). A guide to pre-processing high throughput tracking data.
}

\clearpage


% \part{Long-Term and Large-Scale Studies in Animal Space-Use}



\cleardoublepage
%************************************************
\chapter{Rapid Evolution of Movement Strategies Following Pathogen Introduction}\label{ch:pathomove}
\chaptermark{Disease \& Movement}
%************************************************

{\noindent \sffamily\textbf{Pratik R. Gupte}, Gregory F. Albery\textsuperscript{1}, Jakob R.L. Gismann, and Franz J. Weissing}

\marginpar{
    \sffamily
    \textsuperscript{1} Wissenschaftskolleg zu Berlin, Germany.
}

\section*{Abstract}

\small{
    Animal social interactions always have a spatial context, and are the outcomes of evolved strategies that balance the costs and benefits of being sociable.
    We examine how animals balance the risk of pathogen transmission against the benefits of social information about resource patches, and the consequences for the emergent structure of animal social networks.
    We study a scenario in which an undetectable yet fitness-reducing infectious pathogen spills over into a population which has initially evolved movement rules in its absence.
    Pathogen spillover leads to a rapid evolutionary shift in animal social-movement strategies.
    The post-spillover strategy mix is controlled by a combination of landscape productivity and disease cost.
    Generally, animals adopt a dynamic social distancing approach, trading more movement (and less intake) for lower infection risk.
    Post-spillover populations are more widely dispersed over the landscape, and thus have less clustered social networks than their pre-spillover ancestors.
    Simple network epidemiological models show that diseases do indeed spread more slowly through pathogen-adapted animal societies.
    Our model suggests how the introduction of an infectious pathogen to a population rapidly changes social structure even when infections are undetectable, and how such events might make populations more resilient to future disease outbreaks.
    Overall, we offer both a general modelling framework and initial predictions for the evolutionary consequences of wildlife pathogen spillovers.

    \medskip

    % \noindent {\large{\color{Maroon}$\Delta$}} Published in the \textit{Journal of Animal Ecology} as Gupte et al. (2021). A guide to pre-processing high throughput tracking data.
}

\clearpage

% \include{Chapters/Chapter02}
%\addtocontents{toc}{\protect\clearpage} % <--- just debug stuff, ignore
% \include{Chapters/Chapter03}
%\include{multiToC} % <--- just debug stuff, ignore for your documents


\phantomsection
\addtocontents{toc}{\protect\vspace{\beforebibskip}}%
% \addcontentsline{toc}{chapter}{\tocEntry{\color{black}\itshape{General Discussion: Linking the Ecology and Evolution of Animal Movement with Mechanistic, Individual Based Models}}}%
\chapter{General Discussion: Linking the Ecology and Evolution of Animal Movement with Mechanistic, Individual Based Models}\label{ch:discussion}
\chaptermark{General Discussion}

{{Pratik R. Gupte}}

To be completed.

\newrefcontext[sorting=ynt]

\section*{Recapitulation of this thesis}

\section*{Case studies: why movement is key to ecological patterns}

\section*{Why include evolution in animal movement studies}

\subsection*{Importance of the individual-centric view}

\subsection*{How evolved movement strategies are different from random movement}

\subsection*{Evolution of complex traits can be very rapid}

\section*{Conceptual ingredients of eco-evolutionary individual-based models}

\section*{Practical aspects of implementing eco-evolutionary individual-based models}

\subsection*{Modelling time}

\subsection*{Modelling individuals}

\subsection*{Modelling landscapes}

The key question when modelling individuals' environment is whether the model has an implicit spatial component, or whether it is spatially explicit --- and indeed, whether the choice matters.

\subsubsection*{Spatially implicit or explicit?}

The cornerstone theoretical models of animals' movement ecology are not spatially explicit (and neither are they individual-based).
Neither the Ideal Free Distribution \citep{fretwell1970}, nor Optimal Foraging Theory \citep{charnov1976} have a specifically spatial component.
This is not to say that the two models do not have a clear spatial context, as both were inspired by specific scenarios --- the micro-scale distribution of New World warblers \citep[IFD:][]{fretwell1970}, and the movement of a forager between resource patches \citep{charnov1976}.
It is rather the case that given the very simple assumptions of these models, that explicitly modelling space does not add substantially to the points they are attempting to make.

When these assumptions are adjusted even slightly, taking space into account begins to make a difference to model outcomes [CITE HERE].
For example, in models where individuals differ in competitive ability, there is a distinct spatial separation of more and less competitive types of individuals.
Competitive animals occupy the most productive patches, where they have to contend with other competitive individuals, while less competitive individuals are displaced on to less productive patches [CITE HERE].
This simple theoretical prediction is recovered in multiple systems and at multiple scales --- for example, in different species of geese, large families which as social units have high competitive ability relative to single birds, tend to occupy the profitable leading edge of grazing flocks (small spatial scale), and also the most profitable wintering sites following annual autumn migration [CITE BLACK AND SOMEONE BARNIES, SOMEONE BRENT GEESE].

When modelling more complex links between individuals' ecological traits, an explicit spatial context becomes more important.
For example, in producer-scrounger models \citep[e.g.][]{beauchamp2008}, or in the kleptoparasite model of Chapter~\ref{ch:kleptomove}, individuals' ability to profit from their landscape is mechanistically linked to \emph{which other individuals} are in the vicinity.
Such models could still be represented in a spatially implicit way \citep{cressman2006,krivan2008,garay2015,garay2020}.
These models are primarily concerned with either micro-scale, or very long term animal distributions, and omit the movement process.

The point at which models tip over into requiring a spatially explicit context is when they consider movement as carrying a cost.
Movement costs may be direct, in the form of the energy required for displacement over a certain distance, and this requires a distance calculation which implies a specific spatial configuration of the landscape.
Yet movement costs may also be indirect, in the form of time or opportunity costs incurred when an individual moves between heterogeneously distributed resources --- this is how models in Chapters~\ref{ch:kleptomove} and \ref{ch:pathomove} are structured.
In this latter case, the spatial structure of the landscape has a strong influence on the structure of movement costs, on which more below.

\subsubsection*{Spatially implicit landscapes}

Early theoretical models in animal spatial ecology were almost all spatially implicit \citep{fretwell1970,charnov1976}.

mention movement between patches which is optimal --- movement that is random and taken from ideal gas theory --- random walks and animal spatial distributions resulting from random walks --- etc etc
end with, this is not what we really want to cover here.

cover some modern examples, lunn 2021 etc. mention also dinuzzo and griffen 2020.

\subsubsection*{Spatially explicit landscapes}

Spatially explicit representations of landscapes in animal movement models have become more common with the increase and proliferation of both computing power, and the programming skills of researchers in the field.

give early examples --- mcnamara and houston --- deangelis' work --- deangelis review 2019 --- spiegel 2017, white 2018, etc etc --- talk about the different models and what they showed

Spatially explicit landscapes can be challenging to implement, and involve multiple design decisions.
These decisions are influenced by the biology of the system being modelled, as well as theory and previous implementations in the field.
For example, many of the design decisions of Chapter~\ref{ch:kleptomove} are based on a previous implementation of a similar model in \citet{netz2021a}, and the spatial component of Chapter~\ref{ch:pathomove} is strongly influenced by the example models shown in \citet{spiegel2017}.
Overall, two major design decisions are whether to implement discrete or continuous space, and whether to have this space host discrete or continuous resource items.

\paragraph*{Discrete space implementations}

In a discrete space implementation, the landscape has a finite number of locations at which an individual can reside, and to and from which it can move.
Essentially, the landscape takes the form of a grid, and individuals can only occupy one tile or grid cell at a time.
Individuals usually only move between connected cells, with cells connected with other cells depending on the implementation.
Most grids are considered to be comprised of square cells, and these cells may be connected to those with which they share a side (4 cells, the von Neumann neighbourhood), or to those with which they share a corner (8 cells, the Moore neighbourhood).
This square grid-cell implementation has a background in theoretical spatial ecology as the basis for cellular automata models \citep[even those concerned with animals][]{jeltschf.1997}, and it is widely used in animal movement modelling.
For example, it is used in Chapter~\ref{ch:kleptomove}, as well as in \citet{white2018,dinuzzo2020,scherer2020} and \citet{netz2021a}.
However, more complex implementations, including movements between un-connected cells (a Moore neighbourhood $>$ 1), can also be implemented --- this is already done in Chapter~\ref{ch:kleptomove} for fleeing handlers.

[MENTION THEORETICAL BACKGROUND]

Implementing discrete space landscapes leads to the question of whether the cells and the landscape overall represent relative spatial scales correctly.
In Chapter~\ref{ch:kleptomove}, the landscape has 512 cells per side, and this is 25\% larger than the maximum number of cells a model individuals could reach in their lifetimes.
This large spatial scale, relative to individuals' per-timestep or per-lifetime movements, allows Chapter~\ref{ch:kleptomove} to present its gridded landscape as representing relatively small, yet not microscopic, resource patches.
The cells in discrete-space models are as as large or as small as they need to be from the perspective of biological theory.
In Chapter~\ref{ch:kleptomove}, the $512^2$ cells represent `patches' of intermediate size, and movement is limited to the Moore neighbourhood.
It is important to remember that in such conceptual models, the insights are more important than the exact matching of the model's various spatial and temporal scales.

Gridded landscapes with discrete cells are relatively easy to implement in a range of programming languages.
The grid cells are easily captured by a simple, two dimensional matrix data type in most cases, and such data types are easily constructed in low level languages, or available by default in scripting languages such as R \citep{rcoreteam2020}.
Multiple such matrices can be stacked into an array, with each layer representing some specific ecological aspect of the landscape, and the model.
While one layer may hold the number of individuals in each cell, another can easily represent the quantity of resource items available in the cell, and yet more layers can represent a range of other environmental conditions that impose costs or confer benefits upon individuals.

The advantage of gridded landscapes is that relations between cells, and between individuals and other landscape or model components, are strongly constrained.
Individuals usually cannot affect the localised dynamics of any cell but their own; for example, they cannot interact with competitors or resources on a neighbouring cell.
These limits on possible interactions have clear computational benefits in terms of reducing model run times.
To calculate the number of potential social interactions on a cell for instance, it is sufficient to determine the set of all individuals occupying the same cell.
This task can be sped up or made easier to code by assigning each cell a unique identifying index (with cell [0,0] being cell 1, cell [0,1] = 2), and so forth.
This bounds the maximum number of comparisons required to find all the potential interaction partners of an individual to $N - 1$, as a single integer (the focal individual's cell) is matched with the remaining $N - 1$ individuals in a population of $N$ individuals.
In contrast, the same task --- counting neighbours for potential interactions --- requires potentially much more computation in continuous space models (see below).

Individual-based models using gridded landscapes are common extensions to an increasingly popular method in the empirical study of animal movement, step-selection analysis (SSA) \citep{fortin2005,avgar2016,fieberg2021}.
The link between mechanistic movement modelling and step-selection analyses is covered below, as well as in Chapter~\ref{ch:patternprocess}.
The main reason that SSA --- which is agnostic to how space is implemented --- uses grid-based landscapes in its simulations of animal movement and habitat use, is that environmental data is available in the from of gridded raster layers.
Consequently, packages such as \emph{amt} analyse animals' movements in continuous space, but predict their habitat utilisation as discrete raster layers \citep{signer2017,signer2019}.

Simplified interactions are also the main disadvantage of discrete, gridded landscapes.
It is somewhat unrealistic to maintain that two individuals can be in adjacent grid cells and yet are unable to interact with each other.
This has implications for the local, small scale population and resource density, which may be lower when calculated separately per cell, than when calculated at the level of the group in continuous space.
On the other hand, determining group size and local density in continuous space models can also be challenging, as delineating groups (i.e., clusters of individuals) is a complex task --- discrete space models make this step easier to implement by having each group be restricted to a single cell.
One way of reconciling the natural objection to the `cell-as-group' idea is to maintain that the cells are of a spatial scale that allows for probabilistic interactions within the cell, but not with individuals in neighbouring cells.
However, there is no specific constraint when implementing a model that an individual should \emph{not} be able to interact with individuals from neighbouring cells.
Coding this latter case, though, discards the advantages of the gridded landscape (limited interactions), and researchers should probably consider implementing a continuous space landscape instead.

While this section has focussed on two dimensional landscapes, it is often worth considering a one dimensional grid as well.
One dimensional landscapes have a long history in animal spatial ecology, with an early appearance in the concept of the `selfish herd' [cite Hamilton 1969/7?].

cite some other recent papers using one dimensional landscapes --- explain why they worked or did not etc etc.

\paragraph*{Continuous space implementations}

Models with continuous space implementations allow individuals (and discrete resources, if any) to take fractional positions on the model landscape.
Examples of this approach include Chapter~\ref{ch:pathomove}, as well as \citet{spiegel2016,spiegel2017}, among many others [ADD MORE EXAMPLES].
This modelling approach aligns closely with how spatial positions are usually represented in empirical methods, as pairs of decimal coordinates.
Individuals' movement distances and angles are constrained only by the specifics of the implementation (see below).
[TALK ABOUT THEORETICAL BACKGROUND --- IDEAL GAS EQUATION ETC.?]

Continuous space implementations can handle the issue of spatial scales somewhat better than discrete space models.
For example, movement distances can be capped at a fraction of perception ranges, and both can be fractions of the total modelled landscape size.
Similarly, the distribution of resources is also easier to implement in terms of a density ($N$ resources per unit area), although this does depend on the spatial distributions of resources.
In Chapter~\ref{ch:pathomove}, the landscape has an area of $60^2$ units$^2$, with an individual perception and movement range of 1.0 units, and a resource density of 0.5 food items per unit area.
While the scaling of movement and perception in Chapter~\ref{ch:pathomove} is arbitrary (as the model aims primarily for insight), it is often easier to convince readers that continuous space models are `more realistic' than patch-based models with gridded landscapes.

In some models, individuals simply move about on a landscape without interacting with a large number of landscape components \citep[see some models in][]{spiegel2017}.
In such models it is sufficient to keep track of individuals' immediate coordinates, and these models are equally easily implemented as continuous space models or discrete space models.
Continuous space implementations may be better suited to models in which \textit{(1)} individuals can vary their movement distance and heading, or in which \textit{(2)} realistic local interactions are especially important.
For instance, Chapter~\ref{ch:pathomove} models two small-scale processes: exploitation competition, and pathogen transmission.
Both of these processes are strongly dependent on exactly how many, and which, individuals are within interaction or transmission range.
This makes the exact distance between individuals and features (such as resources) important, making the precise locations of individuals key to the model (although some disease-movement models adopt a patch approach; see \citealt{white2018,white2018b,jeltschf.1997}).
This, combined with the capacity to handle multiple spatial scales, is continuous space landscapes' main advantage over grid-based landscapes.

Continuous space landscapes can unlock model choices that would not be possible with a gridded landscape.
For instance, individual movement can be modelled in much more flexible ways: individuals need not choose among movement locations, but can instead choose a movement distance and a heading, and move to that location (see details below) \citep{spiegel2017,mueller2011}. [CITE MORE].
While it is also possible to combine this latter option with gridded landscapes, it is challenging to combine gridded and continuous space representations of related phenomena (landscape and movement).
Continuous space also allows for various implementations of resource landscape structure that is not always easy to fit into a gridded format, and is especially useful when trying to create an environment with patchily distributed resources (see below).
Algorithms that discretise heterogeneity in continuous space (such as for resources) into gridded values are abundant, however, and can achieve similar implementations for landscapes with discrete cells [CITE NLMR and NLMPy].

Continuous space simulations are also widely available as extensions to popular analyses in the empirical study of animal movement \citep{noonan2019,calabrese2016,calabrese2018,fleming2014,fleming2015,gurarie2017,gurarie2016}.
These implementations however, are not intended to take into account interactions with the environment, or with other individuals [BUT SEE THIS SINGLE AUTHOR PAPER - SPANISH NAME].
Rather, these methods simulate plausible animal movement paths after fitting a continuous time movement model to empirical animal position-tracking data.
Since these methods are not conditioned on environmental data, they are not constrained by the gridded structure of remote sensing data layers.

The wide flexibility of continuous space, and the many interactions that may occur among landscape components (depending on model parameterisation) offer opportunities for complex model structures, but this can easily lead to drawbacks.
First, because interactions between model components, such as individuals, are not neatly compartmentalised into cells, it is necessary to repeatedly determine whether any two interacting entities are within the appropriate range to do so.
This problem is exacerbated when there are multiple types of interaction, each with their own ranges.
For example, in Chapter~\ref{ch:pathomove}, individuals perceive food items, neighbours handling food items, and neighbours looking for food items.
When infected, they also unknowingly transmit a pathogen to neighbours within range.
This requires \emph{(1)} a distance calculation between each individual and each food item, and \emph{(2)} a distance calculation between all pairs of individuals --- this is repeated up to 500,000 times in the \emph{Pathomove} model.
These calculations represent a substantial computational challenge, especially for studies requiring multiple replicates and parameter combinations.

\paragraph*{Representing resources: Discrete and continuous implementations}

Many ecological models that explicitly or implicitly have a movement component include some representation of resources.
For example, classical individual-to-population models often assume a steady inflow of resources from the which individuals can benefit, but in such a way that the rate of inflow is not affected (continuous input models [CITE]).
The reason for this is that an input rate is analytically tractable, whereas realistic implementations of discrete, depleteable resource items are challenging to incorporate in mathematical models.
Individual-based simulation models, however, can represent substantially more complex consumer-resource interactions, and a broad range of mechanisms linking the two, from how individuals detect resources, to how they gain benefits from them, and how inter-individual associations are shaped by the presence or possession of resources (as in Chapters~\ref{ch:kleptomove} and \ref{ch:pathomove}).
In addition, simulation models can represent both large- and fine-scale variation in the number, quality, and type of resources available.

Spatial models representing continuous resource input usually define the resources at any given location as the outcome of a mathematical function of the location coordinates: $\text{resources} = f(x,y)$, in the case of two dimensional landscapes.
This approach works equally well for one dimensional landscapes, and these can implement, among others, a resource gradient (i.e., a linear function of landscape extent), or a series of resource peaks and troughs (e.g. using a sine function).
In the latter case, the largest positive values might represent areas with lots of resources; the lowest negative values, on the other hand, may represent areas of the landscape that are a net drain on individual resources.
Two options to deal with this are to set negative values to zero, creating intermittent resource vacuums, or to scale the function's values to an acceptable range, such as [0, 1].
The two implementations may lead to different outcomes for the evolution of perception and movement distances.
For example, in the first case with intermittent resource vacuums, individuals initialised in these areas will have few or no resource cues upon which to base their decisions.
In the latter case, the resource gradient is maintained, as at the lowest resource values, individuals can use cues in surrounding areas to move to better locations.

The same issues apply to two dimensional landscapes; however, the functions used to generate two dimensional spatial heterogeneity can be much more complex.
Functions that were developed in the visual effects realm to create realistic variation on two-dimensional surfaces, such as \textit{Noise} \citep{perlin}, Simplex Noise \citep{simplexnoise}, and OpenSimplexNoise \citep{opensimplexnoise}, are good candidates for functions that can create smooth spatial variation on two dimensional landscapes.
Noise has been used to represent [CITE WHERE NOISE HAS BEEN USED].

Advantages and disadvantages of continuous resource implementations --- advantages --- follow the assumptions of continuous input models, relatively easy to implement other assumptions such as interference competition and variations thereof \citep{tregenza1995,vandermeer1997} --- easy to scale intake as a function of the number of neighbours, or as a function of another layer of the landscape --- Noise and related implementations are almost indefinitely extensible in space --- multiple pre-built implementations are available\\
--- disadvantages --- difficult to implement resource depletion, which is key to exploitation competition --- including complex noise functions in simulation models is not easy and requires substantial programming skills --- Noise and other two dimensional functions look realistic but should be remembered as abstract representations of natural patterns.

Models which represent resources as discrete items (as in Chapters~\ref{ch:kleptomove} and \ref{ch:pathomove}) typically treat each item similarly to individuals --- in the parlance of programming, as an object with certain attributes and functions.
How are item impemented --- how should they be implemented --- what are the options --- how are they different from continuous input/resource models --- what can be done with discrete resources that cannot be done with continuous input --- e.g. exploitation competition due to item depletion --- handling time and regeneration time for easy implementation of theoretical costs (e.g. costs of grouping near resources) --- rgeneration time allows landscape heterogeneity even without specific spatial structure, and allows individuals to shape their landscape --- what are the advantages --- what are the disadvantages [MAYBE split into two paras]

Combinations of discrete and continuous resource landscapes --- briefly mention with examples, recommend against as they are difficult to combine.

\subsection*{Modelling animal movement}

What is the classical way of modelling animal movement --- movements between patches --- what are the assumptions (ideal, free) --- what are some other spatially explicit ways in which this can be implemented --- random walks etc. this section will focus on spatially explicit landscapes and not implicit ones

Modelling random walks --- how to implement them --- parameters of the random walk --- combining two different random walks (e.g. levy flight and brownian motion) into a multi-stage movement process shaped by the environment --- including evolutionary dynamics in random walk parameters as attributes --- including evolutionary dynamics of other parts of the movement process, such as the explore-exploit tradeoff as a switch between random walks, or as the duration of a specific random walk phase (speigel 2017 paper)

Modelling directed movement --- the optimal movement approach \citep{scherer2020} --- why is this not always suitable: assumes the importance of various cues to animals, and that the most important cues are captured in the model --- alternative: individuals use id-speicfic attributes to assign suitability and move to the most suitable location --- cite own chapters, cite also white et al and amt modelling of utilisation distribution

\subsection*{Modelling ecological interactions}

\subsection*{Translating ecological outcomes into evolutionary consequences}

\section*{Relating individual-based models with empirical approaches in movement ecology}

\section*{Conclusion: Where are we now, and where is focus necessary?}

\newrefcontext[sorting=nyt]
\section*{Literature Cited}
\printbibliography[title={Literature~Cited},heading=none]
\end{refsection}

% %*******************************************************
% Acknowledgments
%*******************************************************
% \pdfbookmark[1]{Acknowledgments}{acknowledgments}
\addchap{Acknowledgments}\label{ch:ack}
% \chaptermark{Reflections and Acknowledgments}

% \begin{flushright}{\slshape
%     We have seen that computer programming is an art, \\
%     because it applies accumulated knowledge to the world, \\
%     because it requires skill and ingenuity, and especially \\
%     because it produces objects of beauty.} \\ \medskip
%     --- \defcitealias{knuth:1974}{Donald E. Knuth}\citetalias{knuth:1974} \citep{knuth:1974}
% \end{flushright}

\bigskip

\begingroup
\let\clearpage\relax
\let\cleardoublepage\relax
\let\cleardoublepage\relax

I would like to write down a few acknowledgments for people who have had a substantial role over the course of my PhD.

First I'd like to acknowledge the role of my supervisory team, starting with those who were initially on the team in 2018, but are no longer.
Ton Groothuis and Theunis Piersma were initially involved, but as my PhD veered further away from its planned form, they understandably dropped out.
I'm grateful that neither of them insisted that I should stick to the original structure of the PhD, and regret not working on more applied projects with Theunis.
Allert Bijleveld, then a newly minted PI from the NIOZ, was also initially involved and I was supposed to collaborate closely with one of his other PhD students.
Unfortunately, I found him to be an unreliable and stubborn supervisor with little big picture thinking to his work.
I had three very frustrating years trying to work with him and his lab, before I cut myself off from the collaboration; I'm glad that he too recently withdrew as a co-promotor.

Second and much more positively, Ran Nathan joined my supervisory team very late, just a few days before I turned in my thesis.
Ran had actually been involved with my work since 2020, and we had met at conferences in 2018 and 2019.
It was entirely by chance that I presented my work in methods development at an online workshop hosted by his lab; this prompted Ran to suggest the writing of what eventually became Chapter~\ref{ch:preprocessing} of this thesis.
Later, he invited me to be part of a large author team working on a review in \textit{Science}, which was very exciting.
I began collaborating with Ran directly in the summer of 2021, to replace the projects I closed when I stopped working with Allert.
This project, now Chapter~\ref{ch:holeybirds}, went better than I could have hoped, and that led me to suggest including him as a supervisor and indeed as a `promotor', which is a special role in the Dutch academic system.
I found Ran, even when he was not yet my supervisor, to be warm, supportive and helpful, with exactly the sort of big picture thinking I had been lacking in relation to animal tracking.
I'm grateful that he accepted a role as one of my promotors, and indeed that he initially agreed to work with me; I look forward to working with him in the future.

Finally, I must acknowledge Franjo Weissing who has been my main PhD supervisor, and is now my first promotor and the sole survivor of the original supervisory team.
I had actually been in touch, indirectly at least, with Franjo since 2015, when I had turned down an offer to join the MEME master's programme that he established.
I have mixed feelings about accepting this position, to which I did not actually apply.
On the one hand I am neither a theoretician nor an evolutionary biologist and might have done better sticking to ecological data; on the other hand I picked up an interesting set of skills here that I might not have done in a data-focused lab.
I also thought Franjo's lab was too conceptually dispersed for there to be much peer support for my work.
This effect was exacerbated by my being one of the first people in the newly reconstituted lab, and also because I initially worked with data (through the NIOZ collaboration) more than my colleagues.
I appreciated that Franjo allowed me to continue working on movement data, even though at the time, there was no apparent link with any theoretical work.
In general, I'm grateful that he allowed me to do pretty much as I wanted with my PhD, and for my part I tried to make sure that I was as productive as possible.
I found Franjo to be a supportive and caring supervisor, and I'm grateful that he decided to hire me; I hope our work together will help establish his approach to eco-evolutionary theory.

\medskip

I've had the good fortune to have worked with two ambitious and driven people; both are early career researchers who are doing extremely well for themselves scientifically, and I'm sure both will go very far.

Vijay Ramesh, then at Columbia University (and now at Cornell's Lab of Ornithology) reached out to me in late 2018.
He was trying to use data from \textit{eBird} to study birds in the southern Western Ghats of India, and asked whether I would help him with some spatial data analysis.
I joined him in a six month project that ran for three and a half years, and was recently published in \textit{Ecography}.
I used this project as a testbed for new techniques and ideas, such as using Python for some spatial computations; and this was my first professional foray out of R programming.
Last year, in 2021, we began work on another exciting project to study changes in bird distributions in the Nilgiri Hills over the past 150 years using museum specimen data.

Greg Albery is a recent collaborator, but someone whose work I've held in high regard for some time now.
In early 2021, Greg tried to recruit me to his supervisor's lab at Georgetown University; this was a massive confidence boost just as my collaboration with the NIOZ was failing.
I was sounded out to work on building social networks from animal movement data, with a potential expansion into examining disease transmission; this gave me the idea for Chapter~\ref{ch:pathomove}, on which he's a co-author.
Along the way, I picked up the skills in coding simulations for animal movement and pathogen transmission that were no doubt invaluable in landing me my current job.

I'm grateful that both Vijay and Greg reached out to me when they did, and I look forward to working with them in the years to come.

I've had the help of many other collaborators at all levels in academia, who appear as authors on some of the chapters in this thesis, and on manuscripts yet to come: Orr Spiegel, Sivan Toledo, Yosef Kiat, Yoav Barton, Ulrike Schl{\"a}gel, Johannes Signer, Mark Adams, Rebecca Rimbach, Mridula Paul, Morgan Tingley, VV Robin, Ruth de Fries, and Amy Sweeny.

\paragraph*{TRES and Surrounds}

Joining TRES in mid-2018, I found a department that, pre-pandemic, was very similar to the Centre for Ecological Sciences I was coming from --- full of intelligent people, and most importantly, social.
I will freely admit to not finding the majority of the work in this department interesting, and that was a function of the diversity, or divergence, of topics among and within labs.
Yet the general gregariousness of the PhD students who were my colleagues more than made up for this.
I now think departments with better social than professional ties are probably healthier workplaces.

A number of people here 

I'm quite sure my life would be very different without Josh Lambert.
I would never have tried a number of things I now enjoy without Josh suggesting them, and often, accompanying me: squash, long-distance cycling, programming in Julia, considering the UK to live and work.
I also made a more serious change, of which Josh convinced me: to eventually leave the academic track.
I'm immensely pleased that I was able to find a cluster-hire at the London School of Hygiene and Tropical Medicine, through which we were both offered positions.
Josh is both very smart and grounded, and he's the first and last person I go to for advice --- I look forward to having him around.

\endgroup


% %*******************************************************
% Publications
%*******************************************************
% \pdfbookmark[1]{Acknowledgments}{acknowledgments}
% \phantomsection
% \addtocontents{toc}{\protect\vspace{\beforebibskip}}%
% \addcontentsline{toc}{chapter}{\tocEntry{Publications Related to this Thesis}}%
\addchap{About the Author}\label{ch:pubs}
\chaptermark{About the Author}

Pratik R. Gupte was born in India in 1993.
After schooling in Hyderabad and an undergraduate degree in zoology from St. Xavier's College, Mumbai in 2014, he worked on field projects in southern India, Ladakh, and in South Africa.
In 2017 he received a master's degree from the University of Kiel, as part of the International Master's in Applied Ecology, a programme spread over France, Portugal, Germany, and Ecuador.
His master's thesis on families of wintering Arctic geese saw fieldwork in the Russian Arctic and the Netherlands.
In 2018, he began his PhD in Franjo Weissing's lab at the University of Groningen.
Pratik is broadly intereste in the spatial ecology of animals, especially birds.
Recognising the unsustainability of the current model of academic science, he left the academic career track, and is now a research software engineer at the London School of Hygiene and Tropical Medicine, where he develops epidemiological models to inform policy responses to disease outbreaks.

\subsection*{Publications}

\begin{refsection}
    \small
    \toggletrue{bbx:allnames}
    \nocite{gupte2021a,gupte2022c,gupte2022d,thaker2019,nathan2022,netz2022,ramesh2022,
        bijleveld2021,rimbach2022un,gupte2019} % is local to to the enclosing refsection
    \printbibliography[heading=none]
\end{refsection}

\subsection*{Code and Data}

\begin{refsection}
    \small
    \nocite{gupte2022,gupte2022e,gupte2022b,gupte2021b,gupte2020a,gupte2022f,netz2021b,netz2022a,netz2021b} % is local to to the enclosing refsection
    \printbibliography[heading=none]
\end{refsection}

{ \begin{center} \barfont{-.-} \end{center} }

% \emph{Attention}: This requires a separate run of \texttt{bibtex} for your \texttt{refsection}, \eg, \texttt{ClassicThesis1-blx} for this file. You might also use \texttt{biber} as the backend for \texttt{biblatex}. See also \url{http://tex.stackexchange.com/questions/128196/problem-with-refsection}.


% ********************************************************************
% Backmatter
%*******************************************************
\appendix
%\renewcommand{\thechapter}{\alph{chapter}}
% \cleardoublepage
% \part{Appendix}

%********************************************************************
% Other Stuff in the Back
%*******************************************************
%\cleardoublepage%********************************************************************
% Bibliography
%*******************************************************
% work-around to have small caps also here in the headline
% https://tex.stackexchange.com/questions/188126/wrong-header-in-bibliography-classicthesis
% Thanks to Enrico Gregorio
% \defbibheading{bibintoc}[\bibname]{%
%   \phantomsection
%   \manualmark
%   \markboth{\spacedlowsmallcaps{#1}}{\spacedlowsmallcaps{#1}}%
%   \addtocontents{toc}{\protect\vspace{\beforebibskip}}%
%   \addcontentsline{toc}{chapter}{\tocEntry{#1}}%
%   \chapter*{#1}%
% }
% \printbibliography[heading=bibintoc]

% BIBLIOGRAPHY
\begingroup
  \begin{bibenv}
  \newrefcontext[sorting=nyt]
  \addtocontents{toc}{\protect\vspace{\beforebibskip}}%
  \addchap{Literature Cited in this Thesis}
  \markboth{\color{gray}\small\scshape\rmfamily{Bibliography}}{\color{gray}\small\scshape\rmfamily{Bibliography}}
  \printbibliography[heading=none]
  \end{bibenv}
\endgroup

% Old version, will be removed later
% work-around to have small caps also here in the headline
%\manualmark
%\markboth{\spacedlowsmallcaps{\bibname}}{\spacedlowsmallcaps{\bibname}} % work-around to have small caps also
%\phantomsection
%\refstepcounter{dummy}
%\addtocontents{toc}{\protect\vspace{\beforebibskip}} % to have the bib a bit from the rest in the toc
%\addcontentsline{toc}{chapter}{\tocEntry{\bibname}}
%\label{app:bibliography}
%\printbibliography

% \cleardoublepage\include{FrontBackmatter/Declaration}

% \cleardoublepage%*******************************************************
% Publications
%*******************************************************
% \pdfbookmark[1]{Acknowledgments}{acknowledgments}
% \phantomsection
% \addtocontents{toc}{\protect\vspace{\beforebibskip}}%
% \addcontentsline{toc}{chapter}{\tocEntry{Publications Related to this Thesis}}%
\addchap{About the Author}\label{ch:pubs}
\chaptermark{About the Author}

Pratik R. Gupte was born in India in 1993.
After schooling in Hyderabad and an undergraduate degree in zoology from St. Xavier's College, Mumbai in 2014, he worked on field projects in southern India, Ladakh, and in South Africa.
In 2017 he received a master's degree from the University of Kiel, as part of the International Master's in Applied Ecology, a programme spread over France, Portugal, Germany, and Ecuador.
His master's thesis on families of wintering Arctic geese saw fieldwork in the Russian Arctic and the Netherlands.
In 2018, he began his PhD in Franjo Weissing's lab at the University of Groningen.
Pratik is broadly intereste in the spatial ecology of animals, especially birds.
Recognising the unsustainability of the current model of academic science, he left the academic career track, and is now a research software engineer at the London School of Hygiene and Tropical Medicine, where he develops epidemiological models to inform policy responses to disease outbreaks.

\subsection*{Publications}

\begin{refsection}
    \small
    \toggletrue{bbx:allnames}
    \nocite{gupte2021a,gupte2022c,gupte2022d,thaker2019,nathan2022,netz2022,ramesh2022,
        bijleveld2021,rimbach2022un,gupte2019} % is local to to the enclosing refsection
    \printbibliography[heading=none]
\end{refsection}

\subsection*{Code and Data}

\begin{refsection}
    \small
    \nocite{gupte2022,gupte2022e,gupte2022b,gupte2021b,gupte2020a,gupte2022f,netz2021b,netz2022a,netz2021b} % is local to to the enclosing refsection
    \printbibliography[heading=none]
\end{refsection}

{ \begin{center} \barfont{-.-} \end{center} }

% \emph{Attention}: This requires a separate run of \texttt{bibtex} for your \texttt{refsection}, \eg, \texttt{ClassicThesis1-blx} for this file. You might also use \texttt{biber} as the backend for \texttt{biblatex}. See also \url{http://tex.stackexchange.com/questions/128196/problem-with-refsection}.

% \cleardoublepage%*******************************************************
% Acknowledgments
%*******************************************************
% \pdfbookmark[1]{Acknowledgments}{acknowledgments}
\addchap{Acknowledgments}\label{ch:ack}
% \chaptermark{Reflections and Acknowledgments}

% \begin{flushright}{\slshape
%     We have seen that computer programming is an art, \\
%     because it applies accumulated knowledge to the world, \\
%     because it requires skill and ingenuity, and especially \\
%     because it produces objects of beauty.} \\ \medskip
%     --- \defcitealias{knuth:1974}{Donald E. Knuth}\citetalias{knuth:1974} \citep{knuth:1974}
% \end{flushright}

\bigskip

\begingroup
\let\clearpage\relax
\let\cleardoublepage\relax
\let\cleardoublepage\relax

I would like to write down a few acknowledgments for people who have had a substantial role over the course of my PhD.

First I'd like to acknowledge the role of my supervisory team, starting with those who were initially on the team in 2018, but are no longer.
Ton Groothuis and Theunis Piersma were initially involved, but as my PhD veered further away from its planned form, they understandably dropped out.
I'm grateful that neither of them insisted that I should stick to the original structure of the PhD, and regret not working on more applied projects with Theunis.
Allert Bijleveld, then a newly minted PI from the NIOZ, was also initially involved and I was supposed to collaborate closely with one of his other PhD students.
Unfortunately, I found him to be an unreliable and stubborn supervisor with little big picture thinking to his work.
I had three very frustrating years trying to work with him and his lab, before I cut myself off from the collaboration; I'm glad that he too recently withdrew as a co-promotor.

Second and much more positively, Ran Nathan joined my supervisory team very late, just a few days before I turned in my thesis.
Ran had actually been involved with my work since 2020, and we had met at conferences in 2018 and 2019.
It was entirely by chance that I presented my work in methods development at an online workshop hosted by his lab; this prompted Ran to suggest the writing of what eventually became Chapter~\ref{ch:preprocessing} of this thesis.
Later, he invited me to be part of a large author team working on a review in \textit{Science}, which was very exciting.
I began collaborating with Ran directly in the summer of 2021, to replace the projects I closed when I stopped working with Allert.
This project, now Chapter~\ref{ch:holeybirds}, went better than I could have hoped, and that led me to suggest including him as a supervisor and indeed as a `promotor', which is a special role in the Dutch academic system.
I found Ran, even when he was not yet my supervisor, to be warm, supportive and helpful, with exactly the sort of big picture thinking I had been lacking in relation to animal tracking.
I'm grateful that he accepted a role as one of my promotors, and indeed that he initially agreed to work with me; I look forward to working with him in the future.

Finally, I must acknowledge Franjo Weissing who has been my main PhD supervisor, and is now my first promotor and the sole survivor of the original supervisory team.
I had actually been in touch, indirectly at least, with Franjo since 2015, when I had turned down an offer to join the MEME master's programme that he established.
I have mixed feelings about accepting this position, to which I did not actually apply.
On the one hand I am neither a theoretician nor an evolutionary biologist and might have done better sticking to ecological data; on the other hand I picked up an interesting set of skills here that I might not have done in a data-focused lab.
I also thought Franjo's lab was too conceptually dispersed for there to be much peer support for my work.
This effect was exacerbated by my being one of the first people in the newly reconstituted lab, and also because I initially worked with data (through the NIOZ collaboration) more than my colleagues.
I appreciated that Franjo allowed me to continue working on movement data, even though at the time, there was no apparent link with any theoretical work.
In general, I'm grateful that he allowed me to do pretty much as I wanted with my PhD, and for my part I tried to make sure that I was as productive as possible.
I found Franjo to be a supportive and caring supervisor, and I'm grateful that he decided to hire me; I hope our work together will help establish his approach to eco-evolutionary theory.

\medskip

I've had the good fortune to have worked with two ambitious and driven people; both are early career researchers who are doing extremely well for themselves scientifically, and I'm sure both will go very far.

Vijay Ramesh, then at Columbia University (and now at Cornell's Lab of Ornithology) reached out to me in late 2018.
He was trying to use data from \textit{eBird} to study birds in the southern Western Ghats of India, and asked whether I would help him with some spatial data analysis.
I joined him in a six month project that ran for three and a half years, and was recently published in \textit{Ecography}.
I used this project as a testbed for new techniques and ideas, such as using Python for some spatial computations; and this was my first professional foray out of R programming.
Last year, in 2021, we began work on another exciting project to study changes in bird distributions in the Nilgiri Hills over the past 150 years using museum specimen data.

Greg Albery is a recent collaborator, but someone whose work I've held in high regard for some time now.
In early 2021, Greg tried to recruit me to his supervisor's lab at Georgetown University; this was a massive confidence boost just as my collaboration with the NIOZ was failing.
I was sounded out to work on building social networks from animal movement data, with a potential expansion into examining disease transmission; this gave me the idea for Chapter~\ref{ch:pathomove}, on which he's a co-author.
Along the way, I picked up the skills in coding simulations for animal movement and pathogen transmission that were no doubt invaluable in landing me my current job.

I'm grateful that both Vijay and Greg reached out to me when they did, and I look forward to working with them in the years to come.

I've had the help of many other collaborators at all levels in academia, who appear as authors on some of the chapters in this thesis, and on manuscripts yet to come: Orr Spiegel, Sivan Toledo, Yosef Kiat, Yoav Barton, Ulrike Schl{\"a}gel, Johannes Signer, Mark Adams, Rebecca Rimbach, Mridula Paul, Morgan Tingley, VV Robin, Ruth de Fries, and Amy Sweeny.

\paragraph*{TRES and Surrounds}

Joining TRES in mid-2018, I found a department that, pre-pandemic, was very similar to the Centre for Ecological Sciences I was coming from --- full of intelligent people, and most importantly, social.
I will freely admit to not finding the majority of the work in this department interesting, and that was a function of the diversity, or divergence, of topics among and within labs.
Yet the general gregariousness of the PhD students who were my colleagues more than made up for this.
I now think departments with better social than professional ties are probably healthier workplaces.

A number of people here 

I'm quite sure my life would be very different without Josh Lambert.
I would never have tried a number of things I now enjoy without Josh suggesting them, and often, accompanying me: squash, long-distance cycling, programming in Julia, considering the UK to live and work.
I also made a more serious change, of which Josh convinced me: to eventually leave the academic track.
I'm immensely pleased that I was able to find a cluster-hire at the London School of Hygiene and Tropical Medicine, through which we were both offered positions.
Josh is both very smart and grounded, and he's the first and last person I go to for advice --- I look forward to having him around.

\endgroup

% ********************************************************************
% Game Over: Restore, Restart, or Quit?
%*******************************************************
\end{document}
% ********************************************************************
