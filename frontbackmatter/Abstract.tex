%*******************************************************
% Abstract
%*******************************************************
% \phantomsection
% % \addtocontents{toc}{\protect\vspace{\beforebibskip}}%
% \addcontentsline{toc}{chapter}{\tocEntry{Summary}}}%
\addchap{Thesis Abstract}\label{ch:abstract}
\chaptermark{Abstract}

\lettrine{M}{ovement} is a fundamental phenomenon in the natural world, and active movement in response to environmental drivers is key to animal ecology.
Individuals' positions in a landscape determine what they perceive, and with which other animals they associate and how.
These ecological interactions feed back into decisions on \textit{where to go next}.
Such fine-scale, individual-level decisions, made by each individual in a population or species, whether alone or in concert with others, scale up over time and space to affect large-scale ecological phenomena such as species distributions and interactions.
Over the past twenty years, the field of \textit{movement ecology}, driven by rapid advances in animal tracking technology, has revealed fascinating connections between animal movement and ecological drivers that were previously impossible to measure.

Now, movement ecology is advancing on new frontiers.
This thesis is an episodic, personal account of developments on two of these frontiers, with which I have been involved: \textit{(i)} the study of animal movements using massive datasets, and \textit{(ii)} the exploration of the evolutionary causes and consequences of animal movement, using computer simulation models.
These advances have been made possible by methodological innovations --- such as new technologies for animal tracking, or by adopting new approaches --- such as evolutionary individual-based simulations, and reflections on these methods are woven into this work. 
Organising this thesis into two parts, one for each of the themes above, I have tried to gain and present new insight, but also to lay the groundwork for future developments.

{\scshape~Chapter~\ref{ch:introduction}} provides a broad introduction to these two themes.
I set out my view for how we could better understand the ecology and evolution of animal movement and spatial distributions by using mechanistic, individual-based simulation models.
In brief, I cover why animals' foraging dynamics, rather than rare or sporadic events such as natal dispersal or annual migration, are especially suited to understanding the evolutionary causes and consequences of movement as an adaptive behaviour.


\medskip

In {\scshape~\textbf{Part I}}, I look at our advances in studying the fine-scale movement decisions of animals using \textit{big data} collected with new \textit{high-throughput} animal tracking systems.
A useful primer to high-throughput tracking, with which I was involved, but which is not presented in this thesis, is a recent review in \textit{Science}, \citetitle{nathan2022} \footfullcite{nathan2022}.

\medskip

{\scshape~Chapter~\ref{ch:preprocessing}} lays out some practical aspects of dealing with the massive spatial datasets that are generated by \textit{high-throughput animal tracking} systems, which can track the movement of hundreds of individuals at a very high spatio-temporal resolution (a few metres' accuracy, and a few seconds' interval).
I cover data cleaning, aggregation, and first principles-based segmentation-clustering, as well as how to implement these methods in reproducible and efficient ways.
Adopting computational best-practices from software development and other big-data fields such as genomics is the way forward for robust methods development and reproducible data-processing in movement ecology.

\medskip

In {\scshape~Interlude~\ref{box:mapping}}, I include some thoughts on both the technical and aesthetic aspects of visualising animal movements, and show how I applied them while making a map that won the British Ecological Society Movement Ecology Special Interest Group's \emph{Mapping Animal Movements} competition in 2021.

\medskip

{\scshape~Chapter~\ref{ch:holeybirds}} combines high-throughput animal movement data with high-resolution data on the fine-scale, three-dimensional spatial structure of the biotic and abiotic environment.
Specifically, I take a mechanistic look at the proximate drivers of the movement and habitat selection of moulting birds.
I show how simple mechanistic aspects of a landscape --- the visibility of one location from another, interacts with the physical determinants of movement --- the surface area of birds' wings, to shape how individuals use their environment.
A viewshed analysis approach that computes \textit{fearscapes} --- areas of high visibility --- reveals that animal movements are a joint outcome of individuals' current physiological state (i.e., the condition of their wings), and individuals' likely perception of landscape risk, in terms of whether they could potentially be seen by \textit{other individuals}.

\medskip

In {\scshape~\textbf{Part II}}, I look at how we can tackle questions about the evolutionary causes and consequences of animal movement strategies, using mechanistic \textit{individual-based models} of movement decisions.
These models, I suggest, are key to understanding the evolutionary ecology of movement, because they can incorporate both essential ecological detail as well as allowing evolutionary dynamics that are impossible to measure in natural systems.

\medskip

In {\scshape~Interlude~\ref{box:demos}}, I demonstrate how to implement conceptual models that link the ecology and evolution of animals' fine-scale movement strategies.
Using a prototype model that draws on principles laid out in the \emph{Introduction}, I show a simple prototype of the mechanistic models used in this part of the thesis.
I show how such models could lead to qualitatively and quantitatively different outcomes from those that would be obtained by structuring models according to classical assumptions --- such as random or optimal movement --- from evolutionary ecology.

\medskip

{\scshape~Chapter~\ref{ch:kleptomove}} presents a mechanistic, individual-based model of the joint evolution of animal movement and foraging competition strategies.
This is the first fully fleshed out study using the class of models I advocate in the \emph{Introduction}.
In this model, individuals' movement and foraging decisions depend on local environmental cues, and simultaneously, individual foraging decisions leads to a restructuring of the cues available in the environment.
I show how movement strategies evolve to match individuals' competitive context as well as the availability of information on the resource landscape.
Substantial individual variation is evolved in movement strategies among foragers, and furthermore, I find tight correlations between evolved movement and foraging strategies under some conditions.
Modelling animal movement decisions in an eco-evolutionary context can help define the envelope of potential outcomes under different ecological scenarios in which there are complex feedback loops between individual movement and environmental cues.

\medskip

In {\scshape~Interlude~\ref{box:details}}, I include a brief comment about the importance of attention to detail when building individual-based simulation models.
That this comment had to be written in response to published work shows how it can actually be quite challenging to interpret and implement even a classic theoretical model (the Ideal Free Distribution; `IFD') in terms of computational methods --- specifically, as an individual-based simulation model.
Such implementations therefore require both skill and care while coding, as well as a firm understanding of the biological processes (perception and movement) underlying phenomena such as the IFD.

\medskip

{\scshape~Chapter~\ref{ch:pathomove}} looks at the evolution of animal movement strategies following the introduction of an infectious, chronic pathogen, and examines how animals balance the benefits of social information on resource distributions, against the risks of pathogen transmission, and the consequences of this evolutionary change for animal sociality.
I show that introducing a pathogen to a population that has evolved to use social information leads to very rapid changes in movement strategies; this leads to cascading outcomes including more movement overall, fewer individual associations, lower intake, but also reduced pathogen transmission compared to non-adapted ancestral populations.
Mechanistically modelling the introduction and spread of a novel infectious pathogen, a scenario of increasing global concern, can help to predict the direct and indirect consequences for individual-level outcomes, as well as impacts on the spatial-social organisation of animal societies.

\medskip

{\scshape~Chapter~\ref{ch:patternprocess}} uses simulated movement data from individuals in Chapter~\ref{ch:kleptomove} to validate popular methods in the study of empirical animal movement data: repeatability analysis, and step-selection functions.
% In the model, individuals move according to their evolved strategies which are fully known; I examine what the results of repeatability analysis and step-selection analysis can tell us about these methods.
I show that individual differences in movement strategies do not always result in differences in movement paths, and consequently, statistical tools including repeatability analysis and step-selection analysis, may not be able to detect often substantial underlying variation in animals' movement strategies.
Applying statistical methods common in movement ecology to simulated movement data where the mechanisms controlling movement are known, can help reveal ecological and evolutionary scenarios which may confound these methods, enabling more precise inferences from tracking data.

\medskip

Finally, in {\scshape~Chapter~\ref{ch:discussion}}, I reflect on the findings in this thesis, and suggest how an energetics approach could be used to estimate some of the fitness consequences of animal movement.

% Add final sentence.

\endgroup

{ \begin{center} \barfont{-.-} \end{center} }

\vfill

\cleardoublepage
