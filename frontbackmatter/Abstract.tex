%*******************************************************
% Abstract
%*******************************************************
% \phantomsection
% % \addtocontents{toc}{\protect\vspace{\beforebibskip}}%
% \addcontentsline{toc}{chapter}{\tocEntry{Summary}}}%
\addchap{Thesis Abstract}\label{ch:abstract}
\chaptermark{Abstract}

% \begin{center}
%     \emph{Spatial is special.}\\
%     \medskip
%     -- \small{A common maxim in data science}
% \end{center}

\lettrine{M}{ovement} is a fundamental phenomenon in the natural world, and active movement in response to environmental drivers is key to animal ecology.
Individuals animals' positions in a landscape determine what they perceive, and with which other animals they associate and how.
These ecological interactions feed back into decisions on \textit{where to go next}.
Such mall-scale, individual-level decisions, made by each individual in a population or species, whether alone or in concert with others, scale up over time and space to affect large-scale ecological phenomena such as species distributions and interactions.
Over the past twenty years, the field of \textit{movement ecology}, driven by rapid advances in animal tracking technology, has revealed fascinating connections between animal movement and ecological drivers that were previously impossible to measure.

\noindent Now, movement ecology is advancing on new frontiers.%, which are covered in Chapter~\ref{ch:introduction}.
% In Part~\ref{part:eco} of this thesis, showing how there is strong, distinct need for the study of animal movement to make \textit{methodological advances}, borrowing from other fields that have undergone a `big-data revolution'.
% One frontier, the \textit{evolution of movement}, is especially challenging to explore, and I lay out the need for why the evolutionary feedbacks in animal movement ecology should be considered.

\noindent This thesis is an episodic, personal account of developments on two of these frontiers, with which I have been involved: \textit{(i)} the study of animal movements and distributions, using massive new datasets, and \textit{(ii)} the exploration of the evolutionary causes and consequences of animal movement, using computer simulation models.
These advances have been made possible by methodological innovations --- such as new technologies for animal tracking, or by adopting new approaches --- such individual-based simulations, and reflections on these methods are woven through this work. 
Organising this thesis into two parts, one for each of the themes above, in each chapter, I have tried to gain and present new insight, but also to lay the groundwork for future developments.
{\scshape~Chapter~\ref{ch:introduction}} provides a broad introduction to these two themes.

\medskip

\noindent In {\scshape~\textbf{Part I}}, I look at our advances in studying the movement and distribution of animals --- primarily birds --- using \textit{big data}, either collected with new \textit{high-throughput} animal tracking systems, or through the efforts of hundreds of \textit{citizen scientists} logging their observations.
A useful primer to high-throughput tracking, with which I was involved, but is not presented in this thesis, is \textcite{nathan2022}.

\medskip

{\scshape~Chapter~\ref{ch:preprocessing}} lays out some practical aspects of dealing with the massive spatial datasets that are generated by \textit{high-throughput animal tracking} systems, which can track the movement of multiple individuals at very high spatio-temporal resolution (a few metres' accuracy, and a few seconds' interval).
I cover data cleaning, aggregation, and first principles-based segmentation-clustering, as well as how to implement these methods in reproducible and efficient ways.
Adopting computational best-practices from software development and other big-data fields such as genomics is the way forward for robust methods development and reproducible data-processing in movement ecology.

{\scshape~Chapter~\ref{ch:holeybirds}} combines high-throughput animal movement data with high-resolution data on the fine-scale, three-dimensional spatial structure of the biotic and abiotic environment, and takes a mechanistic look at the proximate drivers of the movement and habitat selection of moulting birds.
I show how simple mechanistic aspects of a landscape --- the visibility of one location from another, interacts with the physical determinants of movement --- the surface area of birds' wings, to shape how individuals use their environment.
A viewshed analysis approach that computes \textit{fearscapes} --- areas of high visibility --- reveals that animal movements are a joint outcome of individuals' current physiological state (i.e., the condition of their wings), and individuals' likely perception of landscape risk, in terms of whether they could potentially be seen by \textit{other individuals}.

{\scshape~Chapter~\ref{ch:hillybirds}} leverages a qualitatively different big data source, the citizen science effort \textit{eBird}, to examine how landcover and climate drive the large-scale distributions of birds in tropical mountains.
Together with collaborators, I show how \textit{eBird} data from southern India can be processed to fit occupancy models, and we test the climatic variability hypothesis that species endemic to climatically stable areas have lower tolerances for climate seasonality.
Combining large amounts of low-resolution citizen science data with appropriate environmental predictors can help expand studies of animal space use to large spatial scales, help study areas where tracking technology is not widely used or usable, and help to generate broad predictions that can be examined in detail with individual tracking.

In two separate {\scshape~Interludes}, I include some thoughts on visualising animal movements, and a call for real environmental governance in India.

\medskip

\noindent In {\scshape~\textbf{Part II}}, I look at how we can tackle questions about the evolutionary causes and consequences of animal movement strategies, using mechanistic \textit{individual-based models} of movement decisions.
These models, I suggest, are key to understanding the evolutionary ecology of movement, because they can incorporate both essential ecological detail as well as allowing evolutionary dynamics that are impossible to measure in natural systems.

\medskip

{\scshape~Chapter~\ref{ch:kleptomove}} presents a mechanistic, individual-based model of the joint evolution of animal movement and foraging competition strategies, and in these models, individuals' decisions not only depend on environmental cues, but foraging leads to individuals structuring their local landscapes.
I show how movement strategies evolve to match individuals' competitive context as well as the availability of information on the resource landscape; there is substantial individual variation evolved in movement strategies among foragers, as well as tight correlations between evolved movement and foraging strategies.
Modelling animal movement decisions in an eco-evolutionary context can help define the envelope of potential outcomes under different ecological scenarios in which there are complex feedback loops between individual movement and environmental cues.

{\scshape~Chapter~\ref{ch:patternprocess}} uses movement data from individuals in Chapter~\ref{ch:kleptomove} moving according to their evolved strategies, in which there is substantial variation, and examines whether these individual differences can be detected by mainstream statistical methods in animal ecology.
I show that individual differences in movement strategies do not always result in differences in movement paths, and consequently, statistical tools including repeatability analysis and step-selection analysis, may not be able to detect often substantial underlying variation in animals' movement strategies.
Applying statistical methods common in movement ecology to simulated movement data where the mechanisms controlling movement are known, can help reveal ecological and evolutionary scenarios which may confound these methods, enabling more precise inferences from tracking data.

{\scshape~Chapter~\ref{ch:pathomove}} looks at the evolution of animal movement strategies following the introduction of an infectious, chronic pathogen, and examines how animals balance the benefits of social information on resource distributions, against the risks of pathogen transmission, and the consequences of this evolutionary change for animal sociality.
I show that introducing a pathogen to a population that has evolved to use social information leads to very rapid changes in movement strategies; this leads to cascading outcomes including more movement overall, fewer individual associations, lower intake, but also reduced pathogen transmission compared to non-adapted ancestral populations.
Mechanistically modelling the introduction and spread of a novel infectious pathogen, a scenario of increasing global concern, can help to predict the direct and indirect consequences for individual-level outcomes, as well as impacts on the spatial-social organisation of animal societies.

In an {\scshape~Interlude}, I include a brief comment about the importance of attention to detail when building individual-based simulation models.

\medskip

Finally, in the {\scshape~Synthesis (Chapter~\ref{ch:discussion})}, I reflect on the thesis chapters.
I set out my view for how we could better understand the ecology and evolution of animal movement and spatial distributions by using mechanistic, individual-based simulation models.
I support this view with a simple worked-out example model that draws on principles from Chapters~\ref{ch:kleptomove} and~\ref{ch:pathomove}.
I demonstrate how the mechanistic modelling used in this thesis leads to qualitatively and quantitatively different outcomes from those that would be obtained by structuring models according to classical assumptions --- such as random or optimal movement --- from evolutionary ecology.

% \noindent Add final sentence.

\endgroup

{ \begin{center} \barfont{-.-} \end{center} }

\vfill

\cleardoublepage
