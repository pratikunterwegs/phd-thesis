
\addtocontents{toc}{\protect\vspace{\beforebibskip}}
\addchap{Nederlandse Samenvatting}\label{ch:dutchabstract}
\chaptermark{Nederlandse Samenvatting}

\lettrine{B}{eweging} is een fundamenteel proces in de natuur en het ontrafelen van de oorzaken en consequenties van bewegingspatronen is een belangrijke doelstelling van de dierecologie. De posities van individuele dieren in een landschap bepalen wat ze kunnen waarnemen en met welke andere dieren ze interacties kunnen aangaan. De uitkomsten van deze waarnemingen en interacties hebben invloed op beslissingen over waar ze vervolgens heengaan. Dergelijke individuele beslissingen vormen de basis voor grootschalige ecologische verschijnselen, zoals de verspreiding en de ecologische interacties van een soort. In de afgelopen twintig jaar heeft het vakgebied bewegingsecologie, dankzij snelle ontwikkelingen op het gebied van trackingtechnologie, fascinerende en nooit eerder waargenomen relaties tussen de bewegingspatronen van dieren en ecologische processen aangetoond.

Nog steeds is de bewegingsecologie een van de meest dynamische gebieden in de biologie. Dit proefschrift is een episodisch, persoonlijk verslag van twee ontwikkelingen, waarbij ik betrokken ben geweest: \textit{(i)} de ontwikkeling van statistische methoden om uit de enorme tracking datasets ecologisch betekenisvolle inzichten te verkrijgen over de oorzaken en consequenties van bewegingspatronen; en \textit{(ii)} de ontwikkeling van simulatiemodellen om de evolutie van bewegingsstrategie{\"e}n beter te begrijpen. Beide ontwikkelingen zijn mogelijk gemaakt door methodologische innovaties die in dit proefschrift nader worden beschreven en nader worden uitgewerkt. Mijn proefschrift bevat twee delen, die die overeenkomen met de twee bovengenoemde thema's.

Hoofdstuk~\ref{ch:introduction} biedt een uitgebreide introductie op de twee thema's. Ik leg uit hoe mechanistische, individu-gebaseerde simulatiemodellen kunnen bijdragen aan een beter begrip vande ecologie en evolutie van de dierlijke bewegings- en verdelingspatronen.. Onder meer leg ik uit dat vaak voorkomende bewegingspatronen, zoals de verplaatsingen van dieren tijdens het foerageren, even goed (of zelfs beter) geschikt zijn om de ecologische en evolutionaire oorzaken van dierlijke beweging te begrijpen als grootschalige maar sporadische gebeurtenissen, zoals geboorteverspreiding of jaarlijkse migratie.

In Deel~\ref{part:eco} beschouw ik het probleem hoe de verplaatsingsstrategie{\"e}n met behulp van de big data van trackingsystemen ontrafeld kunnen worden. Ik was als coauteur betrokken bij een recent overzichtsartikel in Science (niet opgenomen in mijn proefschrift; zie de lijst met publicaties) dat een handige inleiding tot de problematiek geeft.

Hoofdstuk~\ref{ch:preprocessing} beschrijft een aantal praktische aspecten van het werken met de enorme ruimtelijke datasets die worden gegenereerd door high-throughput trackingsystemen, die de verplaatsingen van honderden dieren met een zeer hoge spatio-temporele resolutie kunnen tracken (met een nauwkeurigheid van een paar meter en met een interval van een paar seconden). Ik behandel het opschonen, aggregeren, segmenteren en clusteren van data, en bespreekmanieren om deze methoden op een reproduceerbare en effici{\"e}nte manier te implementeren. Hierbij maak ik gebruik van recentelijk ontwikkelde methoden uit andere vakgebieden (softwareontwikkeling en andere big data vakgebieden zoals genomics). Het ontwikkelen van robuuste en reproduceerbare methoden voor dataverwerking is volgens mij een hoeksteen van de bewegingsecologie van de toekomst.

In Intermezzo~\ref{box:mapping} illustreer ik aan de hand van een voorbeeld zowel de technische als de esthetische aspecten van het visualiseren van bewegingsdata. Dit resulteerde in een kaart die in 2021 de Mapping Animal Movements-competitie van de British Ecological Society heeft gewonnen.

In Hoofdstuk~\ref{ch:holeybirds} laat ik zien hoe een combinatie van fijnmazige bewegingsdata en de analyse van `gezichtsvelden' (wat een individueel dier daadwerkelijk kan zien vanuit zijn locatie) nieuwe inzichten geeft in de verplaatsingsstrategie{\"e}n en habitatselectie van ruiende (en dus kwetsbare) vogels. De analyse laat zien dat de beslissingen van ruiende vogels voornamelijk worden bepaald door de toestand van hun verenkleed (dat bepaald hoe makkelijk zij kunnen vluchten) en de vraag of en in hoeverre potenti{\"e}le bestemmingen zichtbaar zijn voor predatoren.

In Deel~\ref{part:evo} beschrijf ik hoe we met behulp van individu-gebaseerde modellen inzichten kunnen verkrijgen in de evolutie van bewegingsstrategie{\"e}n en de ecologische consequenties van deze strategie{\"e}n. In Intermezzo~\ref{box:demos} illustreer ik aan de hand van een eenvoudig voorbeeld hoe dit soort conceptuele modellen kunnen worden geïmplementeerd. Ook laat ik zien dat de evolutionaire en ecologische voorspellingen van dit soort modellen substantieel kunnen verschillen (kwantitatief en kwalitatief) van de uitkomsten van wiskundige modellen.

In Hoofdstuk~\ref{ch:kleptomove} bestudeer ik een model voor de evolutie van bewegingsstrategie{\"e}n in het verband van voedselcompetitie. Dit is het eerste, volledig uitgewerkte onderzoek dat gebruikmaakt van het type modellen dat ik in de Inleiding bepleit. In dit model zijn de verplaatsings- en foerageerbeslissingen van individuele dieren afhankelijk van lokale omgevingssignalen (zoals de dichtheid van voedsel en de aanwezigheid van soortgenoten) en leidt evolutie tot een steeds betere aanpassing van deze beslissingen aan de competitieve context. De simulaties laten zien dat verschillende competitiestrategie{\"e}n geassocieerd raken met verschillende bewegingsstrategie{\"e}n. Dit leidt tot een verdeling van de concurrerende individuen over de ruimte die sterk afwijkt van de voorspellingen van klassieke modellen. Voor elke competitievorm volgen de bewegingsstrategie{\"e}n een bepaald patroon, maar binnen dit patroon bestaat een grote diversiteit aan bewegingsstrategie{\"e}n. Dit heeft belangrijke consequenties, want vanwege deze diversiteit kunnen bewegingsstrategie{\"e}n zeer snel evolueren als omgevingsfactoren (zoals de voedselverdeling) veranderen.

Intermezzo~\ref{box:details} is een (gepubliceerd) commentaar op een individu-gebaseerde simulatiestudie die niet voldoende rekening houdt met de door mij voorgestelde principes voor het modelleren van beweging en competitie. We laten zien dat kleine onnauwkeurigheden en fouten bij de implementatie van een simulatiemodel grote consequenties voor het systeemgedrag kunnen hebben.

Hoofdstuk~\ref{ch:pathomove} behandelt de evolutie van verplaatsingsstrategie{\"e}n na de introductie van een besmettelijk pathogeen. Via evolutie moet een nieuwe balans worden gevonden tussen de voordelen van sociale contacten (het verkrijgen van informatie over potenti{\"e}le voedselbronnen) en de nieuw ontstane  risico's van dit soort contacten (de overdracht van het pathogeen). Ik laat zien dat de evolutie verrassend snel verloopt en grote consequenties heeft voor de structuur van sociale netwerken en de foerageereffici{\"e}ntie. Een mechanistische modellering van de introductie en verspreiding van een nieuw, besmettelijk pathogeen, een scenario dat wereldwijd tot steeds grotere zorgen leidt, kan dus helpen om de directe en indirecte gevolgen op individueel niveau te voorspellen, evenals de gevolgen voor de ruimtelijk-sociale organisatie van dierengemeenschappen.

Hoofdstuk~\ref{ch:patternprocess} combineert de methoden van deel I en deel II van mijn proefschrift. Met behulp van de individuele bewegingspatronen in de simulaties in hoofdstuk~\ref{ch:kleptomove} valideer ik twee populaire statistische methoden in de bewegingsecologie: herhaalbaarheidsanalyse en de analyse van stapselectiefuncties. Ik laat zien dat de in hoofdstuk~\ref{ch:kleptomove} gevonden aanzienlijke individuele verschillen in verplaatsingsstrategie{\"e}n door deze methoden vaak niet gedetecteerd worden. Deze studie laat zien dat simulatiegegevens zeer nuttig kunnen zijn om de mogelijkheden en beperkingen van statistische tools in kaart te brengen.

In Hoofdstuk~\ref{ch:discussion} kijk ik ten slotte terug op de bevindingen van dit proefschrift en stel ik voor hoe een energetica-aanpak zou kunnen worden gebruikt om sommige van de fitnessgevolgen van verplaatsingen van dieren in te schatten.

{ \begin{center} \barfont{-.-} \end{center} }
