
\addtocontents{toc}{\protect\vspace{\beforebibskip}}
\addchap{Nederlandse Samenvatting}\label{ch:dutchabstract}
\chaptermark{Nederlandse Samenvatting}

\lettrine{V}{oor} dieren is het verplaatsen van plaats naar plaats een belangrijk onderdeel van hun ecologie. 
De locaties van individuen in hun omgeving hebben invloed op wat ze waarnemen en hoe ze omgaan met andere dieren. Deze interacties hebben vervolgens invloed op de volgende stap. Deze fijnmazige beslissingen gebeuren op individueel niveau. Deze keuzes kunnen door veel individuen in een populatie of soort worden gemaakt, alleen of in overleg met anderen. Ze schalen in tijd en ruimte op en be{\"i}nvloeden grootschalige fenomenen zoals de verspreiding van soorten en ecologische interacties. In de afgelopen twintig jaar heeft het gebied van ruimtelijke ecologie van dieren snelle vorderingen gemaakt in de technologie voor het volgen van dieren. Deze revolutie op het gebied van het volgen van dieren heeft fascinerende aspecten van de ruimtelijke ecologie van dieren aan het licht gebracht die voorheen onmogelijk te onderzoeken waren.

Nu gaat de ruimtelijke ecologie van dieren verder op nieuwe onderzoeksterreinen. Dit proefschrift is een enigszins episodische, persoonlijke vertelling van ontwikkelingen op twee van deze grenzen. De eerste is de studie van dierenbewegingen met behulp van enorme trackingdatasets. De tweede is het onderzoek naar de evolutionaire oorzaken en gevolgen van dierbeslissingen in een ruimtelijke context, met behulp van computersimulatiemodellen. Deze ontwikkelingen zijn mogelijk gemaakt door methodologische innovaties --- zoals nieuwe technologie{\"e}n voor het volgen van dieren, of door nieuwe methoden toe te passen --- zoals simulaties op basis van de evolutie van individuele beslissingen. Reflecties op deze methoden zijn verweven in dit werk. Ik heb dit proefschrift opgedeeld in twee delen, {\'e}{\'e}n voor elk van de bovenstaande thema's, en ik heb geprobeerd om nieuwe inzichten te verkrijgen en te presenteren, maar ook om de basis te leggen voor toekomstige ontwikkelingen.

Hoofdstuk \ref{ch:introduction} geeft een brede inleiding op deze twee thema's. Ik zette mijn visie uiteen over hoe we de ecologie en evolutie van dierenbewegingen en ruimtelijke distributies beter zouden kunnen begrijpen door gebruik te maken van mechanistische, individueel gebaseerde simulatiemodellen (`individual based models' IBM). Ik leg uit waarom de foerageer beslissingen van dieren, in plaats van zeldzame of sporadische gebeurtenissen zoals geboorteverspreiding of jaarlijkse migratie, vooral geschikt zijn om de evolutionaire oorzaken en gevolgen van beweging als adaptief gedrag te begrijpen.

In deel \ref{part:eco} kijk ik naar onze vorderingen bij het bestuderen van de fijnmazige bewegingsbeslissingen van dieren met behulp van `big data' verzameld met nieuwe `high-throughput' diervolgsystemen.

Hoofdstuk \ref{ch:preprocessing} beschrijft enkele praktische aspecten van het omgaan met de enorme ruimtelijke datasets die worden gegenereerd door high-throughput diervolgsystemen, die de beweging van honderden individuen kunnen volgen met een zeer hoge ruimte-tijdresolutie: een nauwkeurigheid van enkele meters, en een interval van enkele seconden. Ik behandel gegevensopschoning, aggregatie en op eerste principes gebaseerde segmentatie-clustering, evenals hoe deze methoden op reproduceerbare en effici{\"e}nte manieren kunnen worden ge{\"i}mplementeerd. Het toepassen van computationele best-practices uit softwareontwikkeling en andere gebieden die te maken hebben met enorme datasets, zoals genomica, is de weg vooruit voor de ontwikkeling van robuuste methoden en reproduceerbare gegevensverwerking in bewegingsecologie.

In Interlude A neem ik enkele gedachten op over zowel de technische als de esthetische aspecten van het visualiseren van dierenbewegingen, en laat ik zien hoe ik deze heb toegepast tijdens het maken van een kaart die in 2021 de wedstrijd `Mapping Animal Movements' van de British Ecological Society Movement Ecology Special Interest Group had gewonnen.

Hoofdstuk \ref{ch:holeybirds} combineert locatiegegevens van vogels met hoge doorvoer met gegevens met een hoge resolutie over de fijnschalige, driedimensionale ruimtelijke structuur
In het bijzonder kijk ik mechanisch naar de naaste drijfveren van de
Ik laat zien hoe eenvoudige mechanistische aspecten van een landschap --- de zichtbaarheid van de ene locatie vanaf de andere, interageert met de fysieke determinanten van beweging --- het oppervlak van de vleugels van vogels, om
Een viewshed-analysebenadering die `fearscapes' berekent --- gebieden met een hoge zichtbaarheid --- onthult dat bewegingen van dieren een gezamenlijk resultaat zijn van de huidige fysiologische toestand van individuen (d.w.z. de conditie van hun vleugels), en de waarschijnlijke perceptie van vogels van het landschap risico; dat wil zeggen, of ze mogelijk kunnen worden gezien door andere dieren zoals roofdieren.

In deel \ref{part:evo} bekijk ik hoe we vragen over de evolutionaire oorzaken en gevolgen van dierbewegingsstrategie{\"e}n kunnen aanpakken, met behulp van mechanistische modellen die zijn gebaseerd op de ruimtelijke beslissingen van meerdere individuele dieren. Ik stel voor dat deze modellen de sleutel zijn tot het begrijpen van de evolutionaire ecologie van beweging, omdat ze zowel essenti{\"e}le ecologische details kunnen bevatten als evolutionaire dynamiek mogelijk maken die onmogelijk te onderzoeken is in systemen in de echte wereld.

In Interlude B laat ik zien hoe conceptuele modellen kunnen worden ge{\"i}mplementeerd die de ecologie en evolutie van de fijnmazige ruimtelijke strategie{\"e}n van dieren met elkaar verbinden. Op basis van principes die in de inleiding zijn uiteengezet, laat ik een eenvoudig prototype zien van de mechanistische modellen die in dit deel van het proefschrift worden gebruikt. Ik laat zien hoe dergelijke modellen kunnen leiden tot kwalitatief en kwantitatief andere resultaten dan die verkregen zouden worden door modellen te structureren volgens klassieke veronderstellingen --- zoals willekeurige of optimale beweging --- uit de evolutionaire ecologie.

Hoofdstuk \ref{ch:kleptomove} presenteert een mechanisch, individueel gebaseerd model van de gezamenlijke evolutie van dierbewegingen en foerageercompetitiestrategie{\"e}n.
Dit is de eerste volledig uitgewerkte studie waarbij gebruik wordt gemaakt van de klasse van modellen die ik in de inleiding bepleit. In dit model zijn de verplaatsings- en foerageerbeslissingen van dieren afhankelijk van lokale omgevingssignalen, en tegelijkertijd leiden individuele foerageerbeslissingen tot een herstructurering van de signalen die beschikbaar zijn in de omgeving. Ik laat zien hoe bewegingsstrategie{\"e}n evolueren om te passen bij de competitieve context van dieren en en de beschikbaarheid van informatie over de verdeling van voedsel over het landschap. Aanzienlijke variatie tussen individuen evolueert in bewegingsstrategie{\"e}n onder verzamelaars, en bovendien vind ik onder bepaalde omstandigheden nauwe correlaties tussen ge{\"e}volueerde beweging en foerageerstrategie{\"e}n. Het modelleren van beslissingen over het verplaatsen van dieren in een eco-evolutionaire context kan helpen bij het defini{\"e}ren van de omhullende van mogelijke resultaten onder verschillende ecologische scenario's waarin er complexe feedbacklussen zijn tussen individuele bewegingen en signalen uit de omgeving.

In Interlude C neem ik een korte opmerking op over het belang van aandacht voor detail bij het bouwen van op individuen gebaseerde simulatiemodellen. Dat dit commentaar moest worden geschreven als reactie op gepubliceerd werk, laat zien hoe het eigenlijk een hele uitdaging kan zijn om zelfs een klassiek theoretisch model (de ``Ideal Free Distribution''; IFD) te interpreteren en te implementeren in termen van rekenmethoden --- specifiek, zoals een individueel gebaseerd simulatiemodel. Dergelijke implementaties vereisen daarom zowel vaardigheid als zorg bij het coderen, evenals een goed begrip van de biologische processen (perceptie en beweging) die ten grondslag liggen aan fenomenen zoals de IFD.

Hoofdstuk \ref{ch:pathomove} kijkt naar de evolutie van strategie{\"e}n voor het verplaatsen van dieren na de introductie van een besmettelijke, chronische ziekteverwekker, en onderzoekt hoe dieren de voordelen van sociale informatie over de verdeling van hulpbronnen afwegen tegen de risico's van overdracht van ziekteverwekkers, en de gevolgen van deze evolutionaire verandering voor dieren socialiteit. Ik laat zien dat het introduceren van een ziekteverwekker in een populatie die is ge{\"e}volueerd om sociale informatie te gebruiken, leidt tot zeer snelle veranderingen in bewegingsstrategie{\"e}n; dit leidt tot trapsgewijze resultaten, waaronder meer beweging in het algemeen, minder individuele associaties, lagere inname, maar ook verminderde overdracht van pathogenen in vergelijking met niet-aangepaste voorouderlijke populaties. Mechanistisch modelleren van de introductie en verspreiding van een nieuwe infectieuze ziekteverwekker, een scenario van toenemende wereldwijde bezorgdheid, kan helpen om de directe en indirecte gevolgen voor de resultaten op individueel niveau te voorspellen, evenals de effecten op de ruimtelijk-sociale organisatie van dierengemeenschappen.

Hoofdstuk \ref{ch:patternprocess} gebruikt gesimuleerde bewegingsgegevens van individuen in Hoofdstuk \ref{ch:kleptomove} om populaire methoden te valideren in de studie van empirische dierbewegingsgegevens: herhaalbaarheidsanalyse en stapselectiefuncties (SSF). In het model bewegen individuen zich volgens hun ontwikkelde strategie{\"e}n die volledig bekend zijn. Ik onderzoek wat de resultaten van herhaalbaarheidsanalyse en stapselectieanalyse ons kunnen vertellen over deze methoden. Ik laat zien dat individuele verschillen in bewegingsstrategie{\"e}n niet altijd resulteren in verschillen in bewegingspaden, en bijgevolg zijn statistische hulpmiddelen, waaronder herhaalbaarheidsanalyse en stapselectieanalyse, mogelijk niet in staat om vaak substanti{\"e}le onderliggende variatie in bewegingsstrategie{\"e}n van dieren te detecteren.
Het toepassen van statistische methoden die gebruikelijk zijn in bewegingsecologie op gesimuleerde bewegingsgegevens waarvan de mechanismen die beweging beheersen bekend zijn, kan helpen ecologische en evolutionaire scenario's te onthullen die deze methoden kunnen verwarren, waardoor nauwkeurigere gevolgtrekkingen uit trackinggegevens mogelijk zijn.

Tot slot, in Hoofdstuk \ref{ch:discussion}, reflecteer ik op de bevindingen in dit proefschrift en stel ik voor hoe een energetische benadering kan worden gebruikt om enkele van de fitnessgevolgen van dierenbewegingen in te schatten.
