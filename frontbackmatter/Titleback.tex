\thispagestyle{empty}

\hfill

\vfill

\noindent {\scshape{COLOPHON}}

\noindent The research presented in this thesis was carried out at the Department for Theoretical Research in Evolutionary Life Sciences (TRES), at the Groningen Institute for Evolutionary Life Sciences (GELIFES), at the University of Groningen's Faculty of Science and Engineering (FSE).
This research was made possible by an \emph{Adaptive Life} grant from GELIFES; part of the work in this thesis was funded by the European Research Council (ERC Advanced Grant No. 789240).
Support for this research was provided primarily by the University of Groningen; other institutions whose support made parts of this work possible are acknowledged within.
The production of this thesis was partly funded by GELIFES, the FSE, and the University of Groningen.

\medskip

\noindent This document was typeset using the \emph{classicthesis} \LaTeX~style, developed by Andr\'e Miede and Ivo Pletikosić.
% This style was inspired by Robert Bringhurst's seminal book on typography \emph{The Elements of Typographic Style}.
The text is mostly set in \emph{Timepos Text} from the New Zealand-based Klim Type Foundry, with some headings in \emph{Plex Serif} from the Dutch type foundry Bold Monday. Friedrich Althausen's \emph{Vollkorn} is used for chapter breaks.

\bigskip

\noindent\finalVersionString

\noindent\myName. \textit{\myTitle.}% \mySubtitle, %\myDegree,
\\
\noindent \textcopyright\ \today
