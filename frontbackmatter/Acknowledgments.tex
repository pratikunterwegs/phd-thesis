%*******************************************************
% Acknowledgments
%*******************************************************
% \pdfbookmark[1]{Acknowledgments}{acknowledgments}
\addchap{Acknowledgments}\label{ch:ack}
% \chaptermark{Reflections and Acknowledgments}

% \begin{flushright}{\slshape
%     We have seen that computer programming is an art, 
%     because it applies accumulated knowledge to the world, 
%     because it requires skill and ingenuity, and especially 
%     because it produces objects of beauty.}  \medskip
%     --- \defcitealias{knuth:1974}{Donald E. Knuth}\citetalias{knuth:1974} \citep{knuth:1974}
% \end{flushright}

\bigskip

\begingroup
\let\clearpage\relax
\let\cleardoublepage\relax
\let\cleardoublepage\relax

I would like to write down a few acknowledgments for people who have had a substantial role over the course of my PhD.

First I'd like to acknowledge the role of my supervisory team, starting with those who were initially on the team in 2018.
Ton Groothuis and Theunis Piersma from the University of Groningen were initially involved, but as my PhD veered further away from its planned form, they understandably dropped out.
I'm grateful that neither of them insisted that I should stick to the original structure of the PhD, and regret not working on more applied projects with Theunis.
Allert Bijleveld, a new PI from the NIOZ, was also initially involved and I was supposed to collaborate closely with one of his other PhD students.
I had three very frustrating years trying to keep this collaboration going before I distanced myself from it, and I think that was among the best decisions I took in my PhD.

Much more positively, Ran Nathan from the Hebrew University of Jerusalem officially joined my supervisory team in early 2022, but he had been involved with my work since 2020.
% , and we had met at conferences in 2018 (Biomove in Potsdam) and 2019 (the Gordon conference in Lucca, Italy).
It was entirely by chance that I presented my work in methods development at an online workshop hosted by his lab; this prompted Ran to suggest the writing of what eventually became Chapter~\ref{ch:preprocessing} of this thesis.
% Later, he invited me to be part of a large author team working on a review in \textit{Science}, which was very exciting.
I began collaborating with Ran directly in the summer of 2021, to replace the projects I closed when I stopped working with Allert.
This project, now Chapter~\ref{ch:holeybirds}, went better than I could have hoped, and that led me to suggest including him as a supervisor and indeed as a `promotor', which is a special role in the Dutch academic system.
I found Ran, even when he was not yet my supervisor, to be warm, supportive and helpful, with lots of big picture thinking in relation to animal tracking.
I'm grateful that he accepted a role as one of my promotors, and indeed that he initially agreed to work with me; I look forward to working with him in the future.

Finally, I must acknowledge Franjo Weissing who has been my main PhD supervisor, and is now my first promotor and the sole survivor of the original supervisory team.
I had actually been in touch, indirectly at least, with Franjo since 2015, when I had turned down an offer to join the MEME master's programme that he established.
% I have mixed feelings about accepting this position, to which I did not actually apply.
When I began my PhD, I was neither a theoretician nor an evolutionary biologist (and I still would not maintain I am), so I'm glad that Franjo decided to take a chance with a relatively inexperienced candidate.
% I thought Franjo's lab was too conceptually dispersed for there to be much peer support for my work.
% This effect was exacerbated by my being one of the first people in the newly reconstituted lab, and also because I initially worked more with data (through the NIOZ collaboration) than my colleagues.
I appreciated that Franjo allowed me to continue working on movement data, even though at the time, there was no apparent link with any theoretical work.
I'm grateful that he allowed me to do pretty much as I wanted with my PhD, and for my part I tried to make sure that I was as productive as possible.
I found Franjo to be a supportive and caring supervisor, and I'm grateful that he decided to hire me; I hope our work together will help establish his approach to eco-evolutionary theory.

I also want to say that none of these wranglings in the PhD process --- adding and removing supervisors, dealing with finances, or organising many other matters --- would have been possible without Ingeborg Jansen, who runs the Theoretical Biology secretariat and by extension the department; I am very grateful to her for her help over the years.

\medskip

I've had the good fortune to have worked with two ambitious and driven peers over the past few years.

\noindent Vijay Ramesh at Columbia University reached out to me in late 2018.
Vijay was trying to use data from \textit{eBird} to study birds in the southern Western Ghats of India, and asked whether I would help him with some spatial data analysis.
I joined him in a six month project that actually ran for three and a half years, and was recently published in \textit{Ecography} --- this project was even included in earlier versions of this thesis.
I used this project as a testbed for new techniques and ideas, such as using Python for some spatial computations; and this was my first professional foray out of R programming.
It also helped keep me connected with ecological work with a focus on India.
Last year, in 2021, we began another exciting project to study changes in bird distributions in the Nilgiri Hills over the past 150 years using museum specimen data and modern survey work --- I look forward to contiuing that work as Vijay moves on to Cornell.

\noindent Greg Albery is a recent collaborator but someone whose work I've held in high regard for some time.
In early 2021, he tried to recruit me to his supervisor's lab and it was the first time that somebody had reached out to me in this way. 
Coming just after the pandemic winter of 2020, in the midst of the slow collapse of my collaboration with the NIOZ, and just as I was seriously considering (or worrying about) my future after my PhD, this was a massive confidence boost.
I was sounded out to work on building social networks from animal movement data, with a potential expansion into examining disease transmission; this gave me the idea for Chapter~\ref{ch:pathomove}, on which Greg is a co-author.
Greg is a prolific scientist and overall decent person who's and I'm very keen to be involved with his work at the newly N.S.F-funded Verena Institute for the study of emerging diseases.

\noindent I've had the help of many other external colleagues at all levels in academia, some of whom who appear as authors on chapters in this thesis, and others with whom I worked on projects outside my PhD.
I'm grateful for their help and acknowledge them here:
Orr Spiegel and Emmanuel Lourie, who helped substantially with feedback on the methods and guidance that I wrote into Chapter~\ref{ch:preprocessing};
Yosef Kiat, Ulrike Schl{\"a}gel, and Johannes Signer who vetted the bird biology and the statistical methods I used in Chapter~\ref{ch:holeybirds}, and Yoav Bartan who helped with getting the essential canopy height model;
Sivan Toledo who gave me feedback on all things ATLAS --- and also developed the system in the first place;
Amy Sweeny who ensured that Chapter~\ref{ch:pathomove} remained rooted in the biological reality of host-pathogen systems;
Morgan Tingley, V. V. Robin, and Ruth De Fries from Vijay's supervisory team who guided the \textit{eBird} project through three long years of development and review;
and Rebecca Rimbach, who was my supervisor during my time in South Africa, and who allowed me to practice my Python geospatial skills on two recent projects on urbanisation.

I worked with a smaller number of peers within the University, and I'd like to thank them for the time and effort that went into chapters that are part of this thesis:
Christoph Netz for all his work over the more than three years that Chapter~\ref{ch:kleptomove} has been in development, and for co-supervising both bachelor's and master's projects with me;
and Jakob Gismann for helping with the development of Chapter~\ref{ch:pathomove}, and for getting me involved in the analysis of fish social networks in the pond mesocosms.

Many other people helped make a difference to various projects --- appearing in this thesis and otherwise: Anat Levi, Adi Ben-Nun, Yotam Orchan, and Sivan Margalit in Israel; Mridula Mary Paul and a large number of citizen-scientist birdwatchers in India; Mark Adams and Henk Van Grouw in Tring; and Thijs Jansen, Hanno Hildenbrandt, Apu Ramesh, Selin Ersoy, Christine Beardsworth, Anne Dekinga, Frank Van Maarseveen, Bas Denissen, many field workers in the Dutch Wadden Sea, and the crew of the \textit{RV Navicula} in the Netherlands.

\medskip

Joining TRES in mid-2018, I was lucky to arrive at a time when the department was at full strength, with multiple labs full of intelligent and social people.
Although the pandemic severely impacted this dynamic, I want to thank everybody in the labs of the three main PIs, Franjo, Rampal, and Charlotte.
From Franjo's lab, Christoph, Jakob, Apu, and Stefano for lots of joint activities including badminton, but also Timo, Ines, and Jana for interesting discussions usually commencing around five o'clock;
from Rampal's lab, Theo, Raphael, Karen, Sebastian, Megan and Elisa for being a mainstay of my social life before the pandemic, as well as Shu for excellent dinners;
from Charlotte's lab, Marina and Lauren for fun pre- and post-pandemic events of all sorts;
and Boris and Jan, whom we adopted into Franjo's lab.
I especially want to thank Lucas, who was only with us for a year but made it an extremely fun year, from trying to convince us about snake jazz in Lisbon, to introducing me to \textit{Stranger Things} in Paris --- I'm looking forward to meeting again in London.

I also especially want to thank Pedro for being the bedrock of my social and work life for four years.
It was Pedro who introduced me to \textit{R} package development for example, and also the one who invited me and others out to Lisbon for New Year's in 2019, where I had a really lovely time.
Pedro is the wise elder everybody needs in their lives, and I think TRES is lucky to be able to keep him for a while longer.

I'm not really able to thank Josh enough, except to say that my life would be very different without him.
Josh convinced me to try many new things that I eventually came to enjoy, and it was on his suggestion that I seriously began considering the UK to live and work.
Josh is both very smart and very grounded, making him the first and last person I go to for advice; I'm overjoyed to have him around in this next chapter of my life, in London.

I also want to thank all my other friends, visiting and talking with whom helped me prioritise where I should take my life: Kevin, Thore, Paula, Amanda, Maria, Nick, Andr{\'e}, Merijn, and Norah from my master's; Ayushman, Vignesh, and Nehal from my time in India; and Connor, who I hope will be the first of new friends in England.

Finally, I'm grateful to lots of people for their help and support in what seems like a past life in India, while I was on the path to my PhD position: my parents and grandparents, and the many supervisors who offered me opportunities in their labs, both in Europe and India.

{ \begin{center} \barfont{-.-} \end{center} }

\endgroup
