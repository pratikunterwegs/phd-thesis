%*******************************************************
% Acknowledgments
%*******************************************************
% \pdfbookmark[1]{Acknowledgments}{acknowledgments}
\phantomsection
\addtocontents{toc}{\protect\vspace{\beforebibskip}}%
\addcontentsline{toc}{chapter}{\tocEntry{Reflections and Acknowledgments}}%
\chapter*{Reflections and Acknowledgments}\label{ch:ack}
\chaptermark{Reflections and Acknowledgments}

% \begin{flushright}{\slshape
%     We have seen that computer programming is an art, \\
%     because it applies accumulated knowledge to the world, \\
%     because it requires skill and ingenuity, and especially \\
%     because it produces objects of beauty.} \\ \medskip
%     --- \defcitealias{knuth:1974}{Donald E. Knuth}\citetalias{knuth:1974} \citep{knuth:1974}
% \end{flushright}

\bigskip

\begingroup
\let\clearpage\relax
\let\cleardoublepage\relax
\let\cleardoublepage\relax

Having read many thesis ``Acknowledgments'', the general pattern is to express a great sense of achievement, with thanks to family, friends, and supervisors.
Instead, I want to reflect on the PhD and the past four years generally.
People tend to rationalise their choices in hindsight, and this section may be no different.

It was probably not a good idea for me to have undertaken this PhD, perhaps a PhD at all.
I don't have the unbounded curiosity or creativity that defines `good' academics.
Rather, I should probably have been an engineer --- I enjoy getting things done.
I suppose I must say that I didn't know any better than to start a PhD, and that I wanted to conform socially with my friends, many of whom were starting PhDs, and that it did at the time represent a great deal of stability at a relatively high income.

I'm now convinced that I did not choose my PhD situation wisely.
Both of my `main' supervisors were in a state which I now recognise is not at all easy for their first few PhD students --- they were starting, or re-starting, their labs.
While one issue with a new lab is the relative inexperience of the supervisor, the more important one is actually the lack of a lab culture.
This includes accumulated wisdom that is invaluable to those receiving it; for those first uncovering it, it is often dearly won.
This is exactly the situation in which I found myself from beginning to end --- there was nobody around to show me the ropes, because there was nobody ahead of me to know them.

My lab also suffered --- and still does --- from being much too large. 
There were (and are) too many lines of research, all of them quite different.
I would have appreciated a smaller, more focused lab.
It was also very frustrating that my supervisors were pulling in wildly different directions (in a meeting, they once called each others' work ``irrelevant'').
There was a constant tension between where I was based (in a theoretical lab), and what I was good at (working with empirical data).

Then the pandemic happened.
In 2020, being securely employed for two more years was a huge advantage.
For that, I was, and remain, grateful; without the pandemic, though, I would have quit my PhD midway.

\paragraph*{Supervisors}

I've thought a great deal about whether to thank my supervisors --- this should already make it clear that I was less than pleased with how they operated.

First, I bear Ton Groothuis --- who was initially supposed to have been my second promotor, as well as Theunis, no ill will.
They were largely absent for most of my PhD, and I also did not seek them out.
I'm glad Ton did not insist that I should align with his interests, which don't overlap with mine at all.
However, I regret not working with Theunis because I think I would have found some satisfaction in the kind of work he does.

Second, I think Allert shouldn't be supervising anybody, let alone PhD students.
He was unreliable across contexts, quick to agree with senior people he respected, quick to rubbish ideas he didn't, and I never sensed a `big picture' in his work.
I felt he tried to hold me back to protect his less able student, and took advantage of my skills and generally helpful nature.
He appears in this thesis simply because it was considered too time-consuming to remove him.

In contrast, Ran Nathan joined my promotors' committee very late --- just a few days before I turned in my thesis.
The idea of including him grew on me slowly, and was cemented when I began collaborating with him directly in the summer of 2021, to replace the projects I lost when I stopped working with Allert.
I found Ran, even when not my supervisor, to be supportive and helpful, with exactly the sort of `big picture' view I had been lacking in relation to tracking data.

Finally \ldots

\paragraph*{Collaborators}

I had the good fortune to have worked with two ambitious and driven people.

Vijay Ramesh, then at Columbia University (now at Cornell) reached out to ask whether I would help him with some spatial data in late 2018.
I joined him in a 6-month project that ran for three and a half years --- see Chapter~\ref{ch:hillybirds}.
I used this project as a testbed for new techniques --- using Python for some computations, and ideas --- spatial thinning using a network approach, that caught my interest.

Greg Albery is among my most recent collaborators, and someone whose work I've held in high regard for a while.
In early 2021, Greg tried to recruit me to his supervisor's lab at Georgetown (he tried again in early 2022).
I was sounded out to work on building social networks from animal movement data, with a potential expansion into examining disease transmission.
This gave me the idea for Chapter~\ref{ch:pathomove}, during which I learned the movement and disease modelling that landed me my current job.

I look forward to working with both Vijay and Greg in the years to come.

I've had the help of many other collaborators at all levels in academia, who appear as authors on some of the chapters in this thesis, and on manuscripts yet to come: Orr Spiegel, Sivan Toledo, Yosef Kiat, Yoav Barton, Ulrike Schl{\"a}gel, Johannes Signer, Mark Adams, Rebecca Rimbach, Mridula Paul, Morgan Tingley, VV Robin, Ruth de Fries, and Amy Sweeny.

\paragraph*{TRES and Surrounds}

Joining TRES in mid-2018, I found a department that, pre-pandemic, was very similar to the Centre for Ecological Sciences I was coming from --- full of intelligent people, and most importantly, social.
I will freely admit to not finding the majority of the work in this department interesting, and that was a function of the diversity, or divergence, of topics among and within labs.
Yet the general gregariousness of the PhD students who were my colleagues more than made up for this.
I now think departments with better social than professional ties are probably healthier workplaces.

A number of people here 

I'm quite sure my life would be very different without Josh Lambert.
I would never have tried a number of things I now enjoy without Josh suggesting them, and often, accompanying me: squash, long-distance cycling, programming in Julia, considering the UK to live and work.
I also made a more serious change, of which Josh convinced me: to eventually leave the academic track.
I'm immensely pleased that I was able to find a cluster-hire at the London School of Hygiene and Tropical Medicine, through which we were both offered positions.
Josh is both very smart and grounded, and he's the first and last person I go to for advice --- I look forward to having him around.

\endgroup
