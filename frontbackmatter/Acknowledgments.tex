%*******************************************************
% Acknowledgments
%*******************************************************
% \pdfbookmark[1]{Acknowledgments}{acknowledgments}
\addchap{Acknowledgments}\label{ch:ack}
% \chaptermark{Reflections and Acknowledgments}

% \begin{flushright}{\slshape
%     We have seen that computer programming is an art, \\
%     because it applies accumulated knowledge to the world, \\
%     because it requires skill and ingenuity, and especially \\
%     because it produces objects of beauty.} \\ \medskip
%     --- \defcitealias{knuth:1974}{Donald E. Knuth}\citetalias{knuth:1974} \citep{knuth:1974}
% \end{flushright}

\bigskip

\begingroup
\let\clearpage\relax
\let\cleardoublepage\relax
\let\cleardoublepage\relax

I would like to write down a few acknowledgments for people who have had a substantial role over the course of my PhD.

First I'd like to acknowledge the role of my supervisory team, starting with those who were initially on the team in 2018, but are no longer.
Ton Groothuis and Theunis Piersma were initially involved, but as my PhD veered further away from its planned form, they understandably dropped out.
I'm grateful that neither of them insisted that I should stick to the original structure of the PhD, and regret not working on more applied projects with Theunis.
Allert Bijleveld, then a newly minted PI from the NIOZ, was also initially involved and I was supposed to collaborate closely with one of his other PhD students.
Unfortunately, I found him to be an unreliable and stubborn supervisor with little big picture thinking to his work.
I had three very frustrating years trying to work with him and his lab, before I cut myself off from the collaboration; I'm glad that he too recently withdrew as a co-promotor.

Second and much more positively, Ran Nathan joined my supervisory team very late, just a few days before I turned in my thesis.
Ran had actually been involved with my work since 2020, and we had met at conferences in 2018 and 2019.
It was entirely by chance that I presented my work in methods development at an online workshop hosted by his lab; this prompted Ran to suggest the writing of what eventually became Chapter~\ref{ch:preprocessing} of this thesis.
Later, he invited me to be part of a large author team working on a review in \textit{Science}, which was very exciting.
I began collaborating with Ran directly in the summer of 2021, to replace the projects I closed when I stopped working with Allert.
This project, now Chapter~\ref{ch:holeybirds}, went better than I could have hoped, and that led me to suggest including him as a supervisor and indeed as a `promotor', which is a special role in the Dutch academic system.
I found Ran, even when he was not yet my supervisor, to be warm, supportive and helpful, with exactly the sort of big picture thinking I had been lacking in relation to animal tracking.
I'm grateful that he accepted a role as one of my promotors, and indeed that he initially agreed to work with me; I look forward to working with him in the future.

Finally, I must acknowledge Franjo Weissing who has been my main PhD supervisor, and is now my first promotor and the sole survivor of the original supervisory team.
I had actually been in touch, indirectly at least, with Franjo since 2015, when I had turned down an offer to join the MEME master's programme that he established.
I have mixed feelings about accepting this position, to which I did not actually apply.
On the one hand I am neither a theoretician nor an evolutionary biologist and might have done better sticking to ecological data; on the other hand I picked up an interesting set of skills here that I might not have done in a data-focused lab.
I also thought Franjo's lab was too conceptually dispersed for there to be much peer support for my work.
This effect was exacerbated by my being one of the first people in the newly reconstituted lab, and also because I initially worked with data (through the NIOZ collaboration) more than my colleagues.
I appreciated that Franjo allowed me to continue working on movement data, even though at the time, there was no apparent link with any theoretical work.
In general, I'm grateful that he allowed me to do pretty much as I wanted with my PhD, and for my part I tried to make sure that I was as productive as possible.
I found Franjo to be a supportive and caring supervisor, and I'm grateful that he decided to hire me; I hope our work together will help establish his approach to eco-evolutionary theory.

\medskip

I've had the good fortune to have worked with two ambitious and driven people; both are early career researchers who are doing extremely well for themselves scientifically, and I'm sure both will go very far.

Vijay Ramesh, then at Columbia University (and now at Cornell's Lab of Ornithology) reached out to me in late 2018.
He was trying to use data from \textit{eBird} to study birds in the southern Western Ghats of India, and asked whether I would help him with some spatial data analysis.
I joined him in a six month project that ran for three and a half years, and was recently published in \textit{Ecography}.
I used this project as a testbed for new techniques and ideas, such as using Python for some spatial computations; and this was my first professional foray out of R programming.
Last year, in 2021, we began work on another exciting project to study changes in bird distributions in the Nilgiri Hills over the past 150 years using museum specimen data.

Greg Albery is a recent collaborator, but someone whose work I've held in high regard for some time now.
In early 2021, Greg tried to recruit me to his supervisor's lab at Georgetown University; this was a massive confidence boost just as my collaboration with the NIOZ was failing.
I was sounded out to work on building social networks from animal movement data, with a potential expansion into examining disease transmission; this gave me the idea for Chapter~\ref{ch:pathomove}, on which he's a co-author.
Along the way, I picked up the skills in coding simulations for animal movement and pathogen transmission that were no doubt invaluable in landing me my current job.

I'm grateful that both Vijay and Greg reached out to me when they did, and I look forward to working with them in the years to come.

I've had the help of many other collaborators at all levels in academia, who appear as authors on some of the chapters in this thesis, and on manuscripts yet to come: Orr Spiegel, Sivan Toledo, Yosef Kiat, Yoav Barton, Ulrike Schl{\"a}gel, Johannes Signer, Mark Adams, Rebecca Rimbach, Mridula Paul, Morgan Tingley, VV Robin, Ruth de Fries, and Amy Sweeny.

\paragraph*{TRES and Surrounds}

Joining TRES in mid-2018, I found a department that, pre-pandemic, was very similar to the Centre for Ecological Sciences I was coming from --- full of intelligent people, and most importantly, social.
I will freely admit to not finding the majority of the work in this department interesting, and that was a function of the diversity, or divergence, of topics among and within labs.
Yet the general gregariousness of the PhD students who were my colleagues more than made up for this.
I now think departments with better social than professional ties are probably healthier workplaces.

A number of people here 

I'm quite sure my life would be very different without Josh Lambert.
I would never have tried a number of things I now enjoy without Josh suggesting them, and often, accompanying me: squash, long-distance cycling, programming in Julia, considering the UK to live and work.
I also made a more serious change, of which Josh convinced me: to eventually leave the academic track.
I'm immensely pleased that I was able to find a cluster-hire at the London School of Hygiene and Tropical Medicine, through which we were both offered positions.
Josh is both very smart and grounded, and he's the first and last person I go to for advice --- I look forward to having him around.

\endgroup
