%*******************************************************
% Acknowledgments
%*******************************************************
\pdfbookmark[1]{Acknowledgments}{acknowledgments}

% \begin{flushright}{\slshape
%     We have seen that computer programming is an art, \\
%     because it applies accumulated knowledge to the world, \\
%     because it requires skill and ingenuity, and especially \\
%     because it produces objects of beauty.} \\ \medskip
%     --- \defcitealias{knuth:1974}{Donald E. Knuth}\citetalias{knuth:1974} \citep{knuth:1974}
% \end{flushright}

\bigskip

\begingroup
\let\clearpage\relax
\let\cleardoublepage\relax
\let\cleardoublepage\relax
\chapter*{Acknowledgments}

Having read many thesis ``Acknowledgments'', the general pattern is to express a great sense of achievement, with with tearful thanks to family, friends, and supervisors.

\noindent I am a pessimist (though not by choice), so this is not one of those.

It was probably not a good idea for me to have undertaken this PhD, perhaps a PhD at all.
I don't have the unbounded curiosity or creativity that defines `good' academics.
Rather, I should probably have been an engineer: I enjoy getting things done.

My lab also suffered --- and still does --- from being much too large. 
On top of that, there were (and are) too many lines of research, all of them quite different.
I would have appreciated a smaller, more focused lab.

It was also very frustrating that my supervisors were pulling in wildly different directions (in a meeting, they once called each others' work ``irrelevant'').
There was a constant tension between where I was based (in a theoretical lab), and what I was good at (working with empirical data).

Then the pandemic happened.
In 2020, being securely employed for two more years was a huge advantage.
For that, I was grateful; without the pandemic, though, I would have quit my PhD midway.

\paragraph*{My supervisors}

I'm not sure whether to thank my supervisors.

I didn't actually apply for this position; I was offered it because nobody suitable could be found.
This `last resort' status troubled me for quite some time.



I'd especially like to thank Rebecca Rimbach, with whom I've worked on and off since 2014, when we first met in Goegap, South Africa, where she was my supervisor.
Rebecca recommended me for both my master's and PhD, and it was after tracking radio-tagged mice with a hand-held GPS device, that I became more interested in working with movement loggers and data.

\medskip

I had the good fortune to have worked with two very ambitious and driven people --- these are qualities I prize above many others --- I look forward to working with them in the years to come.
Both have 

Special thanks are due to Vijay Ramesh, with whom I've worked, entirely remotely, since 2019.
After interacting over some R related issues on Twitter, Vijay asked me to join him as the main programmer for a 6-month project that's been running for 3.5 years now.
I used this project as a testbed for new techniques, ideas, and methods that I was interested in and didn't have any other venue to try.

I also want to thank Greg Albery, among my most recent collaborators, and someone whose work I've held in high regard for a while.
It was among my highlights of 2021 when Greg tried to recruit me to his supervisor's lab --- it was a great boost to my confidence at a difficult time.
Meeting in person at a conference in December 2021 helped me to see the value of my own work, with which he is now involved.

Thanks also to all my other collaborators: Orr Spiegel, Sivan Toledo, Yosef Kiat, Yoav Barton, and Amy Sweeny.

\medskip

Most of all I want to thank Josh Lambert.
In three short years, Josh has managed to change me more than anybody else.
I would never have tried a number of things I now enjoy without Josh suggesting them, and often, accompanying me: squash, long-distance cycling, programming in Julia, or baking cakes.
I also accepted a more serious change that had been unthinkable until Josh convinced me that I could successfully see it through: that I would eventually leave academia.
Josh is very smart and grounded, and he's the first person I go to for advice.

\endgroup
